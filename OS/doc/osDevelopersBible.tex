\documentclass[11pt]{book}
\usepackage{graphics,graphicx}
\usepackage[auto]{chappg}
%\usepackage[dvips]{graphics,graphicx}
\DeclareGraphicsExtensions{.ps,.jpg,.eps,.pdf,.png}
\usepackage{boxedminipage,amsmath,amsfonts}
\usepackage{url}
%\usepackage{secdot}
%\usepackage{natbib}
\usepackage{verbatim}
%\usepackage{moreverb}
\usepackage{enumerate}
\usepackage{makeidx}
\usepackage{own}
\bibliographystyle{plain}
\makeindex

   

%%%%%
% some other macros
\newcommand{\figurepath}{./figures}
\newcommand{\bibpath}{/Users/kmartin/Documents/files/misc}
\newcommand{\figfiletype}{pdf}

%Brad Bell Macros

% Latex macros defined for all the CppAD documentation:
\newcommand{\T}{ {\rm T} }
\newcommand{\R}{ {\bf R} }
\newcommand{\C}{ {\bf C} }
\newcommand{\D}[2]{ \frac{\partial #1}{\partial #2} }
\newcommand{\DD}[3]{ \frac{\partial^2 #1}{\partial #2 \partial #3} }
\newcommand{\Dpow}[2]{ \frac{\partial^{#1}}{\partial  {#2}^{#1}} }
\newcommand{\dpow}[2]{ \frac{ {\rm d}^{#1}}{{\rm d}\, {#2}^{#1}} }

% Define the hangref environment used for the References list:
\newenvironment{hangref}
  {\begin{list}{}{\setlength{\itemsep}{4pt}
  \setlength{\parsep}{0pt}\setlength{\leftmargin}{+\parindent}
  \setlength{\itemindent}{-\parindent}}}{\end{list}}

% Set the page margins to 1 inch all around:
\marginparwidth 0pt\marginparsep 0pt \topskip 0pt\headsep
0pt\headheight 0pt \oddsidemargin 0pt\evensidemargin 0pt
\textwidth 6.5in \topmargin 0pt\textheight 9.0in
\newtheorem{theorem}{Theorem}

   
%%%%Added by Leo%%%%
\newcounter{Fig}
\renewcommand{\theFig}{\arabic{Fig}}
\newcommand{\Fig}[2]{\refstepcounter{Fig} \label{#1}
                     {\small\bf Figure \theFig.} {\small\sl #2 \par}}

\setcounter{topnumber}{3}
\renewcommand{\topfraction}{.9}
\setcounter{bottomnumber}{3}
\renewcommand{\bottomfraction}{.9}
\setcounter{totalnumber}{4}
\renewcommand{\textfraction}{.1}
\setlength{\floatsep}{.25in}
\setlength{\intextsep}{.25in}

\setlength{\fboxrule}{2\fboxrule} \setlength{\fboxsep}{3\fboxsep}

\newcommand{\Sa}{8pt}
\newcommand{\Sb}{0pt}

\renewcommand{\_}{{\char"5F}}
\renewcommand{\{}{{\char"7B}}
\renewcommand{\}}{{\char"7D}}
\renewcommand{\^}{{\char"0D}}

\let\accute= \'
\renewcommand{\'}{{\char"0D}}

\newcommand{\bfit}{\bfseries\itshape}

\newlength{\extopskip} \newlength{\exbottomskip}
\setlength{\exbottomskip}{1\baselineskip}
\addtolength{\exbottomskip}{-5.0pt}
\setlength{\extopskip}{1\exbottomskip}
\addtolength{\extopskip}{-1\parskip}

\newenvironment{Example}{\vspace{1\extopskip}\noindent\hspace*{2em}
                         \frenchspacing\small
                         \tt\begin{tabular}{@{}l@{}}}{
                         \end{tabular}\\[1\exbottomskip]}

\newcommand{\Titem}{\item[$\triangleright$]}
\newcommand{\Ditem}{\item[$\diamond$]}

\newenvironment{Itemize}{\begin{quote}\normalsize
   \baselineskip 20pt plus .3pt minus .1pt \begin{itemize}}
   {\end{itemize}\end{quote}}
   % Set path to folder containing figures
\newcommand{\FigureFolder}{figures}

\newif\ifknitro \knitrofalse    % change to \knitrotrue once we get knitro connected again
\newif\ifipopt  \ipopttrue      % change to \ipopttrue  once we get the build problems sorted out

% We use a number of URLs that point to software downloads. These locations are forever changing,
% and maintaining them is a nightmare. All the URLs are gathered here, so that we can at least
% get away with making changes only once...
\newcommand{\OSSolverServer}{http://74.94.100.129:8080/OSServer/services/OSSolverService}
\newcommand{\UrlAmpl}{http://www.ampl.com}
\newcommand{\UrlAmplSandia}{http://netlib.sandia.gov/ampl/}
\newcommand{\UrlAmplSolvers}{http://netlib.sandia.gov/cgi-bin/netlib/netlibfiles.tar?filename=netlib/ampl/solvers}
\newcommand{\UrlApacheFileupload}{http://jakarta.apache.org/commons/fileupload/}
\newcommand{\UrlBlas}{ftp://www.netlib.org/blas/blas.tgz}
\newcommand{\UrlBonmin}{http://projects.coin-or.org/Bonmin}
\newcommand{\UrlBuildtools}{http://projects.coin-or.org/BuildTools}
\newcommand{\UrlCbc}{http://projects.coin-or.org/Cbc}
\newcommand{\UrlCgl}{http://projects.coin-or.org/Cgl}
\newcommand{\UrlCl}{http://www.microsoft.com/express/download/\#webInstall}
\newcommand{\UrlClp}{http://projects.coin-or.org/Clp}
\newcommand{\UrlCoinAllBinaryPdf}{ https://projects.coin-or.org/svn/OS/trunk/OS/doc/UsingCoinAllBinary.pdf}
\newcommand{\UrlCoinAllDownload}{http://www.coin-or.org/download/binary/CoinAll/}
\newcommand{\UrlCoinAllLibsPdf}{ https://projects.coin-or.org/svn/OS/trunk/OS/doc/UsingCoinAllLibs.pdf}
\newcommand{\UrlCoinBinary}{http://projects.coin-or.org/CoinBinary}
\newcommand{\UrlCoinConfig}{https://projects.coin-or.org/BuildTools/wiki/user-configure\#PreparingtheCompilation}
\newcommand{\UrlCoinConfigure}{http://projects.coin-or.org/BuildTools/wiki/user-configure\#CommandLineArgumentsforconfigure}
\newcommand{\UrlCoinCygwin}{http://projects.coin-or.org/BuildTools/wiki/current-issues}
\newcommand{\UrlCoinDownload}{http://projects.coin-or.org/BuildTools/wiki/user-download\#DownloadingtheSourceCode}
\newcommand{\UrlCoinEasy}{http://projects.coin-or.org/CoinEasy}
\newcommand{\UrlCoinNames}{https://projects.coin-or.org/CoinBinary/wiki/ArchiveNamingConventions}
\newcommand{\UrlCoinProjects}{http://www.coin-or.org/projects/}
\newcommand{\UrlCoinUtils}{http://projects.coin-or.org/CoinUtils}
\newcommand{\UrlCouenne}{http://projects.coin-or.org/Couenne}
\newcommand{\UrlCplex}{http://www.ilog.com/products/cplex/}
\newcommand{\UrlCppad}{http://projects.coin-or.org/CppAD}
\newcommand{\UrlCygwinMake}{http://www.cmake.org/files/cygwin/make.exe}
\newcommand{\UrlCygwinSetup}{http://www.cygwin.com/setup.exe}
\newcommand{\UrlDoxygen}{http://www.doxygen.org}
\newcommand{\UrlDylp}{http://projects.coin-or.org/DyLP}
\newcommand{\UrlEpl}{http://www.eclipse.org/legal/epl-v10.html}
\newcommand{\UrlFToC}{http://www.netlib.org/f2c}
\newcommand{\UrlFToCBin}{http://www.netlib.org/f2c/mswin/}
\newcommand{\UrlFToCZip}{http://www.netlib.org/f2c/libf2c.zip}
\newcommand{\UrlGgs}{http://www.g95.org}
\newcommand{\UrlGamslinks}{https://projects.coin-or.org/svn/GAMSlinks/trunk}
\newcommand{\UrlGcc}{http://gcc.gnu.org}
\newcommand{\UrlGfortran}{http://gcc.gnu.org/fortran/}
\newcommand{\UrlGlpk}{http://www.gnu.org/software/glpk/}
\newcommand{\UrlGlpkDownload}{ftp://ftp.gnu.org/gnu/glpk/glpk-4.42.tar.gz}
\newcommand{\UrlHsl}{http://www.cse.scitech.ac.uk/nag/hsl/}
\newcommand{\UrlIpopt}{http://projects.coin-or.org/Ipopt}
\newcommand{\UrlIpoptDoc}{https://projects.coin-or.org/Ipopt/browser/stable/3.5/Ipopt/doc/documentation.pdf?format=raw}
\newcommand{\UrlIpoptDocxiii}{http://www.coin-or.org/Ipopt/documentation/node13.html}
\newcommand{\UrlKippFileupload}{http://gsbkip.chicagogsb.edu/os/fileupload.html}
\newcommand{\UrlKnitro}{http://www.ziena.com/}
\newcommand{\UrlKnitroMan}{http://www.ziena.com/docs/knitroman.pdf}
\newcommand{\UrlLapack}{ftp://www.netlib.org/lapack/lapack-lite-3.1.0.tgz}
\newcommand{\UrlMingw}{http://downloads.sourceforge.net/mingw/MSYS-1.0.11.exe?modtime=1079444447\&big\_mirror=1}
\newcommand{\UrlMsys}{http://sourceforge.net/project/showfiles.php?group\_id=2435}
\newcommand{\UrlMsysBinary}{http://downloads.sourceforge.net/mingw/bash-3.1-MSYS-1.0.11-1.tar.bz2?modtime=1195140582\&big\_mirror=1}
\newcommand{\UrlMsysAddIns}{http://sourceforge.net/project/showfiles.php?group\_id=2435\&package\_id=67879}
\newcommand{\UrlMsysBison}{bison-2.3-MSYS-1.0.11-1.tar.bz2}
\newcommand{\UrlMsysFlex}{flex-2.5.33-MSYS-1.0.11-1.tar.bz2}
\newcommand{\UrlMsysRegex}{regex-0.12-MSYS-1.0.11-1.tar.bz2}
\newcommand{\UrlNewticket}{http://projects.coin-or.org/OS/newticket}
\newcommand{\UrlNightlyBuild}{https://projects.coin-or.org/TestTools/wiki/NightlyBuildInAction}
\newcommand{\UrlOs}{http://www.optimizationservices.org}
\newcommand{\UrlOsBinaries}{http://www.coin-or.org/download/binary/OS/}
\newcommand{\UrlOsCommon}{https://projects.coin-or.org/svn/OS/branches/OScpp/OSCommon}
\newcommand{\UrlOsDoxygen}{http://www.coin-or.org/OS/doxydoc/html/index.html}
\newcommand{\UrlOsi}{http://projects.coin-or.org/Osi}
\newcommand{\UrlOsJava}{https://projects.coin-or.org/svn/OS/branches/OSjava}
\newcommand{\UrlOsRelease}{https://projects.coin-or.org/svn/OS/releases/2.3.0}
\newcommand{\UrlOsStable}{https://projects.coin-or.org/svn/OS/stable/2.3}
\newcommand{\UrlOsTarball}{http://www.coin-or.org/download/source/OS/}
\newcommand{\UrlOsWiki}{http://projects.coin-or.org/OS/}
\newcommand{\UrlOsWin}{https://projects.coin-or.org/CoinBinary/browser/binary/OS}
\newcommand{\UrlParinclinear}{http://www.coin-or.org/OS/parincLinear.osil}
\newcommand{\UrlSdk}{http://www.microsoft.com/downloads/details.aspx?FamilyID=E6E1C3DF-A74F-4207-8586-711EBE331CDC\&displaylang=en}
\newcommand{\UrlSvn}{http://subversion.tigris.org}
\newcommand{\UrlSymphony}{http://projects.coin-or.org/SYMPHONY}
\newcommand{\UrlTomcat}{http://tomcat.apache.org/}
\newcommand{\UrlTortoiseSvn}{http://tortoisesvn.tigris.org}
\newcommand{\UrlTrac}{http://projects.coin-or.org/OS}
\newcommand{\UrlUsingTrac}{http://www.coin-or.org/usingTrac.html}
\newcommand{\UrlVol}{http://projects.coin-or.org/Vol}
\newcommand{\UrlWget}{http://www.christopherlewis.com/WGet/WGetFiles.htm}
\newcommand{\UrlWgetBinary}{http://www.christopherlewis.com/WGet/wget-1.11.4b.zip}


% Current software versions
\newcommand{\OSstable}{2.5}
\newcommand{\OSrelease}{2.5.0}
\newcommand{\OStrunk}{4490}
\newcommand{\MsysVer}{1.0.11}
\newcommand{\MsysFile}{bash-3.1-MSYS-1.0.11}
\newcommand{\GlpkVer}{4.42}



% Some logicals for allowing the build of  different configurations of the material with different content  
\newif\ifruncode   % to build a manual for folks who only want to run the executables 
\newif\ifuselibs   % to build a manual for folks who want to build against the libraries
\newif\ifdevelop   % to build a manual for folks who want to actively develop code
\newif\ifbible    % to build a manual for project managers

\runcodefalse \uselibsfalse \developfalse \biblefalse

 
\topmargin 0pt \headheight 14pt \headsep 8pt \textheight 8.6in


\begin{document}

\runcodefalse   % we do not build a manual for folks who only want to run the executables
\uselibsfalse   % we do not build a manual for folks who want to build against the libraries
\developtrue   % we build a manual for folks who want to actively develop code
\bibletrue     % in addition we include stuff that should only be of interest to project managers

% we can print this bible in regular format (with each chapter starting on a right-hand page)
% or in condensed format, two pages side-by-side on a landscape sheet
\newif\iftwoup
\twoupfalse % regular format --- each chapter starts on a right-hand page 
%\twouptrue % two pages side-by-side in landscape format (paper-conserving) 

\newcommand{\throwpage}{\iftwoup\cleardoublepage\else\clearpage\fi}

%Some cross-referencing index entries (to be expanded as needed)
\index{AMPL Solver Library |see{Third-party software, ASL}}
\index{ASL|see{Third-party software, ASL}}
\index{Blas|see{Third-party software, Blas}}
\index{Harwell Subroutine Library|see{Third-party software, HSL}}
\index{HSL|see{Third-party software, HSL}}
\index{Lapack|see{Third-party software, Lapack}}
\index{Mumps|see{Third-party software, Mumps}}

\index{Bonmin@{\tt Bonmin}|see{COIN-OR projects, {\tt Bonmin}}}
\index{BuildTools@{\tt BuildTools}|see{COIN-OR projects, {\tt BuildTools}}}
\index{Cbc@{\tt Cbc}|see{COIN-OR projects, {\tt Cbc}}}
\index{Cgl@{\tt Cgl}|see{COIN-OR projects, {\tt Cgl}}}
\index{Clp@{\tt Clp}|see{COIN-OR projects, {\tt Clp}}}
%\index{Configuration Manager|see{Microsoft Visual Studio, Configuration Manager}}
\index{Couenne@{\tt Couenne}|see{COIN-OR projects, {\tt Couenne}}}
\index{CppAD@{\tt CppAD}|see{COIN-OR projects, {\tt CppAD}}}
\index{CoinUtils@{\tt CoinUtils}|see{COIN-OR projects, {\tt CoinUtils}}}
%\index{Debug configuration|see{Microsoft Visual Studio, {\tt Debug} configuration}}
\index{DyLP@{\tt DyLP}|see{COIN-OR projects, {\tt DyLP}}}
\index{GLPK@{\tt GLPK}|see{Third-party software, {\tt GLPK}}}
\index{Ipopt@{\tt Ipopt}|see{COIN-OR projects, {\tt Ipopt}}}
\index{nl files|see{AMPL nl format}}
\index{Osi@{\tt Osi}|see{COIN-OR projects, {\tt Osi}}}
%\index{Release configuration|see{Microsoft Visual Studio, {\tt Release} configuration}}
%\index{Release-plus configuration|see{Microsoft Visual Studio, {\tt Release-plus} configuration}}
\index{SYMPHONY@{\tt SYMPHONY}|see{COIN-OR projects, {\tt SYMPHONY}}}
\index{Vol@{\tt Vol}|see{COIN-OR projects, {\tt Vol}}}


\pagenumbering{roman}

\title{Optimization Services \OSstable\ User's Manual (Insider edition)}
\vskip 2in
\author{Horand Gassmann, Jun Ma,  Kipp Martin, and Wayne Sheng}
\maketitle

\chapter*{\centering \begin{normalsize}Abstract\end{normalsize}}
\begin{quotation}
\noindent
This is the extended User's Manual for the Optimization Services (OS) project.  The objective of OS is to provide a
general framework consisting of a set of standards for representing optimization instances, results,
solver options, and communication between clients and solvers in a distributed environment using Web Services.
This COIN-OR\index{COIN-OR} project provides C++ and Java source code for libraries and executable programs that 
implement OS standards.   The OS library includes a robust solver and modeling language interface (API) for linear,
nonlinear and other types of optimization problems.   Also included is the C++ source code for a  command line
executable {\tt OSSolverService}\index{OSSolverService@{\tt OSSolverService}}  for reading problem instances 
(OSiL format\index{OSiL}, nl format\index{AMPL nl format}, MPS format\index{MPS format}) and
calling a solver either locally or on a remote server.  Finally,  both Java\index{Java} source code and a Java {\tt war} 
file are provided for users who wish to set up a solver service on a server running Apache Tomcat\index{Apache Tomcat}.
See the Optimization Services home page {\tt\UrlOs} and the COIN-OR Trac page\index{Trac system} {\tt\UrlTrac} for 
more information.
\end{quotation}
\clearpage


\tableofcontents
\listoffigures
\listoftables
\hyphenation{com-plex-Type}
\hyphenation{GAMS-links}


%\noindent\hrulefill
\throwpage

\pagenumbering{gobble}

\part{Overview}

\pagenumbering{bychapter}

\input{chapters/OSConstitutionArticles.tex}

Categories:

1) General philosophy

2) Design principals 

3) Guidelines / Implementation

%%%%%%%%%%%%%%


I read the OSConstitution.
Before that I deliberately did not read it, that way, I am not limited or affected by seeing it and we can combine into a more comprehensive one.
After I compared mine with yours, I see that it's good that mine is not that much overlapping with yours.
You mainly concentrate on the design principle, while mine is more on the intent, goal and definition plus some design principles.
I didn't write it in the tex file, because it's some informal ideas for the constitution.
Here it is:
-----------------
Optimization Services (OS): Providing "service" to facilitate optimization, optimization as (web) 'service", goal is like utility "service", for open source, academic and commercial use
(OS should be positioned like "a" [meaning one of several] computational infrastructure for Operations Research (OR) and a Operations Research (OR) Internet kind of thing).


OSFramework is Platform independent (OS, programming language, hard ware system)
OS Standards and OSP: OSP is a standard set of "XML" and "SOA" based "application" protocol for optimization computing,
    Includes subprotocol of representation, communication, registration and discovery.

OS System is the implementation of OS Framework with all the OS compatible OS components, can be a local system or a distributed system.
OSComponents on the OS System: OS libraries, instances, model, modeling system, solvers, analyzers, simulation for optimization, communication agents and interface, repositotories, optimization servers/registries, preprocessors,

OS Library is the set of OS compatible libraries that facilitate OS Components to be built on the OS System

OS Customers: optimization users, applications and systems that involve optimization, modeler,  optimization framework, standard and system researchers and developers, optimization component researchers and developers

OS Standardization: staging process is experiment -> draft -> proposal -> recommendation -> finalization -> version 1.0, 1.1 2.0

OS License: any license (e.g. CPL) that follows the OS constitution. Derived research, development and business model, commercial or non-commercial, can be carried out and built upon OS.

Design principles:
No platform specific features.
Should make effort to keep stability of standards.
Cleanly built from scratch, with both the bottom-up and top-down approach, e.g. thinking about the whole picture while building a small components.
By the whole picture, it means different current and future optimization application, types, and domains, in inter-disciplinary interaction between different OSP sub-protocols and OS components.
Implementation/prototyping is suggested to be carried out before standard recommendation.
Conservative approach should be taken when modifying existing standards. Each major version should be made backward compatible for that version.
Separation of concern/functionality
Extensibility first, but with no scalability issue
Computer interfacing first, but without sacrificing user experience, e.g. readability, honoring original model intent
-----------------

Here is an excerpt from the OS Thesis on OS definitions:

Optimization Services is a framework that specifies how a set of cooperative classes and interfaces
should be designed and implemented in order to solve an optimization problem. The Optimization Services
framework has the following properties:
. It consists of multiple classes or components, each of which may provide an abstraction of some
particular optimization concept.
. It defines how these abstractions work together to solve an optimization problem.
. Its optimization-related components are reusable, which is what makes Optimization Services a good
framework, since it provides generic behavior that many different types of OR applications can use.
. It organizes patterns at a higher level. By "pattern" we mean a tried and true way to deal with an
optimization process, from the whole context to the problem and to the final solution that appears over
and over again. Thus the adopted patterns in the Optimization Services is an effective means of
communication between OR software components, therefore bringing order into chaos.
There is a key difference between a library and a framework. A library contains functions or routines
that an application or a user can invoke. A framework provides generic, cooperative components that
software can follow and extend. The Optimization Services framework provides a foundation upon which OR applications, software, and libraries
are built, whereas an OR library is a piece of software used by other OR applications.



Jun



%%%%%%%%%%%%%



\throwpage


\section{The Optimization Services (OS) Project}

The objective of Optimization Services (OS) is to provide a general framework consisting of a set of standards
for representing optimization instances, results, solver options, and communication between clients and solvers
in a distributed environment using Web Services. This COIN-OR project provides source code for libraries and
executable programs that implement OS standards.  See the COIN-OR Trac page {\tt\UrlTrac}\index{Trac system}
or the Optimization Services Home Page {\tt\UrlOs}\index{Optimization Services} for more information.

Like other COIN-OR projects, OS has a versioning system that ensures end users some degree of stability 
and a stable upgrade path as project development continues. The current stable version of OS is \OSstable, 
and the current stable release is \OSrelease\index{OS project!stable release}, based on trunk version~\OStrunk.

\ifruncode
This document provides descriptions for the following components of the OS project:
\else
The OS project provides the following:
\fi

\begin{enumerate}
\item{}  A set of XML\index{XML} based standards for representing optimization instances (OSiL)\index{OSiL}, 
optimization results (OSrL)\index{OSrL}, and optimization solver options (OSoL)\index{OSoL}. 
There are other standards, but these are the main ones. 
The schemas for these standards are described in Section~\ref{section:schemadescriptions}.

\ifruncode\else
\item{}  Open source libraries  that support and implement many of the standards.

\item{}  A robust solver and modeling language interface (API) for linear and nonlinear optimization problems.
Corresponding to the OSiL problem instance representation there is an in-memory object,
{\tt OSInstance}\index{OSInstance@{\tt OSInstance}},
along with a collection of  {\tt get()},   {\tt set()}, and {\tt calculate()} methods for accessing and creating
problem instances. This is a very general API for linear, integer, and nonlinear programs.
Extensions for other major types of optimization problems are also in the works. Any modeling language that can
produce OSiL can easily communicate with any solver that uses the OSInstance API.   
The {\tt OSInstance}\index{OSInstance@{\tt OSInstance}} object
is described in more detail in Section~\ref{section:osinstanceAPI}. The nonlinear part of the API is based on the
COIN-OR project CppAD\index{COIN-OR projects!CppAD@{\tt CppAD}} by Brad Bell ({\tt\UrlCppad}) but is written 
in a very general manner and could be used with other algorithmic differentiation packages. More detail on 
algorithmic differentiation is provided in Section~\ref{section:ad}.
\fi

\item{}  A  command line executable {\tt OSSolverService}\index{OSSolverService@{\tt OSSolverService}}  for reading
problem instances (OSiL format\index{OSiL}, AMPL  nl format\index{AMPL nl format},  
MPS format\index{MPS format}) and calling a solver either locally or on a remote server.
This is described in Section~\ref{section:ossolverservice}.


\item{} Utilities that convert AMPL nl files  and MPS files into the OSiL XML format.
This is described in Section~\ref{section:osmodelinterfaces}.


\item{}  Standards that facilitate the communication between clients and optimization solvers using Web Services.
\ifruncode\else
In  Section~\ref{section:osagent} we describe the {\tt OSAgent}\index{OSAgent@{\tt OSAgent}} part of the OS library
that is used to create Web Services SOAP\index{SOAP protocol} packages with OSiL instances and contact a server for 
solution.
\fi

\item{}  An executable program {\tt OSAmplClient}\index{OSAmplClient@{\tt OSAmplClient}} that is designed to work with 
the AMPL\index{AMPL} modeling language. The {\tt OSAmplClient} appears as a ``solver'' to AMPL and, based on options 
given in AMPL, contacts solvers either remotely or locally to solve instances created in AMPL. This is described in
Section~\ref{section:amplclient}.

\ifruncode\else
\item{}  Server software that works with Apache Tomcat\index{Apache Tomcat} and Apache Axis\index{Apache Axis}.
This software uses Web Services technology and acts as middleware between the client that creates the instance
and the  solver on the server that optimizes the instance and returns the result. This is illustrated in
Section~\ref{section:tomcat}.

\item{}  A lightweight version of the project, {\tt OSCommon},\index{OSCommon@{\tt OSCommon}} for modeling language and 
solver developers who want to use OS API, readers and writers, without the overhead of other COIN-OR projects or any 
third-party software. For information on how to download {\tt OSCommon} see Section~\ref{section:oslite}.
\fi
\end{enumerate}



\throwpage

\section{Quick Roadmap}\label{section:roadmap}

If you want to:

\begin{itemize}
\item Download the OS binaries  (executables and libraries) -- see Section~\ref{section:downloadbinaries}.

\item Download the OS source code -- see Section~\ref{section:downloadsource}.

\item Download just the OS API, readers and writers -- see Section~\ref{section:oslite}.

\item Use the OSSolverService to read files in nl\index{AMPL nl format}, OSiL\index{OSiL}, 
or MPS format\index{MPS format} and call a solver locally or remotely -- see Section~\ref{section:ossolverservice}.

\item Use modeling languages to generate model instances in OSiL format -- see Section \ref{section:modellang}.

\item Use AMPL\index{AMPL} to solve problems either locally or remotely
with a COIN-OR solver, Cplex\index{cplex@{\tt cplex}},
GLPK\index{Third-party software, {\tt GLPK}}, \ifknitro Knitro\index{knitro}, \fi
or LINDO\index{LINDO} -- see Section~\ref{section:amplclient}.

\item Use GAMS\index{GAMS} to solve problems either locally or remotely -- see Section~\ref{section:gamslinks}.

\item Use MATLAB\index{MATLAB} to generate problem instances in OSiL format and call a solver either remotely or locally
 -- see Section~\ref{section:usingmatlab}.

\item Create your own applications by linking against the binaries -- see Sections \ref{section:examples} and~\ref{section:OSDip}.

\item Use the OS library to build model instances or use solver APIs -- see Sections \ref{section:osmodelinterfaces},
\ref{section:ossolverinterfaces} and~\ref{section:osinstanceAPI}.

\item Use the OS library for algorithmic differentiation\index{Algorithmic differentiation} (in conjunction with 
COIN-OR CppAD)\index{COIN-OR projects!CppAD@{\tt CppAD}} -- see Section~\ref{section:ad}.

\item Build the OS project from the source code -- see Section~\ref{section:build}.

\item Build a remote solver service using Apache Tomcat\index{Apache Tomcat} -- see Section~\ref{section:tomcat}.
\end{itemize}





% Part 1 for folks who just want to run the executables

\throwpage

\pagenumbering{gobble}

\part{Running the OS executables}

\pagenumbering{bychapter}

\division{Downloading the \ifdevelop OS\else CoinAll\fi  Binaries}\label{section:obtainingbinaries}

\ifdevelop
The OS project is an open-source project  with source code under the Eclipse Public License~(EPL)%
\index{Eclipse Public License (EPL)}.
See~{\tt\UrlEpl}.  This project was initially created by Robert Fourer, Jun Ma, and Kipp Martin.
The code has been written primarily by  Horand Gassmann,   Jun Ma,  and Kipp Martin.    
Horand Gassmann,  Jun Ma,  and Kipp Martin are the COIN-OR project leaders and active developers for the OS project.
\else
The CoinAll project is actually a meta-project consisting of most of the COIN-OR solvers and supporting utility projects.  We describe how to download this project. 
\fi

%Below we describe different methods for obtaining the binaries and C++ source code.
Most users will only be interested in obtaining the binaries, which we describe  next.
%in Section~\ref{section:obtainingbinaries}. The remaining sections of this chapter deal with obtaining %the source code for the project, which will be of interest mostly to developers.
It is also possible to obtain the source code for the project, which will be of interest mostly to developers. 
\ifdevelop
Details can be found in  Section~\ref{section:downloadsource}.
\else
If binaries are not provided for a particular operating system, it may be possible to build them from the source.
For details it is best to start reading the OS web page at~{\tt\UrlOsWiki}.
\fi



%If the user does not wish to compile source code, the OS library, OSSolverService executable
%and Tomcat server software configuration are available in binary format for some operating systems.     
The repository of the binaries is at {\tt\UrlOsBinaries}\index{Downloading!binaries}.
%
\ifdevelop
 Unlike the source code described in Section~\ref{section:downloadwithsvn}, the binary files 
are not subject to version control and can be downloaded using an ordinary browser. 
%If binaries are not provided for a particular operating system,
%it may be possible to build them from the source code. Since the source is under version control, 
%this requires svn. (See Sections \ref{section:svn}, \ref{section:downloadwithsvn} and~\ref{section:build}.)
\fi

The binary distribution for the OS library and executables follows the following naming convention:


\begin{verbatim}
OS-version_number-platform-compiler-build_options.tgz (zip)
\end{verbatim}
For example, OS  Release 2.1.0 compiled with the Intel 9.1 compiler on an Intel 32-bit Linux system is:
\begin{verbatim}
OS-2.1.0-linux-x86-icc9.1.tgz
\end{verbatim}

For more detail on the naming convention and examples see:

\medskip
\noindent{\tt\UrlCoinNames}
\medskip

After unpacking the {\tt tgz} or {\tt zip} archives, the following folders are available.
\begin{itemize}

\item[] {\bf bin --} this directory has the executables {\tt OSSolverService}\index{OSSolverService@{\tt OSSolverService}} 
and {\tt OSAmplClient}\index{OSAmplClient@{\tt OSAmplClient}}.

\item[]  {\bf include --} the header files that are necessary in order to link against the OS library.

\item[] {\bf lib --} the libraries that are necessary for creating applications that use the OS library.

\item[] {\bf  share --} license and author information for all the projects used by the OS project.
\end{itemize}



Files are also provided for an Apache Tomcat\index{Apache Tomcat} Web server along with the associated Web service
that can
read SOAP \index{SOAP protocol} envelopes with model instances in OSiL\index{OSiL} format and/or options in 
OSoL\index{OSoL} format, call the {\tt OSSolverService}\index{OSSolverService@{\tt OSSolverService}},
and return the optimization result in OSrL\index{OSrL} format.
The naming convention\index{file naming conventions} for the server binary is
\begin{verbatim}
OS-server-version_number.tgz (.zip)
\end{verbatim}
For example, the files associated with  OS server release 2.0.0 are in the binary distribution
\begin{verbatim}
OS-server-2.0.0.tgz
\end{verbatim}
There is no platform information given since the server and related binaries were written in Java\index{Java}.
\ifdevelop
The details and use of this distribution are described in Section~\ref{section:tomcat}.
\fi


Finally for Windows users we provide Visual Studio \index{Microsoft Visual Studio} project files 
(and supporting libraries and header files) for building projects based on the OS library and libraries 
used by the OS project. The binary for this is named
\begin{verbatim}
OS-version_number-VisualStudio.zip
\end{verbatim}
For example, the necessary files associated with  OS  stable\index{OS project!stable release} 2.4 
are in the binary distribution
\begin{verbatim}
OS-2.4-VisualStudio.zip
\end{verbatim}
The binaries provided are based on Visual Studio Express 2008.  
\ifdevelop See Section \ref{section:vsexamples} for more detail.\fi


\throwpage

\input{chapters/ossolverservice.tex}

\throwpage

\input{chapters/OSSolverServiceRules.tex}

\throwpage




\section{OS Support for Modeling Languages, Spreadsheets and Numerical Computing Software}\label{section:modellang}

Algebraic modeling languages can be used to generate model instances as input to an OS compliant solver.
We describe two such hook-ups, {\tt OSAmplClient} for AMPL\index{AMPL}, and {\tt CoinOS} for GAMS\index{GAMS} (version 23.8 and above).


\subsection{AMPL Client:  Hooking AMPL to Solvers}\label{section:amplclient}

\index{OSAmplClient@{\tt OSAmplClient}|(}
\index{AMPL|(}




%This section is based on the assumption that the user has installed  AMPL  on his or her machine.   
It is possible to call all of the COIN-OR solvers listed in %Section~\ref{section:overview} 
Table~\ref{table:configurations}~(p.\pageref{table:configurations})
directly from the  AMPL (see {\tt http://www.ampl.com}) modeling language.  In this discussion we assume 
the user has already obtained and installed AMPL.
\ifdevelop  
Both the binary download described in Section~\ref{section:obtainingbinaries}
and the unix and Windows builds (Section \ref{section:unixbuilds}
and~\ref{section:windowsinstall}, respectively) contain
\else
The binary download described in Section~\ref{section:obtainingbinaries}
contains
\fi
%In  the download described in Section~\ref{section:binary} there is 
an executable, {\tt OSAmplClient.exe},
that is linked to all of the COIN-OR solvers  listed in Table~\ref{table:configurations}. %Section~\ref{section:overview}.   
From the  perspective of AMPL, the   {\tt OSAmplClient} acts like an AMPL ``solver''.    
The {\tt OSAmplClient.exe}   can be used to solve problems either locally or remotely.   


\subsubsection{Using OSAmplClient for a Local Solver}\label{section:localampl}

In the following discussion we assume that the AMPL executable {\tt ampl.exe}, the {\tt OSAmplClient},  
and the test problem {\tt  eastborne.mod}\index{eastborne.mod@{\tt eastborne.mod}|(}
 are all in the same directory.  

The  problem instance {\tt eastborne.mod} is an AMPL model file included in the OS distribution 
in the {\tt amplFiles}\index{amplFiles@{\tt amplFiles}} directory.  To solve this problem locally 
by calling {\tt OSAmplClient.exe} from AMPL, first start AMPL and then open the {\tt eastborne.mod} file 
inside AMPL.  The test model {\tt eastborne.mod} is a linear integer program. 


%\begin{verbatim}
%# take in sample integer linear problem
%# assume the problem is in the AMPL directory
\begin{verbatim}
model eastborne.mod;
\end{verbatim}

The next step is to tell AMPL that the solver it is going to use is {\tt OSAmplClient.exe}. 
Do this by issuing the following command inside AMPL.

%\begin{verbatim}
%# tell AMPL that the solver is OSAmplClient
\begin{verbatim}
option solver OSAmplClient;
\end{verbatim}

It is not necessary to provide the  {\tt OSAmplclient.exe} solver with any options. 
You can just issue the {\tt solve} command in AMPL as illustrated below.  

%\begin{verbatim}
%# solve the problem
\begin{verbatim}
solve;
\end{verbatim}

Of the six methods described in Section~\ref{section:ossolverservice} only the {\tt solve} method 
has been implemented to date.

If no options are specified, the default solver is used, depending on the problem characteristics 
(see Table~\ref{table:defaultsolvers} on p.\pageref{table:defaultsolvers}).\index{default solver}
%is to use {\tt Clp}\index{Clp@{\tt Clp}} for linear programs. 
%For continuous nonlinear models {\tt Ipopt}\index{Ipopt@{\tt Ipopt}} is used. 
%For mixed-integer linear models, {\tt Cbc}\index{Cbc@{\tt Cbc}} is used. 
%For mixed-integer nonlinear models  {\tt Bonmin}\index{Bonmin@{\tt Bonmin}} is used.  
If you wish to specify a specific solver, use the {\tt solver} option.   For example,  
since the test problem {\tt eastborne.mod} is a linear integer program, {\tt Cbc} is used by default. 
If instead you want to  use {\tt SYMPHONY}\index{COIN-OR projects!SYMPHONY@{\tt SYMPHONY}|(},
then you would pass a {\tt solver} option to the {\tt OSAmplclient.exe} solver as follows.%
\index{eastborne.mod@{\tt eastborne.mod}|)}

%\begin{verbatim}
%# tell OSAmplClient to use SYMPHONY instead of Cbc
\begin{verbatim}
option OSAmplClient_options "solver symphony";
\end{verbatim}
\index{COIN-OR projects!SYMPHONY@{\tt SYMPHONY}|)}

Valid values for the {\tt solver} option are installation-dependent.
%{\tt bonmin}, {\tt cbc}, {\tt clp}, {\tt couenne}, {\tt dylp}, {\tt symphony}, and {\tt vol}.   
The solver name in the {\tt solver} option is case insensitive.  


\subsubsection{Using OSAmplClient to Invoke an OS Solver Server}\label{section:remoteampl}

Next, assume that you have a large problem you want to solve on a remote solver. It is necessary 
to specify the location of the server solver as an option to OSAmplClient. 
The {\tt serviceLocation} option is used to specify the location of a solver server. 
In this case, the string of options for {\tt OSAmplClient\_options} is:

\begin{verbatim}
serviceLocation  http://xxx/OSServer/services/OSSolverService
\end{verbatim}
where {\tt xxx} is the URL for the server.  This string is used to replace the string `{\tt solver symphony}' in the previous example. 
The {\tt serviceLocation} option will send the problem to the %solver server at 
location {\tt http://xxx} and, assuming the remote executable is indeed found 
in the indicated folder, will start the executable.  


\medskip


However, each call 
\begin{verbatim}
option OSAmplClient_options
\end{verbatim}
is memoryless. That is, the options set in the last call will overwrite any options set in previous calls
and cause them to be discarded.  For instance, the sequence of option calls
\begin{verbatim}
option OSAmplClient_options "solver symphony";
option OSAmplClient_options "serviceLocation  
    http://xxx/OSServer/services/OSSolverService";
solve;
\end{verbatim}
will result in the default solver being called. 

If the intent is to use the SYMPHONY solver at the remote location, the option must be declared
as follows:

\begin{verbatim}
option OSAmplClient_options "solver symphony 
    serviceLocation http://xxx/OSServer/services/OSSolverService";
solve;
\end{verbatim}


For brevity we will omit the AMPL instruction
\begin{verbatim}
option OSAmplClient_options
\end{verbatim}
the double quotes and the trailing semicolon in the remaining examples.  

\medskip

Finally, the user may wish to pass options to the individual solver. This is done by specifying an options file.
(A sample options file, {\tt solveroptions.osol}\index{solveroptions.osol@{\tt solveroptions.osol}} is 
provided with this distribution).  The name of the options file is the value of the {\tt osol} option.
The string of options to {\tt OSAmplClient\_options} is now
\begin{verbatim}
serviceLocation http://xxx/OSServer/services/OSSolverService
osol solveroptions.osol
\end{verbatim}
This   {\tt solveroptions.osol}  file contains four solver options; two for {\tt Cbc}, one for {\tt Ipopt}, 
and one for {\tt SYMPHONY}\index{COIN-OR projects!SYMPHONY@{\tt SYMPHONY}}.
You can have any number of options. Note the format for specifying an option:
\begin{verbatim}
    <solverOption name="maxN" solver="cbc" value="5" />
\end{verbatim}
The attribute {\tt name} specifies that the option name is {\tt maxN} which is the maximum number of nodes 
allowed in the branch-and-bound tree, the {\tt solver} attribute specifies the name of the solver that the 
option should be applied to, and the {\tt value} attribute specifies the value of the option. 
As a second example, consider the specification
\begin{verbatim}
    <solverOption name="max_iter" solver="ipopt" type="integer" value="2000"/> 
\end{verbatim}
In this example we are specifying an iteration limit for {\tt Ipopt}.  Note the additional attribute 
{\tt type} that has value  {\tt integer}. The Ipopt solver requires specifying the data type 
(string, integer, or numeric) for its options.   Different solvers have different options, 
and we recommend that the user look at the documentation for the solver of interest in order to see 
which options are available.  
A good summary of options for COIN-OR solvers is 
%\url{http://www.coin-or.org/GAMSlinks/gamscoin.pdf}.
\url{http://www.gams.com/dd/docs/solvers/coin.pdf}.

If you examine the file {\tt solveroptions.osol} you will see that there is an XML tag  with the name
{\tt <solverToInvoke>} and that the solver given is {\tt symphony}.   
{\bf This has no effect on a local solve!} However, if this option file is paired with 

\begin{verbatim}
serviceLocation http://xxx/OSServer/services/OSSolverService
osol solveroptions.osol
\end{verbatim}
then in our reference implementation the remote solver service will parse the file {\tt solveroptions.osol}, find the {\tt <solverToInvoke>} tag and then pass the {\tt symphony} solver option to the {\tt OSSolverService} on the remote server.

\subsubsection{AMPL Summary}

\begin{enumerate}
\item Tell  AMPL to use the OSAmplClient as the solver:

\begin{verbatim}
option solver OSAmplClient;
\end{verbatim}

\item Specify options to the OSAmplClient solver by using the AMPL command 

\begin{verbatim}
option OSAmplClient_options "(option string)";
\end{verbatim}

\item There are three possible options to specify:

\begin{itemize}
\item the location of the options file using  the {\tt osol} option;

\item the location of the remote server using   the {\tt serviceLocation} option;

\item the name of the solver using the  {\tt solver} option; valid values for this option  are 
%{\tt clp}, {\tt cbc},  {\tt dylp},  {\tt ipopt}, {\tt bonmin},   {\tt couenne},  and  {\tt symphony}
installation-dependent. 
For details, see Table~\ref{table:configurations} on page~\pageref{table:configurations} 
and the discussion in Section~\ref{section:OSSolverServiceInputParameters}. 

\end{itemize}

These three options behave {\it exactly like} the {\tt solver}, {\tt serviceLocation}, and {\tt osol} options used by the {\tt OSSolverService} described in  Section \ref{section:commandlineparser}.
Note that the {\tt solver} option only has an effect with a local solve; 
if the user wants to invoke a specific solver with a remote solve, then this must be done in the OSoL file using the {\tt <solverToInvoke>} element.

\item  The options given to {\tt OSAmplClient\_options}  can be given in any order.

\item If no solver is specified using {\tt OSAmplClient\_options},  the default solver is used.
(For details see Table~\ref{table:defaultsolvers}).\index{default solver}

\item A remote solver is called if and only if the {\tt serviceLocation} option is specified.

\end{enumerate}

\index{OSAmplClient@{\tt OSAmplClient}|)}
\index{AMPL|)}



\subsection{GAMS and Optimization Services}\label{section:gamslinks}

\index{GAMS|(}

This section pertains to GAMS version 23.8 (and above) that now includes support for OS.  
Here we describe the GAMS  implementation of Optimization Services.  We assume that the user has installed GAMS.

In GAMS, OS is implemented through the {\tt CoinOS} solver that is packaged with GAMS.      
The GAMS {\tt CoinOS} solver is really a {\it solver interface} that links to the OS library.
At present the GAMS  {\tt CoinOS} solver does not support local calls, but it can be used to make
remote calls to an {\tt OSSolverService} executable on a remote server. How this is done is the topic of the next section.



\subsubsection{Using GAMS  to Invoke a Remote OS Solver Service}\label{section:gamsremote}

We now describe how to call  a remote OS   solver service using the GAMS {\tt CoinOS}.  Before proceeding, 
it is important to emphasize that when calling a remote OS solver service, different sets of solvers may be supported, even for the same version of the OS solver service. 
For example, the remote 
implementation may provide access to solvers such as {\tt SYMPHONY}, {\tt Couenne}, {\tt Glpk} and {\tt DyLP}.  
There are several reason why you might wish to use a remote OS solver service. 

\begin{itemize}
\item Have access to a faster machine.

\item  Be able to  submit jobs to run in asynchronous mode -- submit your job,  turn off your laptop,  
and check later to see if the job ran.

\item Call several additional solvers (e.g., {\tt SYMPHONY}, {\tt Couenne}, {\tt Glpk} and {\tt DyLP}).
Note, however, that not all solvers may be available available locally (especially Glpk) may not be available for a remote call.

\end{itemize}

We will illustrate several possible calls with the sample GAMS file {\tt eastborne.gms} which found in the
{\tt  data/gamsFiles} directory. We assume that this file exists in the current directory and that the GAMS executable is found in the search path. The command to execute at the command line would then be

\begin{verbatim} 
gams eastborne.gms MIP=CoinOS optfile=1
\end{verbatim}

The server name ({\tt CoinOS}) is case-insensitive and could equally well have been written as 
``{\tt MIP=coinos}'' or ``{\tt MIP=COINOS}''. Moreover, the file {\tt eastborne.gms} contains the directive

\begin{verbatim}
Option MIP = CoinOS;
\end{verbatim}

\noindent and hence the option {\tt MIP=CoinOS} could have been omitted from the command line.

Since the solver is named {\tt CoinOS}, the options file pointed to by the last part of the command
({\tt optfile=1}) should be named {\tt CoinOS.opt}. In general multiple option files are possible, and the GAMS convention is as follows:

{\tt optfile=1} corresponds to {\tt CoinOS.opt}

{\tt optfile=2} corresponds to {\tt CoinOS.op2}

{\ldots}

{\tt optfile=99} corresponds to {\tt CoinOS.o99}
\medskip
It is important to distinguish between the option files for GAMS just mentioned and the  option file (in OSoL format) passed to the OS solver server (see below).
We now explain the valid options that can go into a GAMS option file when using the CoinOS solver. 
The options are

\vskip 8pt
\noindent{\tt service (string)}: Specifies the URL of  the COIN-OR solver service. 
This option is required in order to direct the remote call appropriately.
\vskip 8pt
Use the following value for this option.
\begin{verbatim}
service http://74.94.100.129:8080/OSServer/services/OSSolverService
\end{verbatim}


\iffalse
\vskip 8pt
\noindent {\tt solver  (string)}:   Specifies the solver that is used to solve an instance. 
Valid values are {\tt clp},  {\tt cbc}, {\tt glpk}, {\tt ipopt},  and {\tt bonmin}.  
If a solver name is specified that is not recognized, the default solver for the problem type is used.  
The value for the solver option is case insensitive. 
For example, if the file {\tt CoinOS.opt} contains the two lines
\begin{verbatim}
service http://74.94.100.129:8080/OSServer/services/OSSolverService
solver glpk
\end{verbatim}
then executing
\begin{verbatim}
gams.exe eastborne.gms optfile 1
\end{verbatim}
will result in  using {\tt Glpk}  to solve the problem.   
\fi

\vskip 8pt
\noindent {\tt writeosil  (string)}:  If this option is used, GAMS will write the optimization instance 
to file {\tt (string)} in    OSiL   format.
\vskip 8pt

\vskip 8pt
\noindent {\tt writeosrl  (string)}:  If this option is used, GAMS will write the result of the optimization 
to file {\tt (string)} in OSrL  format.
\vskip 8pt

The options just described are options for the GAMS modeling language.  
It is also possible to pass options directly to the COIN-OR solvers by using the {\tt OS} interface.
This is done by passing the name of an options file that conforms to the  OSoL  standard.  
%See \url{http://projects.coin-or.org/OS}  for information on Optimization Services.  
The option

\vskip 8pt
\noindent {\tt readosol  (string)}  specifies the name of an OS option  file in OSoL format that is 
given to the solver.  {\bf Note well:} The file  {\tt CoinOS.opt} is an option  file for GAMS but the GAMS option 
{\tt readosol} in the GAMS options file  is specifying the name of an OS options file. 
\vskip 8pt
The file {\tt solveroptions.osol} is contained in the OS distribution in the {\tt osolFiles} directory   
in the {\tt data} directory. This file contains four solver options; two for {\tt Cbc}, one for {\tt Ipopt},
and one for {\tt SYMPHONY} (which is available for remote server calls, but not locally).  
You can have any number of options. Note the format for specifying an option:
\begin{verbatim}
    <solverOption name="maxN" solver="cbc" value="5" />
\end{verbatim}
The attribute {\tt name} specifies that the option name is {\tt maxN} which is the maximum number of nodes 
allowed in the branch-and-bound tree, the {\tt solver} attribute specifies the name of the solver to which
the option should be applied, and the {\tt value} attribute specifies the value of the option. 

Default solver values are present, depending on the problem characteristics. For more details, consult 
Table~\ref{table:defaultsolvers} (p.\pageref{table:defaultsolvers}).
In order to control the solver used, it is necessary to specify the name of the solver
inside the XML tag {\tt <solverToInvoke>}. The example  {\tt solveroptions.osol} file contains the XML tag
\begin{verbatim}
    <solverToInvoke>symphony</solverToInvoke>
\end{verbatim}

\iffalse
If, for example,  the {\tt CoinOS.opt} file is
\begin{verbatim}
solver ipopt
service http://74.94.100.129:8080/OSServer/services/OSSolverService
readosol  solveroptions.osol
writeosrl temp.osrl
\end{verbatim}
then {\tt Ipopt} is ignored as a solver option and the remote server uses the {\tt  SYMPHONY} solver.
\fi  
Valid values for the remote solver service specified in the {\tt <solverToInvoke>} tag are 
installation dependent; the solver service at 
{\tt http://74.94.100.129:8080/OSServer/services/OSSolverService} accepts
{\tt clp},  
{\tt cbc},  {\tt dylp}, {\tt glpk}, {\tt ipopt}, {\tt bonmin},   {\tt couenne},  {\tt symphony}, and 
{\tt vol}.  

%If the problem is solved using a remote solver service the value specified by the 
%GAMS {\tt solver} option is irrelevant and ignored. 

\medskip



By default, the call to the server is a {\it synchronous} call. The GAMS process will wait for the result 
and then display the result. This may not be desirable when solving large optimization models.  
The user may wish to submit a job, turn off his or her computer,  and then check at a later date to see 
if the job is finished.  In order to use the remote solver service in this fashion, i.e., 
{\it asynchronously}, it  is necessary to use the  {\tt service\_method} option.

\vskip 8pt
\noindent {\tt service\_method (string)} specifies the method to execute on a server.  
Valid values for this option are {\tt solve}, {\tt getJobID}, {\tt send}, {\tt knock}, {\tt retrieve}, 
and {\tt kill}. We explain how to use each of these.
\vskip 8pt
The default value of {\tt service\_method} is {\tt solve.} A {\tt solve} invokes the remote service 
in synchronous mode. When using the {\tt solve} method you can optionally specify a set of solver options 
in an OSoL file  by using the {\tt readosol} option. The  remaining values for the {\tt service\_method} 
option are used for an asynchronous call.  We illustrate them in the order in which they would most 
logically be executed. 

\vskip 8pt
\noindent {\tt service\_method getJobID}: When working in asynchronous mode, the server needs to 
uniquely identify each job. The {\tt getJobID} service method will result in the server returning 
a unique job ID. For example if the following {\tt CoinOS.opt} file is used
\vskip 8pt
\begin{verbatim}
service http://74.94.100.129:8080/OSServer/services/OSSolverService
service_method getJobID
\end{verbatim}
with the command
\begin{verbatim}
gams.exe eastborne.gms optfile=1
\end{verbatim}
the user will see a rather long job ID returned to the screen as output. Assume that the job id returned 
is {\tt coinor12345xyz}. This job ID is used to submit a job to the server with the {\tt send} method.
Any job ID can be sent to the server as long as it has not been used before.  

\vskip 8pt
\noindent {\tt service\_method send}: When working in asynchronous mode, use the {\tt send} service method 
to submit a job. When using  the {\tt send} service method a job ID is required. An options file
must be present and must specify a  job ID that has not been used before.  Assume that in the file {\tt CoinOS.opt}  we specify 
the options:
\vskip 8pt
\begin{verbatim}
service http://74.94.100.129:8080/OSServer/services/OSSolverService
service_method send
readosol sendWithJobID.osol
\end{verbatim}
The {\tt sendWithJobID.osol} options file is identical to the {\tt solveroptions.osol} options file except 
that it has an additional XML tag:
\begin{verbatim}
    <jobID>coinor12345xyz</jobID> 
\end{verbatim}
We then execute
\vskip 8pt
\begin{verbatim}
gams.exe eastborne.gms optfile=1
\end{verbatim}
If all goes well, the response to the above command should  be: 
``Problem instance successfully sent to OS service''. 
At this point the server will schedule the job and work on it. It is possible to turn off 
the user computer at this point. At some point the user will want to know if the job is finished. 
This is accomplished using the {\tt knock} service method.
\vskip 8pt
\noindent {\tt service\_method knock}: When working in asynchronous mode, this is used to check the status 
of a job.  Consider the following {\tt CoinOS.opt} file:
\vskip 8pt
\begin{verbatim}
service http://74.94.100.129:8080/OSServer/services/OSSolverService
service_method knock
readosol sendWithJobID.osol 
readospl knock.ospl
writeospl knockResult.ospl
\end{verbatim}
The {\tt knock} service method requires two  inputs. The first input is the name of an options file, 
in this case {\tt sendWithJobID.osol}, specified through the {\tt readosol} option. In addition, a file 
in OSpL format is required. You can use the {\tt knock.opsl} file provided in the binary distribution. 
This file name is specified using the {\tt readospl} option. If no job ID is specified in the OSoL file 
then the status of all jobs on the server will be returned in the file specified by the {\tt writeospl} 
option. If a job ID is specified in the OSoL file, then only information on the specified job ID is 
returned in the file specified by the {\tt writeospl} option.  In this case the file name is 
{\tt knockResult.ospl}. We then execute
\vskip 8pt
\begin{verbatim}
gams.exe eastborne.gms optfile=1
\end{verbatim}
The file {\tt knockResult.ospl} will contain information similar to the following:
\begin{verbatim}
    <job jobID="coinor12345xyz">
        <state>finished</state>
        <serviceURI>http://192.168.0.219:8443/os/OSSolverService.jws</serviceURI>
        <submitTime>2009-11-10T02:13:11.245-06:00</submitTime>
        <startTime>2009-11-10T02:13:11.245-06:00</startTime>
        <endTime>2009-11-10T02:13:12.605-06:00</endTime>
        <duration>1.36</duration>
    </job>
\end{verbatim}
Note that the job is complete as indicated in the {\tt <state>} tag. It is now time to actually retrieve 
the job solution.  This is done with the {\tt retrieve} method.
\vskip 8pt
\noindent {\tt service\_method retrieve}: When working in asynchronous mode, this method is used 
to retrieve the job solution. It is necessary when using {\tt retrieve} %{\tt knock} ???
to specify an options file and in that options file specify a job ID.   
Consider the following {\tt CoinOS.opt} file:
\vskip 8pt
\begin{verbatim}
service http://74.94.100.129:8080/OSServer/services/OSSolverService
service_method retrieve
readosol sendWithJobID.osol
writeosrl answer.osrl
\end{verbatim}
When we then execute
\vskip 8pt
\begin{verbatim}
gams.exe eastborne.gms optfile=1
\end{verbatim}
the result is written to the file {\tt answer.osrl}. 

Finally there is a {\tt kill} service method which is used to kill a job that was submitted by mistake 
or is running too long on the server. 
\vskip 8pt
\noindent {\tt service\_method kill:} When working in asynchronous mode, this method is used to terminate 
a job. You should specify an OSoL  file containing the job ID by using the {\tt readosol} option.
\vskip 8pt

\iffalse
\subsubsection{Using GAMS to Invoke the Local OS Solver Service \tt CoinOS}\label{section:gamslocal}

   
The GAMS  {\tt CoinOS} solver is really a {\it solver interface} and is linked through the OS library to the 
following COIN-OR solvers: {\tt Bonmin}, {\tt Cbc}, {\tt Clp},  {\tt Glpk}, and {\tt Ipopt}. 
Think of {\tt CoinOS} as a {\it metasolver}.    As an example (we assume a Windows operating system 
and use the .exe extension), consider:

\begin{verbatim}
gams.exe eastborne.gms MIP=CoinOS
\end{verbatim}
The solver name {\tt CoinOS} is not case sensitive and 
\begin{verbatim}
gams.exe eastborne.gms MIP=coinos
\end{verbatim}
will also work.  In addition, if
\begin{verbatim}
Option MIP = CoinOS ;
\end{verbatim}
is present in the GAMS file, then writing {\tt MIP=CoinOS} on the command line is unnecessary.
Since {\tt Option MIP = CoinOS;} is present in the GAMS model file {\tt eastborne.gms}, 
we will not specify it explicitly on the command line in the ensuing discussion. To summarize,
\begin{verbatim}
gams.exe eastborne.gms 
\end{verbatim}
is equivalent to the two versions of the command given previously.  Executing any of the commands will 
result in the model being solved on the local machine using the COIN-OR solver {\tt Cbc}, the default solver 
for 
%continuous linear models (LP and RMIP), {\tt CoinOS} chooses {\tt Clp}. For continuous nonlinear 
%models (NLP, DNLP, RMINLP, QCP, RMIQCP), {\tt Ipopt} is the default solver. For 
mixed-integer linear models (MIP).
%,  {\tt Cbc} is the default solver. For mixed-integer nonlinear models (MIQCP, MINLP), 
%{\tt Bonmin} is the default solver.

It is possible to control which solver is selected by {\tt CoinOS}.    This is done by providing an {\it options file}  to  GAMS.   
Since the solver is named {\tt  CoinOS}, the options file should  be named {\tt CoinOS.opt}  (the file name is not case sensitive)
and the command line call is 
\begin{verbatim}
gams.exe eastborne.gms optfile 1
\end{verbatim}
Calling multiple GAMS options files uses the convention
\begin{verbatim}
optfile=1 corresponds to CoinOS.opt
optfile=2 corresponds to CoinOS.op2
...
optfile=99 corresponds to CoinOS.o99
\end{verbatim}

We now explain the valid options that can go into a GAMS option file when using the {\tt CoinOS} solver.  They are:

\vskip 8pt
\noindent {\tt solver  (string)}:   Specifies the solver that is used to solve an instance. 
Valid values are {\tt clp},  {\tt cbc}, {\tt glpk}, {\tt ipopt},  and {\tt bonmin}.  
If a solver name is specified that is not recognized, the default solver for the problem type is used.  
The value for the solver option is case insensitive. 
For example, if the file {\tt CoinOS.opt} contains a single line
\begin{verbatim}
solver glpk
\end{verbatim}
then executing
\begin{verbatim}
gams.exe eastborne.gms optfile 1
\end{verbatim}
will result in  using {\tt Glpk}  to solve the problem.   


\vskip 8pt
\noindent {\tt writeosil  (string)}:  If this option is used, GAMS will write the optimization instance 
to file {\tt (string)} in    OSiL   format.
\vskip 8pt

\vskip 8pt
\noindent {\tt writeosrl  (string)}:  If this option is used, GAMS will write the result of the optimization 
to file {\tt (string)} in OSrL  format.
\vskip 8pt

The options just described are options for the GAMS modeling language.  
It is also possible to pass options directly to the COIN-OR solvers by using the {\tt OS} interface.
This is done by passing the name of an options file that conforms to the  OSoL  standard.  
%See \url{http://projects.coin-or.org/OS}  for information on Optimization Services.  
The option

\vskip 8pt
\noindent {\tt readosol  (string)}  specifies the name of an OS option  file in OSoL format that is 
given to the solver.  Note: The file  {\tt CoinOS.opt} is an option  file for GAMS but the GAMS option 
{\tt readosol} in the GAMS options file  is specifying the name of an OS options file. 
\vskip 8pt
The file {\tt solveroptions.osol} is contained in the OS distribution in the {\tt osolFiles} directory   
in the {\tt data} directory. This file contains four solver options; two for {\tt Cbc}, one for {\tt Ipopt},
and one for {\tt SYMPHONY} (which is available for remote server calls, but not locally).  
You can have any number of options. Note the format for specifying an option:
\begin{verbatim}
    <solverOption name="maxN" solver="cbc" value="5" />
\end{verbatim}
The attribute {\tt name} specifies that the option name is {\tt maxN} which is the maximum number of nodes 
allowed in the branch-and-bound tree, the {\tt solver} attribute specifies the name of the solver to which
the option should be applied, and the {\tt value} attribute specifies the value of the option. 

As a second example, consider the specification
\begin{verbatim}
    <solverOption name="max_iter" solver="ipopt" type="integer" value="2000"/> 
\end{verbatim}
In this example we are specifying an iteration limit for {\tt Ipopt}.  Note the additional attribute 
{\tt type} that has value  {\tt integer}. The Ipopt solver requires specifying the data type 
(string, integer, or numeric) for its options.   For a list of options that solvers take, 
see the file
\begin{verbatim}
docs/solvers/coin.pdf
\end{verbatim}
inside the GAMS directory. 
An up-to-date online version of this list is available at \url{http://www.coin-or.org/GAMSlinks/gamscoin.pdf}.

\fi

\subsubsection{GAMS Summary:}\label{section:gamssummary}


\begin{enumerate}

\item[1.]   In order to use OS with GAMS you can either specify {\tt CoinOS} as an option to GAMS 
at the command line,
\begin{verbatim}
gams eastborne.gms MIP=CoinOS
\end{verbatim}
or you can  place the statement {\tt Option ProblemType = CoinOS;} somewhere in the model {\it before} 
the {\tt Solve} statement in the GAMS file.


\item[2.]   If no options are given, then the model will be solved locally using the default solver 
(see Table~\ref{table:defaultsolvers} on p.\pageref{table:defaultsolvers}).
%and {\tt Clp} will be used for 
%linear programs, {\tt Cbc} for integer linear programs, {\tt Ipopt} for continuous nonlinear programs, 
%and {\tt Bonmin} for nonlinear integer programs.

\item[3.] In order to control behavior (for example, whether a local or remote solver is used)  an options
 file,  {\tt CoinOS.opt}, must be used as follows

\begin{verbatim}
gams.exe  eastborne.gms optfile=1
\end{verbatim}

\item[4.]  The  {\tt CoinOS.opt} file is used to specify {\it eight potential options}:


\begin{itemize}
\item {\tt service (string)}: using the COIN-OR solver server; this is done by giving the option

\begin{verbatim}
service  http://74.94.100.129:8080/OSServer/services/OSSolverService
\end{verbatim}


\item  {\tt readosol (string)}: whether or not to send the solver an options file; this is done by 
giving the option
\begin{verbatim}
readosol  solveroptions.osol
\end{verbatim}


\item   {\tt solver (string)}: if a local solve is being done,  a specific solver is specified by 
the option
\begin{verbatim}
solver solver_name
\end{verbatim}

Valid values are {\tt clp},  {\tt cbc}, {\tt glpk}, {\tt ipopt} and {\tt bonmin}. %  and {\tt couenne}.  
When the COIN-OR solver service is being used, the only way to specify the solver to use is through 
the {\tt <solverToInvoke>} tag in an OSoL file. In this case the valid values for the solver are  
{\tt clp}, {\tt cbc}, {\tt dylp}, {\tt glpk}, {\tt ipopt}, {\tt bonmin}, {\tt couenne}, {\tt symphony}
and {\tt vol}.



\item  {\tt writeosrl (string)}:  the solution result can be put into an OSrL file by specifying the option

\begin{verbatim}
writeosrl  osrl_file_name
\end{verbatim}



\item    {\tt writeosil (string)}:   the optimization instance  can be put into an OSiL file by specifying 
the option



\begin{verbatim}
writeosil  osil_file_name
\end{verbatim}


\item {\tt writeospl (string):} Specifies the name of an OSpL  file in which the answer from the 
{\tt knock} or {\tt kill} method is written, e.g.,

\begin{verbatim}
writeospl  write_ospl_file_name
\end{verbatim}


\item {\tt readospl (string):} Specifies the name of an OSpL  file that the {\tt knock} method 
sends to  the server

\begin{verbatim}
readospl  read_ospl_file_name
\end{verbatim}

\item {\tt service\_method (string)}: Specifies the method to execute on a server.  Valid values 
for this option are {\tt solve}, {\tt getJobID}, {\tt send}, {\tt knock}, {\tt retrieve}, and {\tt kill}.

\end{itemize}

\item[5.]  If an OS options file is passed to the GAMS {\tt CoinOS} solver using the GAMS  {\tt CoinOS} option      {\tt readosol}, then GAMS does not interpret  or act on any options in this file. The options in the OS options file are passed directly to either: i) the default local solver, ii) the local solver specified by the  GAMS {\tt CoinOS}  option {\tt solver}, or iii)  to the remote OS solver service if one is specified by the GAMS  {\tt CoinOS} option {\tt service.}

\end{enumerate}

\index{GAMS|)}

\ifruncode\else    % the matlab interface requires the user to compile stuff

\subsection{MATLAB:  Using MATLAB to Build and Run OSiL Model Instances}\label{section:usingmatlab}

\index{MATLAB|(}
MATLAB has powerful matrix generation and manipulation routines. This section is for users who wish to use MATLAB to generate the matrix coefficients for linear or quadratic programs and use the OS library to call a solver and get the result back. Using MATLAB with OS requires the user to compile a file {\tt OSMatlabSolverMex.cpp} into a MATLAB executable file (these files will have a {\tt .mex} extension) after compilation. This executable file is linked to the OS library and works through the MATLAB API to communicate with the OS library. 



The OS MATLAB application differs from the other applications in the {\tt OS/applications} folder in that makefiles are not used.  The file 
\begin{verbatim}
OS/applications/matlab/OSMatlabSolverMex.cpp
\end{verbatim}
must be compiled inside the MATLAB command window.  Building the OS MATLAB application requires the following steps. 


\begin{enumerate}[{\bf Step 1:}]



\item{}   The MATLAB installation contains a file {\tt mexopts.sh} (UNIX) or {\tt mexopts.bat}  (Windows) that must be edited.   This file typically resides  in the {\tt bin} directory of the MATLAB application.    This file  contains compile and link options that must be properly set.   Appropriate paths to header files and libraries must be set.  This discussion is based on the assumption that the user has either done a  {\tt make install} for the OS project or has downloaded a binary archive of the OS project. In either case there will be an {\tt include} directory with the necessary header files and a {\tt lib} directory with the necessary libraries for linking. 

First edit   the {\tt CXXFLAGS} option  to point to  the header files in the {\tt cppad} directory and the {\tt include} directory in the project root. For example, it  should look like:
\begin{verbatim}
CXXFLAGS='-fno-common -no-cpp-precomp -fexceptions
    -I/Users/kmartin/Documents/files/code/cpp/OScpp/COIN-OS/
    -I/Users/kmartin/Documents/files/code/cpp/OScpp/COIN-OS/include'
\end{verbatim}

Next edit the {\tt CXXLIBS} flag so that the OS and supporting libraries are included. For example, it should look like the following\footnote{The libraries to include in CXXLIBS depends upon which projects were compiled with OS.} on a MacIntosh:

\begin{verbatim}
CXXLIBS="$MLIBS -lstdc++ -L/Users/kmartin/coin/os-trunk/vpath/lib 
-lOS -lbonmin -lIpopt -lOsiCbc -lOsiClp -lOsiSym -lOsiVol
-lOsiDylp -lCbc -lCgl -lOsi -lClp  -lSym -lVol -lDylp 
-lCoinUtils -lCbcSolver  -lcoinmumps -ldl -lpthread 
/usr/local/lib/libgfortran.dylib -lgcc_s.10.5 -lgcc_ext.10.5 -lSystem -lm 
\end{verbatim}

{\bf Important:} It has been the authors' experience that setting the necessary MATLAB compiler and linker options to build the {\tt mex} can be tricky.  We include in
\begin{verbatim}
OS/applications/matlab/macOSXscript.txt
\end{verbatim}
the exact options that work on a 64 bit Mac with MATLAB release R2009b.

\item{}  Build the MATLAB executable file. Start MATLAB and in the MATLAB command window connect to the directory {\tt OS/examples/matlab} which  contains the file 

\begin{verbatim}
OSMatlabSolverMex.cpp
\end{verbatim}

\item{} Execute the command:

\begin{verbatim}
mex -v OSMatlabSolverMex.cpp
\end{verbatim}

On a 64 bit machine the command should be

\begin{verbatim}
mex -v -largeArrayDims OSMatlabSolverMex.cpp
\end{verbatim}

The name of the resulting executable is system dependent. 
On an Intel MAC OS X 64 bit chip the name will be  {\tt OSMatlabSolver.mexmaci64}, 
on a Windows system it is {\tt OSMatlabSolver.mexw32}.  



\item{}  Set the MATLAB path to include the directory {\tt  OS/applications/matlab}  (or more generally, the directory with the {\tt mex} executable).


\item{}   In the MATLAB command window, connect to the directory {\tt OS/data/matlabFiles}. Run either of the MATLAB
files {\tt markowitz.m} or {\tt parincLinear.m}.  The result should be displayed in the MATLAB browser window.

\end{enumerate}


To use the {\tt OSMatlabSolver} it is necessary to put the coefficients  from a linear, integer, or quadratic problem into MATLAB arrays.   We illustrate for the linear program:

\begin{alignat}{2}
& \mbox{Minimize} & \quad
10 x_{1} + 9 x_{2}\label{eq:parinobj}\\
& \mbox{Subject to} & \quad .7x_{1} + x_{2}  &\le 630  \label{eq:parinccon1}\\
& & .5x_{1} + (5/6) x_{2} &\le 600 \label{eq:parinccon2}\\
& &  x_{1} + (2/3) x_{2} &\le 708 \label{eq:parinccon3}\\
& & .1x_{1} + .25 x_{2} &\le 135 \label{eq:parinccon4}\\
& & x_{1}, x_{2} &\ge 0 \label{eq:parincnonneg}
\end{alignat}

The MATLAB representation of this problem in MATLAB arrays is
\begin{verbatim}
% the number of constraints
numCon = 4;
% the number of variables
numVar = 2;
% variable types
VarType='CC';
% constraint types
A = [.7  1; .5  5/6; 1   2/3  ; .1   .25];
BU = [630 600  708  135];
BL = [];
OBJ = [10  9];
VL = [-inf -inf];
VU = [];
ObjType = 1;
% leave Q empty if there are no quadratic terms
Q = [];
prob_name = 'ParInc Example'
password = '';
%
%
%the solver
solverName = 'ipopt';
%the remote service address
%if left empty we solve locally -- must solve locally for now
serviceLocation='';
% now solve
callMatlabSolver( numVar, numCon, A, BL, BU, OBJ, VL, VU, ObjType, ...
    VarType, Q, prob_name, password, solverName, serviceLocation)
\end{verbatim}
This example m-file is in the {\tt data} directory and is file {\tt parincLinear.m}. Note that in addition to the problem formulation
we can specify which solver to use through the {\tt solverName} variable.  If solution with a remote solver is desired
this can be specified with the {\tt serviceLocation} variable.  If the {\tt serviceLocation} is left empty, i.e.,
\begin{verbatim}
serviceLocation='';
\end{verbatim}
then a local solver is used. In this case  it is crucial that the appropriate solver is linked in with the {\tt matlabSolver}
executable using the {\tt CXXLIBS} option.


The data directory  also contains the m-file  {\tt template.m} which contains extensive comments about how to formulate
the problems in MATLAB.   The user can edit {\tt template.m} as necessary and create a new instance.




 A second example which is a quadratic problem is given in Section~\ref{section:usingmatlab}.
The appropriate MATLAB m-file is {\tt markowitz.m} in the {data/matlabFiles} directory.
The problem consists in investing  in a number of stocks. The expected returns and risks
(covariances) of the stocks are known. Assume that the decision variables $x_i$
represent the fraction of wealth invested in stock~$i$ and that no stock can have
more than 75\% of the total wealth. The problem then is to minimize the total risk
subject to a budget constraint and a lower bound on the expected portfolio return.

Assume that there are three stocks (variables) and two constraints (not counting the upper limit  %investment
of .75 on the investment variables).


\begin{verbatim}
% the number of constraints
numCon = 2;
% the number of variables
numVar = 3;
\end{verbatim}



All the variables are continuous:


\begin{verbatim}
VarType='CCC';
\end{verbatim}


Next define the constraint upper and lower bounds. There are two constraints, an equality  constraint (an $=$) and a lower bound on portfolio return of .15 (a $\ge$). These two constraints are expressed as



\begin{verbatim}
BL = [1   .15];
BU = [1  inf];
\end{verbatim}



The variables are nonnegative and have upper limits of .75 (no stock can comprise more than 75\% of the portfolio).  This is written as




\begin{verbatim}
VL = [];
VU = [.75 .75 .75];
\end{verbatim}



There are no nonzero linear coefficients in the objective function, but the objective function vector must always be defined and the number of components of this vector is the number of variables.



\begin{verbatim}
OBJ = [0 0 0 ]
\end{verbatim}


 Now the linear constraints.   In the model the two linear constraints are
 \begin{eqnarray*}
 x_{1} + x_{2} + x_{3} &=& 1 \\
 0.3221 x_{1} +   0.0963x_{2} +    0.1187x_{3}  &\ge& .15
 \end{eqnarray*}



 These are expressed as



 \begin{verbatim}
 A = [ 1 1 1  ;
  0.3221   0.0963   0.1187 ];
 \end{verbatim}


Now for the quadratic terms. The only quadratic terms are in the objective function. The objective function is


\begin{eqnarray*}
\min  0.425349694 x_{1}^{2} +  0.445784443 x_{2}^{2} + 0.231430983 x_{3}^{2} + 2 \times 0.185218694 x_{1} x_{2} \\
+ 2 \times 0.139312545 x_{1} x_{3} + 2 \times 0.13881692 x_{2} x_{3}
\end{eqnarray*}


To represent quadratic terms MATLAB uses an array, here denoted $Q$, which has four rows, and a column for each quadratic term. 
In this example there are six quadratic terms. The first row of $Q$ is the row index where the terms appear. By convention, 
the objective function has index -1, and constraints are counted starting at 0.  The first row of $Q$ is


 \begin{verbatim}
 -1 -1 -1 -1 -1 -1
 \end{verbatim}

The second row of $Q$ is the index of the first variable in the quadratic term. We use zero based counting.  
Variable $x_{1}$ has index 0, variable  $x_{2}$ has index 1, and variable $x_{3}$ has index 2.  
Therefore, the second row of $Q$ is



\begin{verbatim}
0 1 2 0 0 1
\end{verbatim}



The third row of $Q$ is the index of the second variable in the quadratic term.   Therefore, the third row of $Q$ is



\begin{verbatim}
0 1 2 1 2 2
\end{verbatim}

Note that terms such as $x_1^2$ are treated as $x_1*x_1$ and that mixed terms such as $x_2x_3$ could be given in either order.

The last (fourth) row is the coefficient. Therefore, the fourth row reads





\begin{verbatim}
.425349654  .445784443  .231430983   .370437388  .27862509   .27763384
\end{verbatim}


The full array is



\begin{verbatim}
Q = [ -1 -1 -1 -1 -1 -1;
      0 1 2 0 0 1 ;
      0 1 2 1 2 2;
      .425349654  .445784443  .231430983   .370437388  .27862509   .27763384
    ];
\end{verbatim}


Finally, name the problem, specify the solver (in this case {\tt ipopt}), the service address (and password if required by the service), and call the solver.



\begin{verbatim}
% replace Template with the name of your  problem
prob_name = 'Markowitz Example from Anderson, Sweeney, Williams, and Martin';
password = '';
%
%the solver
solverName = 'ipopt';
%the remote service service address
%if left empty we solve locally -- must solve locally for now
serviceLocation='';
% now solve
OSCallMatlabSolver( numVar, numCon, A, BL, BU, OBJ, VL, VU, ObjType, VarType, ...
     Q, prob_name, password, solverName, serviceLocation)
\end{verbatim}
\index{MATLAB|)}

\fi



\throwpage


\division{OS Protocols}\label{section:schemadescriptions}

The objective of OS is to provide a set of standards for representing optimization instances, results, solver options,
and communication between clients and solvers in a distributed environment using Web Services.  These standards are
specified by W3C XSD schemas. The schemas for the OS project are contained in the {\tt schemas} folder under the
{\tt OS} root. There are numerous schemas in this directory that are part of the OS standard.
For a full description of all the schemas see  Ma \cite{junma2005}.  We briefly discuss the standards most relevant
to the current version of the OS project.


\subdivision{OSiL (Optimization Services instance Language)} \label{section:osilschema}
OSiL\index{OSiL|(} is
an XML-based language for representing instances of large-scale
optimization problems including linear programs, mixed-integer programs,
quadratic programs, and very general nonlinear programs.

OSiL stores optimization problem instances as XML files.  Consider the following problem instance, which is a
modification of an example of Rosenbrock\index{Rosenbrock, H.H.@{\it Rosenbrock, H.H.}}~\cite{rosenbrock1960}:
%
\begin{alignat}{2}
& \mbox{Minimize} & \quad (1 - x_{0})^{2} + 100(x_{1} - x_{0}^{2})^{2} + 9x_{1} \label{eq:roobj}\\
& \mbox{s.t.} & \quad x_{0} + 10.5 x_{0}^{2} + 11.7 x_{1}^{2} + 3x_{0}x_{1}  &\le 25  \label{eq:ro1}\\
& & \ln(x_{0} x_{1}) + 7.5 x_{0} + 5.25 x_{1} &\ge 10 \label{eq:ro2}\\
& & x_{0}, x_{1} &\ge 0 \label{eq:ro3}
\end{alignat}


There are two continuous variables, $x_{0}$ and $x_{1}$, in this instance, each with a lower bound of 0.
Figure~\ref{figure:variableselement} shows how we represent this information in an XML-based OSiL file.
Like all XML files, this is a text file that contains both {\it markup} and {\it data}. In this case there
are two types of markup, {\it elements} (or {\it tags}\/) and {\it attributes} that describe the elements.
Specifically, there are a {\tt <variables>} element and two {\tt <var>} elements. Each {\tt <var>}
element has attributes {\tt lb}, {\tt name}, and {\tt type} that
describe properties of a decision variable: its lower bound, ``name'', and
domain type (continuous, binary, general integer).


\begin{figure}[b]
\centering
   \small {\obeyspaces\let =\
\fbox{\tt\begin{tabular}{@{}l@{}}
<variables numberOfVariables="2">\\[\Sb]
    <var lb="0" name="x0" type="C"/>\\[\Sb]
    <var lb="0" name="x1" type="C"/>\\[\Sb]
</variables>\\[\Sb]
\end{tabular} }} \medskip
\caption{The {\tt <variables>} element for the example (1)--(4).}\label{figure:variableselement}
\end{figure}


     To be useful for communication between solvers and modeling
languages, OSiL instance files must conform to a standard.
An XML-based representation standard is imposed
through the use of a {\em W3C XML Schema.} The W3C, or World Wide
Web Consortium (\url{www.w3.org}), promotes standards for
the evolution of the web and for interoperability between web
products.  XML Schema (\url{www.w3.org/XML/Schema}) is one
such standard.  A schema specifies the elements and attributes that
define a specific XML vocabulary. The W3C XML Schema is thus a schema
for schemas; it specifies the elements and attributes for a schema
that in turn specifies elements and attributes for an XML
vocabulary such as OSiL. An XML file that conforms to a
schema is called {\it valid} for that schema.

     By analogy to object-oriented programming, a schema is akin to a header file in C++ that defines the members and methods in a class.  Just as a class in C++ very explicitly describes member and method names and properties, a
schema explicitly describes element and attribute names and properties.

{\small
\begin{figure}[b]
   \small {\obeyspaces\let =\
\makebox[0in][t]{\fbox{\tt\begin{tabular}{@{}l@{}}
<xs:complexType name="Variables">\\[\Sb]
    <xs:sequence>\\[\Sb]
        <xs:element name="var" type="Variable" maxOccurs="unbounded"/>\\[\Sb]
    </xs:sequence>\\[\Sb]
    <xs:attribute name="numberOfVariables"\\[\Sb]
            type="xs:positiveInteger" use="required"/>\\[\Sb]
</xs:complexType>\\[\Sb]
\end{tabular} }}} \medskip
\caption{The {\tt  Variables} complexType  in the OSiL
schema.}\label{figure:osilvariables}
\end{figure}
}%end small


{\small
\begin{figure}[b]
   \small {\obeyspaces\let =\
\makebox[0in][t]{\fbox{\tt\begin{tabular}{@{}l@{}}
<xs:complexType name="Variable">\\[\Sb]
    <xs:attribute name="name" type="xs:string" use="optional"/>\\[\Sb]
    <xs:attribute name="init" type="xs:string" use="optional"/>\\[\Sb]
    <xs:attribute name="type" use="optional" default="C">\\[\Sb]
        <xs:simpleType>\\[\Sb]
            <xs:restriction base="xs:string">\\[\Sb]
                <xs:enumeration value="C"/>\\[\Sb]
                <xs:enumeration value="B"/>\\[\Sb]
                <xs:enumeration value="I"/>\\[\Sb]
                <xs:enumeration value="S"/>\\[\Sb]
            </xs:restriction>\\[\Sb]
        </xs:simpleType>\\[\Sb]
    </xs:attribute>\\[\Sb]
    <xs:attribute name="lb" type="xs:double" use="optional" default="0"/>\\[\Sb]
    <xs:attribute name="ub" type="xs:double" use="optional" default="INF"/>\\[\Sb]
</xs:complexType>\\[\Sb]
\end{tabular} }}} \medskip
\caption{The {\tt  Variable} complexType in the OSiL
schema.}\label{figure:osilvar}
\end{figure}
} %end small



Figure~\ref{figure:osilvariables} is a piece of our schema for OSiL. In W3C XML Schema jargon, it defines a {\it complexType,}  whose purpose is to specify elements and attributes that are allowed to appear in a valid XML instance file such as the one excerpted in Figure~\ref{figure:variableselement}. In particular, Figure~\ref{figure:osilvariables} defines the complexType named {\tt Variables}, which
comprises an element named {\tt <var>} and an attribute named {\tt
numberOfVariables}. The {\tt numberOfVariables} attribute is of a
standard type {\tt positiveInteger}, whereas the {\tt <var>} element is
a user-defined complexType named {\tt Variable}. Thus the complexType {\tt
Variables} contains a sequence of {\tt <var>} elements that
are of complexType {\tt Variable}. OSiL's schema must also provide a
specification for the {\tt Variable} complexType, which is shown in
Figure~\ref{figure:osilvar}.

In OSiL the linear part of the problem is stored in the  {\tt
<linearConstraintCoefficients>} element, which stores the coefficient
matrix using three arrays as proposed in the earlier LPFML schema
\cite{fourer2005a}.  There is a child element of {\tt <linearConstraintCoefficients>} 
to represent each array: {\tt <value>} for an array of nonzero coefficients, 
{\tt <rowIdx>} or {\tt <colIdx>} for a corresponding array of row indices or column indices, 
and {\tt <start>} for an array that indicates where each row or column begins in the previous two arrays.
This is shown in Figure~\ref{figure:rowlistMatrix}.


\begin{figure}[ht]
\centering
   \small {\obeyspaces\let =\
\fbox{\tt\begin{tabular}{@{}l@{}}
<linearConstraintCoefficients numberOfValues="3">\\[\Sb]
    <start>\\[\Sb]
        <el>0</el><el>2</el><el>3</el>\\[\Sb]
    </start>\\[\Sb]
    <rowIdx>\\[\Sb]
        <el>0</el><el>1</el><el>1</el>\\[\Sb]
    </rowIdx>\\[\Sb]
    <value>\\[\Sb]
        <el>1.</el><el>7.5</el><el>5.25</el>\\[\Sb]
    </value>\\[\Sb]
</linearConstraintCoefficients>\\[\Sb]
\end{tabular} }} \medskip\\[\Sb]
\caption{The {\tt <linearConstraintCoefficients>} element for constraints
(\ref{eq:ro1}) and (\ref{eq:ro2}).}\label{figure:rowlistMatrix}
\end{figure}

The quadratic part of the problem is represented  in Figure~\ref{figure:qterms}.

\begin{figure}[ht]
\centering
   \small {\obeyspaces\let =\
\fbox{\tt\begin{tabular}{@{}l@{}}
<quadraticCoefficients numberOfQuadraticTerms="3">\\[\Sb]
     <qTerm idx="0" idxOne="0" idxTwo="0" coef="10.5"/>\\[\Sb]
     <qTerm idx="0" idxOne="1" idxTwo="1" coef="11.7"/>\\[\Sb]
     <qTerm idx="0" idxOne="0" idxTwo="1" coef="3."/>\\[\Sb]
</quadraticCoefficients>\\[\Sb]
\end{tabular} }} \medskip
\caption{The {\tt <quadraticCoefficients>} element for constraint (\ref{eq:ro1}).}
\label{figure:qterms}
\end{figure}

The nonlinear part of the problem is given in Figure~\ref{figure:roobjnlnode}.



{\small
\begin{figure}[t]
\centering
   \small {\obeyspaces\let =\
\fbox{\tt\begin{tabular}{@{}l@{}}
<nl idx="-1">\\[\Sb]
     <plus>\\[\Sb]
          <power>\\[\Sb]
               <minus>\\[\Sb]
                    <number value="1.0"/>\\[\Sb]
                    <variable coef="1.0" idx="0"/>\\[\Sb]
               </minus>\\[\Sb]
               <number value="2.0"/>\\[\Sb]
          </power>\\[\Sb]
          <times>\\[\Sb]
               <power>\\[\Sb]
                    <minus>\\[\Sb]
                         <variable coef="1.0" idx="0"/>\\[\Sb]
                         <power>\\[\Sb]
                              <variable coef="1.0" idx="1"/>\\[\Sb]
                              <number value="2.0"/>\\[\Sb]
                         </power>\\[\Sb]
                    </minus>\\[\Sb]
                    <number value="2.0"/>\\[\Sb]
               </power>\\[\Sb]
               <number value="100"/>\\[\Sb]
          </times>\\[\Sb]
     </plus>\\[\Sb]
</nl>\\[\Sb]
\end{tabular} }} \medskip\\[\Sb]
\caption{The {\tt <nl>} element for the nonlinear part of the objective (\ref{eq:roobj}).}\label{figure:roobjnlnode}
\end{figure}
}

The complete OSiL representation can be found in the Appendix (Section~\ref{section:rosenbrockXML}).%
\index{OSiL|)}

\subdivision{OSnL (Optimization Services nonlinear Language)} \label{section:osnlschema}
The OSnL\index{OSnL|(} schema is imported by the OSiL\index{OSiL} schema and is used 
to represent the nonlinear part of an optimization instance. 
This is explained in greater detail in \ifruncode the OS User's Manual\else Section~\ref{section:osexpressiontreeclass}\fi. Also refer to
Figure~\ref{figure:roobjnlnode} for an illustration of elements from the OSnL standard. This figure represents
the nonlinear part of the objective in equation~(\ref{eq:roobj}), that is,
%
$$
(1-x_0)^2 + 100 (x_1-x_0^2)^2.
$$
\index{OSnL|)}


\subdivision{OSrL (Optimization Services result Language)} \label{section:osrlschema}
OSrL\index{OSrL|(} is an XML-based language for representing the solution of large-scale
optimization problems including linear programs, mixed-integer programs,
quadratic programs, and very general nonlinear programs.  An example solution (for the problem given in
 (\ref{eq:roobj})--(\ref{eq:ro3}) ) in OSrL format is given below.

{\small
\begin{verbatim}
<?xml version="1.0" encoding="UTF-8"?>
<?xml-stylesheet type = "text/xsl"
  href = "/Users/kmartin/Documents/files/code/cpp/OScpp/COIN-OSX/OS/stylesheets/OSrL.xslt"?>
<osrl xmlns="os.optimizationservices.org"
      xmlns:xsi="http://www.w3.org/2001/XMLSchema-instance"
      xsi:schemaLocation="os.optimizationservices.org
      http://www.optimizationservices.org/schemas/2.0/OSiL.xsd">
    <general>
        <generalStatus type="normal"/>
        <serviceName>Solved using a LINDO service</serviceName>
        <instanceName>Modified Rosenbrock</instanceName>
    </general>
    <optimization numberOfSolutions="1" numberOfVariables="2" numberOfConstraints="2"
        numberOfObjectives="1">
        <solution targetObjectiveIdx="-1">
            <status type="optimal"/>
            <variables>
                <values numberOfVar="2">
                    <var idx="0">0.87243</var>
                    <var idx="1">0.741417</var>
                </values>
                <other numberOfVar="2" name="reduced costs" description="the variable reduced costs">
                    <var idx="0">-4.06909e-08</var>
                    <var idx="1">0</var>
                </other>
            </variables>
            <objectives>
                <values numberOfObj="1">
                    <obj idx="-1">6.7279</obj>
                </values>
            </objectives>
            <constraints>
                <dualValues numberOfCon="2">
                    <con idx="0">0</con>
                    <con idx="1">0.766294</con>
                </dualValues>
            </constraints>
        </solution>
    </optimization>
\end{verbatim}
}
% Hide this stuff for now...
% The OSrL schema is also used to return timer and system statistics that are sometimes 
% gathered by the solvers themselves or generated as a result of using the {\tt knock} 
% method. (See the example given in Section~\ref{section:knock}.)
\index{OSrL|)}



\subdivision{OSoL (Optimization Services option Language)} \label{section:osolschema}
OSoL\index{OSoL|(} is
an XML-based language for representing options that get passed to an optimization solver or a hosted optimization
solver Web service. It contains both standard options for generic services and extendable option tags for
solver-specific directives.
Several examples of files in OSoL format are presented in Section~\ref{section:servicemethods}.%
\index{OSoL|)}

\subdivision{OSpL (Optimization Services process Language)} \label{section:osplschema}
\index{OSpL|(}This is a standard used to enquire about dynamic process information that 
is kept by the Optimization Services registry. The string passed to the {\tt knock} 
method is in the OSpL format. See the example given in Section~\ref{section:knock}.\index{OSpL|)}




% Part 2 for folks who want to use the libraries to develop their own applications

\throwpage

\pagenumbering{gobble}

\part{Using the OS libraries and API}

\pagenumbering{bychapter}

%\division{Downloading the \ifdevelop OS\else CoinAll\fi  Binaries}\label{section:obtainingbinaries}

\ifdevelop
The OS project is an open-source project  with source code under the Eclipse Public License~(EPL)%
\index{Eclipse Public License (EPL)}.
See~{\tt\UrlEpl}.  This project was initially created by Robert Fourer, Jun Ma, and Kipp Martin.
The code has been written primarily by  Horand Gassmann,   Jun Ma,  and Kipp Martin.    
Horand Gassmann,  Jun Ma,  and Kipp Martin are the COIN-OR project leaders and active developers for the OS project.
\else
The CoinAll project is actually a meta-project consisting of most of the COIN-OR solvers and supporting utility projects.  We describe how to download this project. 
\fi

%Below we describe different methods for obtaining the binaries and C++ source code.
Most users will only be interested in obtaining the binaries, which we describe  next.
%in Section~\ref{section:obtainingbinaries}. The remaining sections of this chapter deal with obtaining %the source code for the project, which will be of interest mostly to developers.
It is also possible to obtain the source code for the project, which will be of interest mostly to developers. 
\ifdevelop
Details can be found in  Section~\ref{section:downloadsource}.
\else
If binaries are not provided for a particular operating system, it may be possible to build them from the source.
For details it is best to start reading the OS web page at~{\tt\UrlOsWiki}.
\fi



%If the user does not wish to compile source code, the OS library, OSSolverService executable
%and Tomcat server software configuration are available in binary format for some operating systems.     
The repository of the binaries is at {\tt\UrlOsBinaries}\index{Downloading!binaries}.
%
\ifdevelop
 Unlike the source code described in Section~\ref{section:downloadwithsvn}, the binary files 
are not subject to version control and can be downloaded using an ordinary browser. 
%If binaries are not provided for a particular operating system,
%it may be possible to build them from the source code. Since the source is under version control, 
%this requires svn. (See Sections \ref{section:svn}, \ref{section:downloadwithsvn} and~\ref{section:build}.)
\fi

The binary distribution for the OS library and executables follows the following naming convention:


\begin{verbatim}
OS-version_number-platform-compiler-build_options.tgz (zip)
\end{verbatim}
For example, OS  Release 2.1.0 compiled with the Intel 9.1 compiler on an Intel 32-bit Linux system is:
\begin{verbatim}
OS-2.1.0-linux-x86-icc9.1.tgz
\end{verbatim}

For more detail on the naming convention and examples see:

\medskip
\noindent{\tt\UrlCoinNames}
\medskip

After unpacking the {\tt tgz} or {\tt zip} archives, the following folders are available.
\begin{itemize}

\item[] {\bf bin --} this directory has the executables {\tt OSSolverService}\index{OSSolverService@{\tt OSSolverService}} 
and {\tt OSAmplClient}\index{OSAmplClient@{\tt OSAmplClient}}.

\item[]  {\bf include --} the header files that are necessary in order to link against the OS library.

\item[] {\bf lib --} the libraries that are necessary for creating applications that use the OS library.

\item[] {\bf  share --} license and author information for all the projects used by the OS project.
\end{itemize}



Files are also provided for an Apache Tomcat\index{Apache Tomcat} Web server along with the associated Web service
that can
read SOAP \index{SOAP protocol} envelopes with model instances in OSiL\index{OSiL} format and/or options in 
OSoL\index{OSoL} format, call the {\tt OSSolverService}\index{OSSolverService@{\tt OSSolverService}},
and return the optimization result in OSrL\index{OSrL} format.
The naming convention\index{file naming conventions} for the server binary is
\begin{verbatim}
OS-server-version_number.tgz (.zip)
\end{verbatim}
For example, the files associated with  OS server release 2.0.0 are in the binary distribution
\begin{verbatim}
OS-server-2.0.0.tgz
\end{verbatim}
There is no platform information given since the server and related binaries were written in Java\index{Java}.
\ifdevelop
The details and use of this distribution are described in Section~\ref{section:tomcat}.
\fi


Finally for Windows users we provide Visual Studio \index{Microsoft Visual Studio} project files 
(and supporting libraries and header files) for building projects based on the OS library and libraries 
used by the OS project. The binary for this is named
\begin{verbatim}
OS-version_number-VisualStudio.zip
\end{verbatim}
For example, the necessary files associated with  OS  stable\index{OS project!stable release} 2.4 
are in the binary distribution
\begin{verbatim}
OS-2.4-VisualStudio.zip
\end{verbatim}
The binaries provided are based on Visual Studio Express 2008.  
\ifdevelop See Section \ref{section:vsexamples} for more detail.\fi


%\throwpage

\section{Code samples to illustrate the OS Project}\label{section:examples}

\ifdevelop
These example executable files are not built by running {\tt configure} and {\tt make}. 
Windows users are advised to download a binary distribution and use the solution file {\tt examples.sln} provided there in the 
\begin{verbatim}
examples\MSVisualStudio
\end{verbatim}
directory.

In order to build the examples in a {\bf unix environment}\index{unix} the user must first run
%
\index{make install@{\tt make install}}
\begin{verbatim}
make install
\end{verbatim}
in the COIN-OS project root directory (the discussion in this section assumes that the project root directory is
{\tt COIN-OS}).  Running {\tt make install}  will  place all the header files required by the examples in the directory
\begin{verbatim}
COIN-OS/include
\end{verbatim}
and all of the libraries required by the examples in the directory
\begin{verbatim}
COIN-OS/lib
\end{verbatim}
In addition the folder {\tt pkgconfig} is placed in the {\tt lib} directory as well. Unix must then be informed of the location of this folder as follows:
\begin{verbatim}
export PKG_CONFIG_PATH=<path to pkgconfig directory>
\end{verbatim}
The source code for the examples is in the {\tt COIN-OS/OS/examples} hierarchy.  For instance, the {\tt osModDemo}
example of section~\ref{section:exampleOSModDemo} is in the directory
\begin{verbatim}
COIN-OS/OS/examples/osModDemo
\end{verbatim}

Next, the user should connect to the appropriate example directory and run {\tt make}.
If the user has done a VPATH\index{VPATH} build, the makefiles\index{makefile|(} will be in each respective example directory under
\begin{verbatim}
vpath_root/OS/examples
\end{verbatim}
otherwise, the makefiles will be in each respective example directory under
\begin{verbatim}
COIN-OS/OS/examples
\end{verbatim}
\else
The binary distribution contains a number of sample applications that illustrate the use of the
OS libraries and other aspects of the OS project. The sample code is found in the {\tt examples}
folder. Each application contains a makefile for unix users; there are also MS Visual Studio project files 
for Windows users. At present only MS Visual Studio 2008 is supported.

Under Windows, connect to the {\tt MSVisualStudio-v9} directory and open {\tt examples.sln} in Visual Studio. All examples can then be built simply by pushing F7 (Build solution). To build only selected examples it is necessary to open the Configuration Manager from the Build menu and select the projects desired to be built.

To build any of the examples under unix, it is at present necessary to set the environment variable
{\tt PKG\_CONFIG\_PATH} to point to the folder {\tt lib/pkgconfig}. Unless some directories were
moved after installing the download, the following unix command will suffice:

\begin{verbatim}
export PKG_CONFIG_PATH=../../lib/pkgconfig
\end{verbatim}

After that, connect to the appropriate directory for the desired project and run {\tt make}. 
For instance, the code and makefile for the {\tt osModDemo}
example of section~\ref{section:exampleOSModDemo} is in the directory
\begin{verbatim}
examples/osModDemo
\end{verbatim}

 
\fi

The {\tt Makefile} in each example directory is fairly simple and is designed to be easily modified 
by the user if necessary.  The part of the Makefile to be adjusted, if necessary, is

\begin{verbatim}
##########################################################################
#    You can modify this example makefile to fit for your own program.   #
#    Usually, you only need to change the five CHANGEME entries below.   #
##########################################################################

# CHANGEME: This should be the name of your executable
EXE = OSModDemo
# CHANGEME: Here is the name of all object files corresponding to the source
#           code that you wrote in order to define the problem statement
OBJS =  OSModDemo.o
# CHANGEME: Additional libraries
ADDLIBS =
# CHANGEME: Additional flags for compilation (e.g., include flags)
ADDINCFLAGS =  -I${prefix}/include
# CHANGEME: SRCDIR is the path to the source code; VPATH is the path to
# the executable. It is assumed that the lib directory is in prefix/lib
# and the header files are in prefix/include
SRCDIR = /Users/kmartin/Documents/files/code/cpp/OScpp/COIN-OS/OS/examples/osModDemo
VPATH = /Users/kmartin/Documents/files/code/cpp/OScpp/COIN-OS/OS/examples/osModDemo
prefix = /Users/kmartin/Documents/files/code/cpp/OScpp/vpath
\end{verbatim}


Developers can use the Makefiles as a starting point for building applications that use the 
OS project libraries\index{makefile|)}.




\subsection{Algorithmic Differentiation:  Using the OS Algorithmic Differentiation Methods}\label{section:cppad}

\index{Algorithmic differentiation|(}
In the {\tt OS/examples/algorithmicDiff} folder is test code {\tt OSAlgorithmicDiffTest.cpp}. This code
illustrates the key methods in the {\tt OSInstance}\index{OSInstance@{\tt OSInstance}} API that are used for
algorithmic differentiation.   These methods are described in Section~\ref{section:ad}.



\subsection{Instance Generator: Using the OSInstance API to Generate Instances}\label{section:exampleOSInstanceGeneration}

This example is found in the {\tt instanceGenerator} folder in the {\tt examples} folder. This example illustrates
how to build a complete in-memory model instance using the {\tt OSInstance}\index{OSInstance@{\tt OSInstance}} API.
See the code {\tt OSInstanceGenerator.cpp} for the complete example. Here we provide a few highlights to illustrate
the power of the API.

The first step is to create an {\tt OSInstance} object.
\begin{verbatim}
OSInstance *osinstance;
osinstance = new OSInstance();
\end{verbatim}

The instance has two variables, $x_{0}$ and $x_{1}$. Variable $x_{0}$ is a continuous variable with lower bound of $-100$ and upper bound of $100$. Variable $x_{1}$ is a binary variable. First declare the instance to have two variables.
\begin{verbatim}
osinstance->setVariableNumber( 2);
\end{verbatim}
Next, add each variable. There is an {\tt addVariable} method with the signature
\begin{verbatim}
addVariable(int index, string name, double lowerBound, double upperBound, char type);
\end{verbatim}
Then the calls for these two variables are
\begin{verbatim}
osinstance->addVariable(0, "x0", -100, 100, 'C');
osinstance->addVariable(1, "x1", 0, 1, 'B');
\end{verbatim}
There is also a method {\tt setVariables} for adding more than one variable simultaneously.  The objective function(s) and constraints are added through similar calls.

Nonlinear terms are also easily added.  The following code illustrates how to add a nonlinear term
$x_{0}*x_{1}$ in the {\tt <nonlinearExpressions>} section of  OSiL. This term is part of constraint~1
and is the second of six constraints contained in the instance.
\begin{verbatim}
osinstance->instanceData->nonlinearExpressions->numberOfNonlinearExpressions = 6;
osinstance->instanceData->nonlinearExpressions->nl = new Nl*[ 6 ];
osinstance->instanceData->nonlinearExpressions->nl[ 1] = new Nl();
osinstance->instanceData->nonlinearExpressions->nl[ 1]->idx = 1;
osinstance->instanceData->nonlinearExpressions->nl[ 1]->osExpressionTree =
new OSExpressionTree();
// the nonlinear expression is stored as a vector of nodes in postfix format
// create a variable nl node for x0
nlNodeVariablePoint = new OSnLNodeVariable();
nlNodeVariablePoint->idx=0;
nlNodeVec.push_back( nlNodeVariablePoint);
// create the nl node for x1
nlNodeVariablePoint = new OSnLNodeVariable();
nlNodeVariablePoint->idx=1;
nlNodeVec.push_back( nlNodeVariablePoint);
// create the nl node for *
nlNodePoint = new OSnLNodeTimes();
nlNodeVec.push_back( nlNodePoint);
// now the expression tree
osinstance->instanceData->nonlinearExpressions->nl[ 1]->osExpressionTree->m_treeRoot =
nlNodeVec[ 0]->createExpressionTreeFromPostfix( nlNodeVec);
\end{verbatim}
\index{Algorithmic differentiation|)}

%\subsection{Excel:  Using VBA To Generate OSiL}\label{section:exampleExcel}

%\subsection{Matlab:  Using  MATLAB To Generate OSiL}\label{section:exampleMatlab}

\subsection{branchCutPrice:  Using Bcp}\label{section:examplebranchCutPrice}

This example illustrates the use of the COIN-OR Bcp (Branch-cut-and-price) project.  
This project offers the user with the ability to have control over each node in the branch and process. 
This makes it possible to add user-defined cuts and/or user-defined variables. At each node in the tree, 
a call is made to the method {\tt process\_lp\_result()}. In the example problem we illustrate 1) adding COIN-OR Cgl cuts, 
2) a user-defined cut, and 3) a user-defined variable. 


\subsection{OSModificationDemo: Modifying an In-Memory {\tt OSInstance} Object}\label{section:exampleOSModDemo}

The {\tt osModificationDemo} folder holds the file {\tt OSModificationDemo.cpp}.
This is similar to the {\tt instanceGenerator} example. In this case, a simple
linear program is generated. However, this example also illustrates how to
modify an in-memory OSInstance object. In particular, we illustrate how to
modify an objective function coeffient. Note the dual occurrence of the
following code

\begin{verbatim}
solver->osinstance->bObjectivesModified = true;
\end{verbatim}

in the {\tt OSModificationDemo.cpp} file (lines 177 and 187).
This line is critical, since otherwise changes made to the OSInstance object
will not be passed to the solver.

This example also illustrates calling a COIN-OR solver,
in this case {\tt Clp}\index{COIN-OR projects!Clp@{\tt Clp}}.

\vskip 8pt

{\bf Important:} the ability to modify a problem instance is still extremely limited in this release.
A better API for problem modification will come with a later release of OS.



\subsection{OSSolverDemo: Building In-Memory Solver and Option Objects}\label{section:exampleOSSolverDemo}

The code in the  example file {\tt OSSolverDemo.cpp} in the folder {\tt osSolverDemo}  illustrates  how to build solver interfaces and  an in-memory {\tt OSOption} object. In this example we  illustrate building a solver interface and corresponding {\tt OSOption} object for the solvers {\tt Clp}, {\tt Cbc}, {\tt SYMPHONY}, {\tt Ipopt},   {\tt Bonmin}, and {\tt Couenne}.   Each solver class inherits from a virtual {\tt OSDefaultSolver} class. Each solver class has the string data members

\begin{itemize}
\item {\tt osil --} this string conforms to the OSiL standard and holds the model instance.

\item {\tt osol --} this string conforms to the OSoL standard and holds an instance with the 
solver options (if there are any); this string can be empty.

\item {\tt osrl --} this string conforms to the OSrL standard and holds the solution instance; 
each solver interface produces an osrl string.
\end{itemize}

Corresponding to each string there is an in-memory object data member, namely

\begin{itemize}
\item {\tt osinstance --}  an in-memory {\tt OSInstance} object containing the model instance
and get() and set() methods to access various parts of the model.


\item {\tt osoption --} an in-memory {\tt OSOption} object; solver options can be accessed or 
set using get() and set() methods.


\item {\tt osresult --}  an in-memory {\tt OSResult} object; various parts of the model solution  
are accessible through get() and set() methods.
\end{itemize}


For each solver we detail five steps:

\begin{itemize}
\item[Step 1:]  Read a model instance from a file  and create the corresponding {\tt OSInstance} object.
For four of the solvers we read a file with the model instance in OSiL format. For the Clp example 
we read an MPS file and convert to OSiL. For the Couenne example we read an AMPL nl file and convert 
to OSiL.

\item[Step 2:]  Create an {\tt OSOption} object and set options appropriate for the given solver.   
This is done by defining

\begin{verbatim}
OSOption* osoption = NULL;
osoption = new OSOption();
\end{verbatim}

A key method in the {\tt OSOption} interface is {\tt setAnotherSolverOption()}.  This method 
takes the following arguments in order.

\begin{itemize}
\item[] {\tt std::string name} -- the option name;
\item[] {\tt std::string value}  -- the value of the option;
\item[] {\tt std::string solver} -- the name of the solver to which the option applies;
\item[] {\tt std::string category} -- options may fall into categories. For example, consider the  
Couenne solver.  This solver is also linked to the Ipopt and Bonmin solvers and  it is possible 
to set options for these solvers through the Couenne API. In order to set an Ipopt option 
you would set the {\tt solver} argument to {\tt couenne} and set the {\tt category} option 
to {\tt ipopt}.

\item[] {\tt std::string type} -- many solvers require knowledge of the data type, so you can set 
the type to {\tt double}, {\tt integer}, {\tt boolean} or {\tt string}, depending on the solver 
requirements. Special types defined by the solver, such as the type {\tt numeric} used by the
Ipopt solver, can also be accommodated. It is the user's responsibility to verify the type
expected by the solver.


\item[] {\tt std::string  description} -- this argument is used to provide any detail or 
additional information about the option. An empty string ({\tt""}) can be passed if such additional
information is not needed.
\end{itemize}

For excellent documentation that details solver options for Bonmin, Cbc, and Ipopt  we recommend 

\begin{center}
\url{http://www.coin-or.org/GAMSlinks/gamscoin.pdf}
\end{center}


\item[Step 3:] Create the solver object. In the OS project there is a {\it virtual} solver that 
is declared by

\begin{verbatim}
DefaultSolver *solver  = NULL;
\end{verbatim}

The Cbc, Clp and SYMPHONY solvers as well as other solvers of linear and integer linear programs
are all invoked by creating a {\tt CoinSolver().} For example, the following is used to invoke Cbc.

\begin{verbatim}
solver = new CoinSolver();
solver->sSolverName ="cbc";
\end{verbatim}

%Then to declare a specific, for example, an {\tt Ipopt} solver, simply write
Other solvers, particularly Ipopt, Bonmin and Couenne are implemented separately. So to declare,
for example, an Ipopt solver, one should write

\begin{verbatim}
solver = new IpoptSolver();
\end{verbatim}

The syntax is the same regardless of solver. 

\item[Step 4:] Import the {\tt OSOption} and {\tt OSInstance} into the solver and solve the model. 
This process is identical regardless of which solver is used. The syntax is:

\begin{verbatim}
solver->osinstance = osinstance;
solver->osoption = osoption;	
solver->solve();
\end{verbatim}

\item[Step 5:] After optimizing the instance,  each of the OS solver interfaces uses the underlying solver API to get the solution result and write the result to a string 
named {\tt osrl} which is a string representing the solution instance in the {\tt OSrL} XML standard.  
This string is accessed by

\begin{verbatim}
solver->osrl
\end{verbatim}


In the example code {\tt OSSolverDemo.cpp} we have written a method,  

\begin{verbatim}
void getOSResult(std::string osrl)
\end{verbatim}

that takes the {\tt osrl} string and creates an {\tt OSResult} object.   
We then illustrate several of the {\tt OSResult} API methods 

\begin{verbatim}
double getOptimalObjValue(int objIdx, int solIdx);
std::vector<IndexValuePair*>  getOptimalPrimalVariableValues(int solIdx);
\end{verbatim}
to get and write out the optimal objective function value, and optimal primal values.  See also Section \ref{section:exampleOSResultDemo}.

\end{itemize}

We now highlight some of the features illustrated by each of the solver examples.

\begin{itemize}
\item {\bf Clp --}  In this example we read in a problem instance in MPS format.  The class 
{\tt OSmps2osil}  has a method {\tt mps2osil} that is used to convert the MPS instance contained 
in a file into an in-memory {\tt OSInstance} object. This example also illustrates how to 
set options using the Osi interface. In particular we turn on intermediate output which is 
turned off by default in the Coin Solver Interface. 

\item {\bf Cbc --}  In this example we read a problem instance that is in OSiL format and create 
an in-memory {\tt OSInstance} object.  We then create an {\tt OSOption} object.  This is quite trivial.  
A  plain-text XML file conforming to the OSiL schema is read into a string {\tt osil} which is then 
converted into the in-memory {\tt OSInstance} object by

\begin{verbatim}
OSiLReader *osilreader = NULL;
OSInstance *osinstance = NULL;
osilreader = new OSiLReader(); 
osinstance = osilreader->readOSiL( osil);
\end{verbatim}


 We set the linear programming algorithm to be the primal simplex method and then set the option 
on the pivot selection to be Dantzig rule.  Finally, we set the print level to be 10.

\item {\bf SYMPHONY --}   In this example we also read a problem instance that is in OSiL format and 
create an in-memory {\tt OSInstance} object.  We then create an {\tt OSOption} object and 
illustrate setting the {\tt verbosity} option.

\item {\bf Ipopt --}   In this example we also read a problem instance that is in OSiL format.  
However, in this case we do  not create an {\tt OSInstance} object. We read the OSiL file into 
a string {\tt osil}.  We then feed the {\tt osil} string directly into the Ipopt solver by
\begin{verbatim}
solver->osil = osil;
\end{verbatim} 
The user always has the option of providing the OSiL to the solver as either a string or in-memory object.

Next we create an {\tt OSOption} object. For Ipopt, we illustrate setting the maximum iteration limit 
and also provide the name of the output file. In addition, the OSOption object can hold initial solution 
values. We illustrate how to initialize all of the variable to 1.0.

\begin{verbatim}
numVar = 2; //rosenbrock mod has two variables 
xinitial = new double[numVar];
for(i = 0; i < numVar; i++){
    xinitial[ i] = 1.0;
}
osoption->setInitVarValuesDense(numVar, xinitial);
\end{verbatim}



\item {\bf Bonmin --}  In this example we read a problem instance that is in OSiL format and create 
an in-memory {\tt OSInstance} object just as was done in the Cbc and SYMPHONY examples.   
We then create an {\tt OSOption} object.  In setting the  {\tt OSOption} object we intentionally 
set an option that will cause the Bonmin solver to terminate early.  In particular we set the 
{\tt node\_limit} to zero. 

\begin{verbatim}
osoption->setAnotherSolverOption("node_limit","0","bonmin","","integer","");
\end{verbatim}

This results in early termination of the algorithm. The {\tt OSResult} class API has a method
\begin{verbatim}
std::string getSolutionStatusDescription(int solIdx);
\end{verbatim}

For this example, invoking
\begin{verbatim}
osresult->getSolutionStatusDescription( 0)
\end{verbatim}
gives the result:
\begin{verbatim}
LIMIT_EXCEEDED[BONMIN]: A resource limit was exceeded, we provide the current solution.
\end{verbatim}


\item {\bf Couenne --}   In this example we read in a problem instance in AMPL nl format.  
The class {\tt OSnl2osil}  has a method {\tt nl2osil} that is used to convert the nl instance 
contained in a file into an in-memory {\tt OSInstance} object. This is done as follows:

\begin{verbatim}
// convert to the OS native format
OSnl2osil *nl2osil = NULL;
nl2osil = new OSnl2osil( nlFileName);
// create the first in-memory OSInstance
nl2osil->createOSInstance() ;
osinstance =  nl2osil->osinstance;
\end{verbatim}
\end{itemize}

This part of the example also illustrates setting options in one solver from another. 
Couenne uses Bonmin which uses Ipopt.  So for example,

\begin{verbatim}
osoption->setAnotherSolverOption("max_iter","100","couenne","ipopt","integer","");
\end{verbatim}
identifies the solver as {\tt couenne}, but the category of value of {\tt ipopt}  tells the solver 
interface to set the iteration limit on the Ipopt algorithm that is solving the continuous relaxation 
of the problem.  Likewise, the setting
\begin{verbatim}
osoption->setAnotherSolverOption("num_resolve_at_node","3","couenne","bonmin","integer","");
\end{verbatim}
identifies the solver as {\tt couenne}, but the category of value of {\tt bonmin}  tells the solver 
interface to tell the Bonmin solver to try three starting points at each node. 

 

\subsection{OSResultDemo: Building In-Memory Result Object to Display Solver Result}\label{section:exampleOSResultDemo}

The OS protocol for representing an optimization result is {\tt OSrL}. Like the {\tt OSiL} and {\tt OSoL} protocol, this protocol has an associated in-memory {\tt OSResult} class with corresponding API.  The use of the API is demonstrated in the code {\tt OSResultDemo.cpp} in the folder {\tt OS/examples/OSResultDemo}.  In the code we solve a linear program with the {\tt Clp} solver.  The OS solver interface builds an {\tt OSrL} string that we read into the {\tt OSrLReader} class and create and {\tt OSResult} object. We then use the {\tt OSResult} API to get the optimal primal and dual solution. We also use the API to get the reduced cost values. 


\subsection{OSCglCuts: Using the OSInstance API to Generate Cutting Planes}\label{section:exampleOSAddCuts}

In this example, we show how to add cuts to tighten an LP using COIN-OR
{\tt Cgl} (Cut Generation Library)\index{COIN-OR projects!Cgl@{\tt Cgl}}.
A file ({\tt p0033.osil}) in OSiL format is used to create an OSInstance object. The linear programming relaxation
is solved. Then, Gomory, simple rounding, and knapsack cuts are added using {\tt Cgl}.  The model is then optimized
using {\tt Cbc}.

\subsection{OSRemoteTest:  Calling a Remote Server}\label{section:exampleOSRemoteTest}

This example illustrates the API for the six service methods described in Section~\ref{section:servicemethods}.
The file {\tt osRemoteTest.cpp} in folder {\tt osRemoteTest} first builds a small linear
example, solves it remotely in synchronous mode and displays the solution.
The asynchronous mode is also tested by submitting the problem to a remote solver,
checking the status and either retrieving the answer or killing the process if it has not
yet finished.

{\bf Windows users should note}
that this project links to {\tt wsock32.lib}, which is not part of the Visual Studio  Express Package.  It is necessary
to also download and install the Windows Platform SDK\index{Windows Platform SDK}, which can be found at

\medskip
\noindent{\scriptsize\tt\UrlSdk}. 
\medskip

\ifdevelop
\noindent See also Section~\ref{section:msvs}.
\else
\noindent Further information is provided in the OS User's Manual.
\fi

\subsection{OSJavaInstanceDemo:  Building an OSiL Instance in
Java}\label{section:exampleOSJavaDemo}
\index{Java|(}

In this example we demonstrate how to build an OSiL instance using the Java
OSInstance API.  The example code also  illustrates calling the {\tt
OSSolverService} executable from Java. In order to use this example, the user should do an svn
checkout:

\begin{verbatim}
svn co https://projects.coin-or.org/svn/OS/branches/OSjava OSjava
\end{verbatim}

The {\tt OSjava} folder contains the file {\tt INSTALL.txt}. Please follow the
instructions in {\tt  INSTALL.txt} under the heading:
\begin{verbatim}
== Install Without a Web Server==
\end{verbatim}

These instructions assume that the user has installed the Eclipse IDE. See
\url{http://www.eclipse.org/downloads/}. At this link we recommend that the 
user get {\tt Eclipse Classic}.  In addition, the user should also have a copy of the
{\tt OSSolverService} executable that is compatible with his or her platform.
The {\tt OSSolverService} executable for several different platforms is
available at \url{http://www.coin-or.org/download/binary/OS/OSSolverService/}. 
The user can also build the executable as described in this Manual.  See Section
\ref{section:build}. The code base for this example is in the folder:
\begin{verbatim}
OSjava/OSJavaExamples/src/OSJavaInstanceDemo.java
\end{verbatim}
The code in the file {\tt OSJavaInstanceDemo.java} demonstrates how the
Java OSInstance API that is in {\tt OSCommon} can be used to generate a linear
program and then call the C++ {\tt OSSolverService} executable 
to solve the problem.\index{Java|)}  Running this example in Eclipse will
generate in the folder
\begin{verbatim}
OSjava/OSJavaExamples
\end{verbatim}
two files. It will generate {\tt parincLinear.osil} which is a linear program in
the OS OSiL format, it will also call the {\tt OSSolverService} executable which
generates the result file {\tt result.osrl} in the OS OSrL format. 




\throwpage

\division{Using Dip (Decomposition In Integer Programming)}\label{section:OSDip}

{\bf Important Note:}  This example uses COIN-OR projects that are not part of the OS distribution and
assumes you have downloaded the {\tt CoinAll }binary.

We follow the notation of Ralphs and Galati~\cite{ralphsgalatiMP}. The integer program of interest is:

\begin{eqnarray}
z_{IP} &=&  \min  \{c^{\top} x \, | \, A^{\prime} x \ge b^{\prime},  \,\, 
A^{\prime \prime} x \ge b^{\prime \prime}, \, \, x \in \mathbb{Z}^{n}  \}
\end{eqnarray}
The problem is divided into two constraint sets, $A^{\prime} x \ge b^{\prime}$
which we refer to as the {\it relaxed}, {\it coupling}, or {\it block constraints}, and the {\it core
constraints} $A^{\prime \prime} x \ge b^{\prime \prime}$.  We then define the
following polyhedron based on the relaxed constraints.
\begin{eqnarray}
{\cal P} &=&  {\rm conv} ( \{ x \in \mathbb{Z}^{n} \, | \, A^{\prime} x \ge
b^{\prime}
\})
\end{eqnarray}
 The LP relaxation of the original problem is:
\begin{eqnarray}
z_{LP} &=&  \min  \{c^{\top} x \, | \, A^{\prime} x \ge b^{\prime},  \,\, 
A^{\prime \prime} x \ge b^{\prime \prime}, \, \, x \in \mathbb{R}^{n} \}
\end{eqnarray}
We also make use of another, related problem $z_D$, defined by
\begin{eqnarray}
z_{D} &=&  \min \{c^{\top} x \, | \, A^{\prime} x \ge b^{\prime},  \,\, 
x \in {\cal P}, \, \, x \in \mathbb{R}^{n} \}.
\end{eqnarray}
Ideally,  the constraints $A^{\prime} x \ge b^{\prime}$
should be selected so that solving $Z_{D}$ is an easy {\it hard problem} and
provides better bounds than $Z_{LP}.$

A generic block-angular decomposition algorithm is now available. 
We employ an implementation that uses the Optimization Services (OS) project together with
another COIN-OR project, Decomposition in Integer Programming (Dip).
%It is 
%based on the Decomposition in Integer Programming (Dip) project jointly with the Optimization Services (OS) project. 
We call this the OS Dip solver.  It has the following features:

\begin{itemize}
\item[1.]  All subproblems are solved via an oracle; either the default oracle
contained in our distribution (see below) or one provided by the user.

\item[2.] The OS Dip Solver code is independent of the oracle used to optimize
the subproblems.

\item[3.] Variables are assigned to blocks using an OS option file; the block
definition and  assignment of variables to these blocks has no effect on the OS
Dip Solver code.

\item[4.] Different blocks can be assigned different solver oracles based on the
option values given in the OSoL file. 

\item[5.] There is a default oracle implemented (called OSDipBlockCoinSolver)
that currently uses Cbc.

\item[6.] Users can add their own oracles without altering the OS Dip Solver
code. This is done via polymorphic factories. The user creates a separate file containing
the oracle class. The user-provided Oracle class  inherits from the generic
OSDipBlockSolver class. The user need only: 1) add the object file name for the
new oracle to the Makefile, and 2) add the necessary line to
OSDipFactoryInitializer.h indicating that the new oracle is present. 

\end{itemize}

In particular, the  implementation of the OS Dip solver provides a virtual class
{\tt OSDipBlockSolver} with a pure virtual function {\tt solve()}.  The user is
expected to provide a class that inherits from {\tt OSDipBlockSolver} and
implements the method {\tt solve()}.  The {\tt solve()} method should optimize a
linear objective function over ${\cal P}.$ More details are provided in Section
\ref{section:osdipsolver}. The implementation is such  that the user only has to
provide a class with a {\tt solve()} method. The user does not have to edit or alter 
any of the OS Dip Solver code.
By using polymorphic factories the actual solver details are hidden from the OS
Solver.  A default solver, {\tt OSDipBlockCoinSolver}, is provided. This default
solver takes no advantage of special structure and simply calls the COIN-OR solver {\tt
Cbc}.
 

\subdivision{Building and Testing the OS-Dip Example}\label{section:buildDip}

Currently, the Decomposition in Integer Programming ({\bf Dip}) package is not a
dependency of the Optimization Services ({\bf OS}) package -- {\bf Dip} is not
included in the {\bf OS} Externals file. In order to run the OS Dip solver it is
necessary to download both the {\bf OS} and {\bf  Dip} projects. Download order is irrelevant. 
In the discussion that follows we assume that for both 
{\bf OS} and {\bf Dip} the user has successfully completed a {\tt
configure}, {\tt make}, and {\tt make install}. We also assume
that the user is working with the trunk version of both {\bf OS} and {\bf Dip.}


The OS Dip solver C++ code is contained in {\tt TemplateApplication/osDip}.
 The {\tt configure}  will create a {\tt Makefile}  in the {\tt
 TemplateApplication/osDip} folder. The {\tt Makefile} must be edited to reflect
 the location of the {\bf Dip} project. The {\tt Makefile} contains the
 line

\begin{verbatim}
DIPPATH = /Users/kmartin/coin/dip-trunk/vpath-debug/
\end{verbatim}

This setting assumes that there is a {\bf lib} directory:

\begin{verbatim}
/Users/kmartin/coin/dip-trunk/vpath-debug/lib
\end{verbatim}
with the {\bf Dip} library that results from {\tt make install} and an {\tt
include} directory
\begin{verbatim}
/Users/kmartin/coin/dip-trunk/vpath/include
\end{verbatim}
with the {\bf Dip} header files generated by {\tt make install}.  The user
should adjust
\begin{verbatim}
/Users/kmartin/coin/dip-trunk/vpath/
\end{verbatim}
to a path containing the {\bf Dip} {\tt lib} and {\tt include} directories. 
After building the executable by executing the {\tt make} command,
 run the {\tt osdip} application using the command:

\begin{verbatim}
./osdip --param osdip.parm
\end{verbatim}

This should produce the following output.


\begin{verbatim}
FINISH SOLVE
Status= 0 BestLB= 16.00000   BestUB= 16.00000   Nodes= 1      
SetupCPU= 0.01 SolveCPU= 0.10 TotalCPU= 0.11 SetupReal= 0.08 
SetupReal= 0.12 TotalReal= 0.16
Optimal Solution
-------------------------
Quality = 16.00
0      1.00
1      1.00
12     1.00
13     1.00
14     1.00
15     1.00
17     1.00

\end{verbatim}

If you see this output,  things are working properly. 
%If this doesn't work, I almost certainly did something stupid and forget to fix it.  

The file
{\tt osdip.parm} is a parameter file. The use of the parameter file is 
explained in Section \ref{section:parameterfile}.


\subdivision{The OS Dip Solver -- Code Description and
Key Classes}\label{section:osdipsolver}

The OS Dip Solver uses {\bf Dip} to implement a Dantzig-Wofe decomposition
algorithm for block-angular integer programs. Here are some key classes.




\vskip 8pt
\noindent {\bf OSDipBlockSolver:}  This is a virtual class with a pure virtual
function: 

\begin{verbatim}
void solve(double *cost, std::vector<IndexValuePair*> *solIndexValPair,
double *optVal)
\end{verbatim}



\vskip 8pt
\noindent {\bf OSDipBlockSolverFactory:}  This is also virtual class with a pure
virtual function: 

\begin{verbatim}
OSDipBlockSolver* create()
\end{verbatim}

This class also has the static method

\begin{verbatim}
OSDipBlockSolver* createOSDipBlockSolver(const string &solverName)
\end{verbatim}

and a map

\begin{verbatim}
std::map<std::string, OSDipBlockSolverFactory*> factories;
\end{verbatim}


\vskip 8pt
\noindent {\bf Factory:}  This class inherits from the class {\bf
OSDipBlockSolverFactory}. Every sover class that inherits from the  {\bf
OSDipBlockSolver} class should have a {\bf Factory} class member and since
this {\bf Factory} class member inherits from the {\bf
OSDipBlockSolverFactory} class it should implement a {\tt create()} method that
creates an object in the class inheriting from {\bf
OSDipBlockSolver}.

\vskip 8pt
\noindent {\bf OSDipFactoryInitializer:}  This class initializes the static map

\begin{verbatim}
OSDipBlockSolverFactory::factories
\end{verbatim}
in the {\bf OSDipBlockSolverFactory} class. 

\vskip 8pt
\noindent {\bf OSDipApp:}  This class inherits from the {\bf Dip} class {\tt
DecompApp}. In {\bf OSDipApp} we implement methods for creating the core
(coupling) constraints, i.e., the constraints $A^{\prime \prime} x \ge
b^{\prime \prime}$.  This is done by implementing the  {\tt createModels()}
method. Regardless of the problem, none of the relaxed or block constraints in $A^{\prime } x \ge
b^{\prime}$ are created. These are treated implicitly in the solver class that
inherits from the class {\bf OSDipBlockSolver.}  This class also implements a
method that defines the variables that appear only in the blocks ({\bf
createModelMasterOnlys2}), and a method for generating an initial master (the
method {\bf generateInitVars()  }). 

Since the constraints $A^{\prime } x \ge
b^{\prime}$ are treated explicitly by the Dip solver the {\tt solveRelaxed()}
method must be implemented. In our implementation we have the {\bf OSDipApp} class data
member
\begin{verbatim}
std::vector<OSDipBlockSolver* > m_osDipBlockSolver;
\end{verbatim}
when the {\tt solveRelaxed()} method is called for block {\tt whichBlock} in
turn we make the call
\begin{verbatim}
m_osDipBlockSolver[whichBlock]->solve(cost, &solIndexValPair, &varRedCost);
\end{verbatim}
and the appropriate solver in class {\bf OSDipBlockSolver} is called. Finally,
the {\bf OSDipApp} class also  initiates the reading of the OS option and
instance files. How these files are used is discussed in Section \ref{section:defineinstance}. 
Based on option input
data this class also creates the appropriate solver object for each block, i.e.,
it populates the {\tt  m\_osDipBlockSolver} vector.

\vskip 8pt
\noindent {\bf OSDipInterface:} This class is used  as an interface between the
{\bf OSDipApp} class and classes in the {\bf OS} library. This provides a number
of get methods to provide information to {\bf OSDipApp} such as the coefficients
in the $A^{\prime \prime}$ matrix, objective function coefficients, number of
blocks etc. The {\bf OSDipInterface} class reads the input OSiL and OSoL files
and creates in-memory data structures based on these files. 


  
\vskip 8pt
\noindent {\bf OSDipBlockCoinSolver:}  This class inherits from the {\bf
OSDipBlockSolver} class. It is meant to illustrate how to create a solver class.
This class solves each block by calling {\bf Cbc}.  Use of this class provides a
generic block angular decomposition algorithm.



\vskip 8pt
There is also  {\bf OSDip\_Main.cpp:} which contains the {\tt main()} routine and is
the entry point for the executable. It first creates a new price-branch-and-cut
decomposition algorithm and then an {\tt Alps} solver for which the {\tt solve()}
method is called. 


\subdivision{User Requirements}\label{section:userreq}


The {\bf OSDipBlockCoinSolver} class provides a solve method for optimizing a
linear objective function over ${\cal P}$ given a linear objective function.
However, this takes no advantage of the special structure available in the
blocks. Therefore, the user may wish to implement his or her own solver class.
In this case the user is required to do the following:
 
 \begin{itemize}
   
   \item[1.] implement a class that inherits from the {\bf OSDipBlockSolver}
   class and implements the solve method,
   
   \item[2.] implement a class {\bf Factory} that inherits from the class {\bf
OSDipBlockSolverFactory} and implements the {\tt create()} method,

	\item[3.] edit the file {\bf OSDipFactoryInitializer.h} and add a line:
	
	\begin{verbatim}
	OSDipBlockSolverFactory::factories["MyBlockSolver"] = new
	MyBlockSolver::Factory;
	\end{verbatim}
   
   \item[4.] alter the Makefile to include the new source code.
 \end{itemize}
 
 \vskip 8pt
 
 {\bf Important -- Directory Structure:} In order to keep things clean, there is
 a directory {\bf solvers} in the {\bf osDip} folder. We suggest using the {\bf
 solvers} directory for all of the solvers that inherit from {\bf
 OSDipBlockSolver}.
 
 \subdivision{Simple Plant/Lockbox Location Example}


 The problem is to minimize
the sum of the cost of capital due to float  and the cost of operating the lock boxes.  

\noindent {\bf Parameters:}
\begin{itemize}
\item[]  $m -$ number of customers to be assigned a lock box

\item[]  $n -$ number of potential lock box sites

\item[]  $c_{ij} -$ annual cost of capital associated with serving customer $j$ from lock box $i$ 

\item[]  $f_{i} -$  annual fixed cost of operating a lock box at location $i$
\end{itemize}

\noindent {\bf Variables:}
\begin{itemize}

\item[]  $x_{ij} - $ a binary variable which is equal to 1 if customer $j$ is assigned to lock box $i$
and 0 if not

\item[]  $y_{i} - $ a binary variable which is equal to 1 if the lock box at location $i$ is opened and 0 if
not

\end{itemize}
The   integer linear program  for the lock box location problem is
$$
\eqnarrayx{
  & \min  &\sum_{i = 1}^{n} \sum_{j = 1}^{m} c_{ij} x_{ij}& + &\sum_{i = 1}^{n} f_{i} y_{i} &&&&&
\eq{eq:lockobj} \cr
(LB) &&x_{ij} - y_{i} &\le& 0, & i = 1, \ldots, n, & j = 1, \ldots, m
&&&\eq{eq:locksetup} \cr  &{\rm s.t.} & \sum_{i = 1}^{n} x_{ij} &=& 1, & j = 1, \ldots, m &&&&\eq{eq:lockdemand} \cr
&& x_{ij}, \, \, y_{i} &\in& \{ 0, 1 \}, & i = 1, \ldots, n, & j = 1, \ldots, m. &&&\eq{eq:lockbinary}
\cr
}
$$

The objective (\ref{eq:lockobj}) is to minimize the sum of the cost of capital plus the fixed cost of
operating the lock boxes.   Constraints (\ref{eq:locksetup})  are forcing 
constraints and require that a lock box be open if a customer is served by that
lock box. For now, we consider these the $A^{\prime} x \ge b^{\prime}$
constraints.  The requirement that every customer be assigned a lock box is
modeled by constraints (\ref{eq:lockdemand}).  For now, we consider these the
$A^{\prime \prime} x \ge b^{\prime \prime}$ constraints.

\vskip 12pt
{\bf Location Example 1:} A three plant, five customer model.

\vskip 8pt

\begin{table}[ht]
\centering
\vskip 8pt
\begin{tabular}{|cc|c|c|} \hline
       &    & CUSTOMER &         \\
      &     &\begin{tabular}{ccccc}
             1&2&3&4&5 \end{tabular} & FIXED COSTS  \\ \hline
     &   1   &\begin{tabular}{ccccc}
             2&3&4&5&7 \end{tabular} &   2  \\
 PLANT & 2   &\begin{tabular}{ccccc}
             4  &  3  &  1  &  2  &  6 \end{tabular} &  3  \\    
       & 3   &\begin{tabular}{ccccc}
            5   &  4  &  2  &  1  &  3 \end{tabular} &  3  \\   \hline
\end{tabular}  
\caption{Data for a 3 plant, 5 customer problem} 
\label{table:spl3by5data}  
\end{table}  
  

\vskip 10pt
\begin{eqnarray*}
\min  && 2x_{11} +3 x_{12} + 4x_{13} + 5x_{14}+ 7x_{15} + 2 y_{1} + \\
&& 4x_{21} +3 x_{22} + x_{23} + 2x_{24}+  6x_{25} + 3y_{2}+ \\
&& 5x_{31} +4 x_{32} + 2 x_{33} + x_{34}+  3x_{35} +   3y_{3} \\
\end{eqnarray*}


\begin{eqnarray*}
\begin{array}{lll}
x_{11}\leq y_{1}\leq 1 & &  \\
x_{12}\leq y_{1}\leq 1 & & \\
x_{13}\leq y_{1}\leq 1 & & \\
x_{14}\leq y_{1}\leq 1 & & \\
x_{15}\leq y_{1}\leq 1 & & \\
x_{21}\leq y_{2}\leq 1 & & \\
x_{22}\leq y_{2}\leq 1 & &   \\
x_{23}\leq y_{2}\leq 1 & & \\
x_{24}\leq y_{2}\leq 1 & & \\
x_{25}\leq y_{2}\leq 1 & & \\
x_{31}\leq y_{3}\leq 1 & & \\
x_{32}\leq y_{3}\leq 1 & &\\
x_{33}\leq y_{3}\leq 1 & &\\ 
x_{33}\leq y_{3}\leq 1 & &\\ 
x_{33}\leq y_{3}\leq 1 & &\\ 
\end{array}
 A^{\prime }x \ge b^{\prime} \,\, {\rm constraints} \\
x_{ij},y_{i}\ge 0 , \,\, i = 1, \ldots, n, \, \, j = 1, \ldots, m.   
\end{eqnarray*}

 
\[
\begin{array}{llll}
{\rm s.t.} &x_{11}+x_{21}+x_{31}  = 1 & & \\
&x_{12}+x_{22}+x_{32} = 1 & &   \\
&x_{13}+x_{23}+x_{33} = 1 & &  \\
&x_{14}+x_{24}+x_{34} = 1 & &  \\
&x_{15}+x_{25}+x_{35} = 1 & &  
\end{array}   A^{\prime \prime}   x \ge b^{\prime \prime} \,\, {\rm
constraints}
\]
  
  


                      


\vskip 12pt

{\bf Location Example 2 (SPL2):} A three plant, three customer model.

\vskip 8pt


\begin{table}[ht]
\centering
\begin{tabular}{|cc|c|c|} \hline
       &    & CUSTOMER &         \\
      &     &\begin{tabular}{ccc}
             1&2&3 
             \end{tabular} & FIXED COSTS  \\ \hline
     &   1   &\begin{tabular}{ccc}
             2&1&1 
             \end{tabular} &   1  \\
 PLANT & 2   &\begin{tabular}{ccc}
             1  &  2  &  1   
             \end{tabular} &  1  \\    
       & 3   &\begin{tabular}{ccc}
            1   &  1  &  2   
            \end{tabular} &  1  \\   \hline
\end{tabular} 
\caption{Data for a three plant, three customer problem} 
\label{table:spl3by3data}   
\end{table}  

  
\vskip 8pt
\begin{eqnarray*}
\min  && 2x_{11} + x_{12} + x_{13}  +  y_{1} + \\
&& x_{21} +2 x_{22} + x_{23} +       y_{2}+ \\
&& x_{31} + x_{32} + 1 x_{33} +  +       y_{3} \\
\end{eqnarray*}


  
\begin{eqnarray*}
\begin{array}{lll}
x_{11}\leq y_{1}\leq 1 & &  \\
x_{12}\leq y_{1}\leq 1 & & \\
x_{13}\leq y_{1}\leq 1 & & \\
x_{21}\leq y_{2}\leq 1 & & \\
x_{22}\leq y_{2}\leq 1 & &   \\
x_{23}\leq y_{2}\leq 1 & & \\
x_{31}\leq y_{3}\leq 1 & & \\
x_{32}\leq y_{3}\leq 1 & &\\
x_{33}\leq y_{3}\leq 1 & &\\ 
\end{array}
A^{\prime}   x \ge b^{\prime} \,\, {\rm
constraints} \\
x_{ij},y_{i}\ge 0 , \,\, i = 1, \ldots, n, \, \, j = 1, \ldots, m.   
\end{eqnarray*}

\[
\begin{array}{llll}
{\rm s.t.} &x_{11}+x_{21}+x_{31} = 1 & & \\
&x_{12}+x_{22}+x_{32} = 1 & &   \\
&x_{13}+x_{23}+x_{33} = 1 & &
\end{array}  A^{\prime \prime}   x \ge b^{\prime \prime} \,\, {\rm
constraints}
  \]
  







\subdivision{Generalized Assignment Problem Example}\label{section:genass}

A problem that plays a prominent role in
vehicle routing is the {\it generalized assignment problem.}    The problem is to assign each of $n$
tasks to $m$ servers without exceeding the resource capacity of the servers.

\noindent{\bf Parameters:}
\begin{itemize}
\item[]  $n -$ number of required tasks
\item[]  $m -$   number of servers
\item[]  $f_{ij} -$ cost of assigning task $i$ to server $j$
\item[]  $b_{j} -$  units of resource available to server $j$
\item[]  $a_{ij} -$ units of server $j$ resource required to perform task $i$
\end{itemize}

\noindent{\bf Variables:}
\begin{itemize}
\item[]  $x_{ij} -$ a binary variable which is equal to 1 if task $i$ is assigned to server $j$
and 0 if not
\end{itemize}
The integer linear program for the generalized assignment problem  is 
$$
\eqnarrayx{
&  \min &\sum_{i = 1}^{n} \sum_{j = 1}^{m} f_{ij} x_{ij} &&&&&&& \eq{eq:gapobj} \cr
(GAP) &{\rm s.t.}& \sum_{j = 1}^{m} x_{ij} &=& 1, & i = 1, \ldots, n  &&&& \eq{eq:gapassign} \cr
&& \sum_{i = 1}^{n} a_{ij} x_{ij} &\le& b_{j}, &j = 1, \ldots, m  &&&&\eq{eq:gapcapacity}  \cr
&& x_{ij} &\in& \{ 0, 1 \}, & i = 1, \ldots, n, & j = 1, \ldots, m.  &&&
\eq{eq:gapbinary}  \cr
}
$$

The objective function (\ref{eq:gapobj}) is to minimize the total assignment cost.  Constraint
(\ref{eq:gapassign}) requires that each task is assigned a server.  These
constraints correspond to the $A^{\prime \prime} x \ge b^{\prime \prime}$
constraints.   The requirement that the server capacity not be exceeded is given
in (\ref{eq:gapcapacity}). These correspond to the $A^{\prime} x \ge
b^{\prime}$ constraints that are used to define ${\cal P}$. The test problem
used in the file {\tt genAssign.osil} is:


 
\begin{eqnarray*}
{\rm min} \quad 2 x_{11} + 11 x_{12} + 7 x_{21} + 7 x_{22} && \\
+ 20 x_{31} + 2 x_{32} + 5 x_{41} + 5x_{42} && \\
x_{11} + x_{12}  &=&    1  \\
x_{21} + x_{22}  &=&    1 \\
x_{31} + x_{32} &=&    1 \\
x_{41} + x_{42} &=&    1 \\
3 x_{11} + 6 x_{21} + 5 x_{31} + 7 x_{41} &\le&   13 \\
2 x_{12} + 4 x_{22} + 10 x_{32} + 4 x_{42} &\le&   10
\end{eqnarray*}
 

\subdivision{Defining the Problem Instance and Blocks}\label{section:defineinstance}

Here we describe how to use the OSOption and OSInstance formats.  We illustrate
with a simple plant location problem. Refer back to the example in Table
\ref{table:spl3by5data} for a three-plant, five-customer problem. We treat the
fixed charge constraints as the block constraints, i.e., we treat constraint set
(\ref{eq:locksetup}) as the set $A^{\prime} x \ge b^{\prime}$ constraints. These
constraints naturally break into a block for each plant, i.e., there is a block
of constraints:
\begin{eqnarray}
x_{ij} \le y_{i}
\end{eqnarray}
In order to use the OS Dip solver it is necessary to: 1) define the set of
variables in each block and 2) define the set of constraints that constitute the
core or coupling constraints. This information is communicated to the OS Dip
solver using Optimization Services option Language (OSoL). The OSoL input file
for the example in  Table \ref{table:spl3by5data} appears in Figures
\ref{figure:parinc-osil} and \ref{figure:parinc-osil2}.  See lines 32-55. There
is an {\tt <other>} option with {\tt name="variableBlockSet"} for each block.
Each block then lists the variables in the block. For example, the first block
consists of the variables indexed by 0, 1, 2, 3, 4, and 15. These correspond to
variables $x_{11},$  $x_{12},$  $x_{13},$  $x_{13},$ $x_{14},$ and  $y_{1}.$
Likewise the second block corresponds to the variable for the second plant and
the third block corresponds to variables for the third plant.


  

{\small
\begin{figure}[hp]
   \small {\obeyspaces\let =\
\makebox[0in][t]{\fbox{\tt\begin{tabular}{@{}l@{}}
1   <?xml version="1.0" encoding="UTF-8"?>\\
2   <osol>\\
3      <general>\\
4         <instanceName>spl1 -- setup constraints are the blocks</instanceName>\\
5      </general>\\
6      <optimization>\\
7         <variables numberOfOtherVariableOptions="6">\\
8            <other name="initialCol" solver="Dip" numberOfVar="6" value="0">\\
9               <var idx="0" value="1"/>\\
10              <var idx="1" value="1"/>\\
11              <var idx="2" value="1"/>\\
12              <var idx="3" value="1"/>\\
13              <var idx="4" value="1"/>\\
14              <var idx="15" value="1"/>\\
15           </other>\\
16           <other name="initialCol" solver="Dip" numberOfVar="6" value="1">\\
17              <var idx="5" value="1"/>\\
18              <var idx="6" value="1"/>\\
19              <var idx="7" value="1"/>\\
20              <var idx="8" value="1"/>\\
21              <var idx="9" value="1"/>\\
22              <var idx="16" value="1"/>\\
23           </other>\\
24           <other name="initialCol" solver="Dip" numberOfVar="6" value="2">\\
25              <var idx="10" value="1"/>\\
26              <var idx="11" value="1"/>\\
27              <var idx="12" value="1"/>\\
28              <var idx="13" value="1"/>\\
29              <var idx="14" value="1"/>\\
30              <var idx="17" value="1"/>\\
31           </other>\\
32           <other name="variableBlockSet" solver="Dip" numberOfVar="6" value="MySolver1">\\
33              <var idx="0"/>\\
34              <var idx="1"/>\\
35              <var idx="2"/>\\
36              <var idx="3"/>\\
37              <var idx="4"/>\\
38              <var idx="15"/>\\
39           </other>\\
40           <other name="variableBlockSet" solver="Dip" numberOfVar="6" value="MySolver2">\\
41              <var idx="5"/>\\
42              <var idx="6"/>\\
43              <var idx="7"/>\\
44              <var idx="8"/>\\
45              <var idx="9"/>\\
46              <var idx="16"/>\\
47           </other>\\
\end{tabular} }}} \medskip
\caption{A sample OSoL file -- SPL1.osol}\label{figure:parinc-osil}
\end{figure}
} %end small


It is also necessary to convey which constraints constitute the core
constraints. This is done in lines 58-64. The core constraints are indexed by
15, 16, 17, 18, 19. These constitute the demand constraints given in Equation
(\ref{eq:lockdemand}). 


Notice also that in lines 32, 40, and 48 there is an attribute {\tt value} in
the {\tt <other>} variable element with the attribute {\tt name} equal to {\tt
variableBlockSet}.  The attribute {\tt value} should be the name of the solver
factory that should be assigned to solve that block. For example, if the
optimization problem that results from solving a linear objective over the
constraints defining the first block is solved using {\tt MySolver1} then this
must correspond to a 

\begin{verbatim}
OSDipBlockSolverFactory::factories["MySolver1"] = new
MySolver1::Factory;
\end{verbatim}  

in the file {\bf OSDipFactoryInitializer.h}.  In the test file, {\tt spl1.osol}
for the first block we set the solver to a specialized solver for the simple
plant location problem ({\tt OSDipBlockSplSolver}) and for the other two blocks
we use the generic solver ({\tt OSDipBlockCoinSolver}).




{\small
\begin{figure}[hp]
   \small {\obeyspaces\let =\
\makebox[0in][t]{\fbox{\tt\begin{tabular}{@{}l@{}}
48           <other name="variableBlockSet" solver="Dip" numberOfVar="6" value="MySolver3">\\
49              <var idx="10"/>\\
50              <var idx="11"/>\\
51              <var idx="12"/>\\
52              <var idx="13"/>\\
53              <var idx="14"/>\\
54              <var idx="17"/>\\
55           </other>\\
56        </variables>\\
57        <constraints numberOfOtherConstraintOptions="1">\\
58           <other name="constraintSet" solver="Dip" numberOfCon="5" type="Core">\\
59              <con idx="15"/>\\
60              <con idx="16"/>\\
61              <con idx="17"/>\\
62              <con idx="18"/>\\
63              <con idx="19"/>\\
64           </other>\\
65        </constraints>\\
66     </optimization>\\
67  </osol>\\
\end{tabular} }}} \medskip
\caption{A sample OSoL file -- SPL1.osol (Continued)}\label{figure:parinc-osil2}
\end{figure}
} %end small
 
One can use the OSoL file to specify a set of starting columns for the initial
restricted master. In Figure \ref{figure:parinc-osil} see lines 8-31.  In an OS
option file (OSoL) there is {\tt <variables>} element that has {\tt <other>}
children. Initial columns are specified using the {\tt <other>} elements. This
is done by using  the {\tt name} attribute and setting its value to {\tt
initialCol}. Then the children of the tag contain index-value pairs that specify
the column. For example, the first initial column corresponds to setting:

\begin{eqnarray*}
x_{11} = 1, \quad  x_{12} = 1, \quad  x_{13} = 1, \quad  x_{14} = 1, \quad
x_{15} = 1, \quad y_{1} = 1
\end{eqnarray*}


Finally note that in all of this discussion we know to apply the options to {\bf
Dip} because the attribute {\tt solver} always had value {\tt Dip}. It is
critical to set this attribute in all of the option tags. 


\subdivision{The Dip Parameter File}\label{section:parameterfile}

The {\bf Dip} solver has a utility class {\bf UtilParameters},  for parsing
a parameter file. The {\bf UtilParameters} class constructor takes a parameter
file as an argument. In the case of the OS Dip solver the name of the parameter
file is {\bf osdip.parm} and the parameter file is read in at the command line with
the command

\begin{verbatim}
./osdip -param osdip.parm
\end{verbatim}

The {\bf UtilParameters} class has a method {\bf GetSetting()} for reading the
parameter values. In the OS Dip implementation there is a class {\bf OSDipParam}
that has as data members key parameters such as the name of the input OSiL file
and input OSoL file. The {\bf OSDipParam} class has a method
{\bf getSettings()} that takes as an argument a pointer to an object in the {\bf
UtilParameters} and uses the {\bf GetSetting()} method to return the relevant
parameter values. For example:

\begin{verbatim}
OSiLFile = utilParam.GetSetting("OSiLFile", "", common); 
OSoLFile = utilParam.GetSetting("OSoLFile", "", common);
\end{verbatim}

In the current {\bf osdip.parm} file we have:

\begin{verbatim}

#first simple plant location problem
OSiLFile = spl1.osil
#setup constraints as blocks
OSoLFile = spl1.osol
#assignment constraints as blocks
#OSoLFile = spl1-b.osol

#second simple plant location problem
#OSiLFile = spl2.osil
#setup constraints as blocks
#OSoLFile = spl2.osol
#assignment constraints as blocks
#OSoLFile = spl2-b.osol

#third simple plant location problem -- block matrix data not used
#OSiLFile = spl3.osil
#setup constraints as blocks
#OSoLFile = spl3.osol

#generalized assignment problem
#OSiLFile = genAssign.osil
#OSoLFile = genAssign.osol

#Martin textbook example
#OSiLFile = smallIPBook.osil
#OSoLFile = smallIPBook.osol
\end{verbatim}
 
By commenting and uncommenting you can run one of four problems that are in the
{\bf data} directory. The first example, {\bf spl1.osil}, corresponds to the
simple plant location model given in Table \ref{table:spl3by5data}. Using the
option file {\bf spl1.osol} treats the setup forcing constraints
\ref{eq:locksetup} as the $A^{\prime} x \ge b^{\prime}$ constraints. Using the
option file {\bf spl1-b.osol} treats the demand constraints
\ref{eq:lockdemand} as the $A^{\prime} x \ge b^{\prime}$ constraints. Likewise
for the problem {\bf spl2.osil} which correponds to the simple plant location
data given in Table \ref{table:spl3by3data}.

In both examples {\bf spl1.osil} and {\bf spl2.osil} the $A^{\prime} x \ge
b^{\prime}$ constraints are explicitly represented in the OSiL file. However,
this is not necessary. The solver Factory {\bf OSDipBlockSlpSolver} is a special
oracle that only needs the objective function coefficients and pegs variables
based on the sign of the objective function coefficients. The {\bf spl3.osil} is
the example given in Table \ref{table:spl3by5data} but without the setup forcing
constraints. Each block uses the {\bf OSDipBlockSlpSolver} oracle. 

The {\bf genAssign.osil} file corresponds to the generalized assignment problem
given in Section \ref{section:genass}.  The option file {\bf genAssign.osol} 
treats the capacity constraints \ref{eq:gapcapacity} as the $A^{\prime} x \ge
b^{\prime}$ constraints. 

The last problem defined in the file {\bf smallIPBook.osil} is based on Example
16.3 on page 567 in {\it Large Scale Linear and Integer Optimization}.  The
option file treats the constraints
$$
4x_{1} + 9 x_{2} \le 18, \quad -2x_{1} + 4 x_{2} \le 4
$$
as the $A^{\prime} x \ge b^{\prime}$ constraints.

The user should also be aware of the parameter {\tt solverFactory}. This
parameter is the name of the default solver Factory. If a solver is not named
for a block in the OSoL file this value is used. We have set the value of this
string to be {\tt OSDipBlockCoinSolver}.


   


\subdivision{Issues to Fix}

\begin{itemize}
  \item Enhance solveRelaxed to allow parallel processing of blocks. See ticket
  30.
  \item Does not work when there are 0 integer variables. See ticket 31.
  
  \item Be able to set options in C++ code. See ticket 41.  It would be nice to
  be able to read all the options from a generic options file. It seems like
  right now options for the {\bf DecompAlgo } class cannot be set inside C++.
  
  \item Problem with Alps  bounds at node 0. See ticket 43
  
  \item Figure out how to use BranchEnforceInMaster or BranchEnforceInSubProb so
  I don't get the large bonds on the variables. See ticket 47.
\end{itemize}



\subdivision{Miscellaneous Issues}
 
 If you want to terminate at the root node and just get the dual value under the {\tt ALPS } option put:
 
\begin{verbatim}
[ALPS]
nodeLimit = 1
\end{verbatim}

More from Matt:



\begin{verbatim}
Kipp - the example you sent finds the optimal solution after a few passes of pricing and therefore never calls the cut generator. By default, the PC solver, in the root node starts with pricing, and does not stop until it prices out (or finds optimal, or within gap limits).

If it prices out and has not yet found optimal, then it will proceed to cuts.

This is parameter driven.


You'll see in the log file (LogDebugLevel = 3),
PRICE_AND_CUT  LimitRoundCutIters       2147483647
PRICE_AND_CUT  LimitRoundPriceIters     2147483647

This is the number of Price/Cut iterations to take before switching off (i.e., MAXINT).

To force it to cut before pricing out, change this parameter in the parm file. For example, if you change to :

[DECOMP]
LimitRoundPriceIters = 1
LimitRoundCutIters   = 1

It will then go into your generateCuts after one pricing iteration.

\vskip 12pt

If there is an integer solution at the root node, it may be the case that we are still not optimal. A perfect example is where you want to add tour-breaking constraints. There could be an integer solution, but you still violate a tour-breaking constraint. Here is what Matt says:
``By default, DIP assumes, that if problem is LP feasible to the linear system and IP feasible, then it is feasible. In the case where the user knows something that DIP does not (e.g., that the linear system does not define the entire valid constraint system, as in TSP), then they must provide a derivation of this function APPisUserFeasible. Then, DIP will check LP feasible, IP feasible and lastly, APPisUserFeasible before declaring a point a feasible solution.''

For an example of using this see, \url{https://projects.coin-or.org/Dip/browser/trunk/Dip/examples/TSP/TSP_DecompApp.cpp}.

\end{verbatim}


\throwpage

\division{The OS Library Components}\label{section:oslibrary}\index{OSLibrary@{\tt OSLibrary}|(} 

\subdivision{OSAgent}\label{section:osagent}

The {\tt OSAgent}\index{OSAgent@{\tt OSAgent}|(}  part of the library is used to facilitate communication
with remote solvers. It is not used if the solver is invoked locally (i.e., on the same machine).
There are two key classes in the {\tt OSAgent} component of the OS library. The two classes are
{\tt OSSolverAgent}\index{OSSolverAgent@{\tt OSSolverAgent}} and {\tt WSUtil}\index{WSUtil@{\tt WSUtil}}.

The {\tt OSSolverAgent} class is used to contact a remote solver service.  For example, assume that {\tt sOSiL}
is a string with a problem instance and {\tt sOSoL} is a string with solver options. Then the following code
will call a solver service and invoke the {\tt solve} method.
\begin{verbatim}
OSSolverAgent *osagent;
string serviceLocation = "http://xxx.xxx.xxx.xxx:8080/OSServer/services/OSSolverService";
osagent = new OSSolverAgent(  serviceLocation );
string sOSrL = osagent->solve(sOSiL, sOSoL);
\end{verbatim}
(The string `{\tt xxx.xxx.xxx.xxx}' must be replaced by the IP address of a computer running the {\tt OSSolverService}.)
Other methods in the {\tt OSSolverAgent} class are {\tt send}, {\tt retrieve}, {\tt getJobID}, {\tt knock}, and {\tt kill}.  
The use of these methods is described in Section~\ref{section:servicemethods}.



The methods in the {\tt OSSolverAgent} class call methods in the {\tt WSUtil} class that perform such tasks as 
creating and parsing SOAP messages and making low level socket calls to the server running the solver service. 
The average user will not use methods in the {\tt WSUtil} class, but they are available to anyone wanting to make socket calls or create SOAP messages.

There is also a method, {\tt OSFileUpload}, in the OSAgentClass that is used to upload files from the hard drive of a client to the server. 
It is very fast and does not involve SOAP or Web Services. The {\tt OSFileUpload}  method is illustrated and described in the example code 
{\tt OSFileUpload.cpp} described in Section~\ref{section:fileupload}.
\index{OSAgent@{\tt OSAgent}|)}

\subdivision{OSCommonInterfaces}

The classes in the OSCommonInterfaces component of the OS library are used to read and write files and strings
in the OSiL and OSrL protocols. See Section~\ref{section:schemadescriptions} for more detail on OSiL, OSrL,
and other OS protocols. For a complete listing of all of the files in {\tt OSCommonInterfaces} see the 
Doxygen\index{Doxygen} documentation we deposited at {\tt\UrlDoxygen}. Users who have 
Doxygen installed on their system can also create their own version of the documentation 
(see \ifdevelop Section~\ref{section:documentation}\else the OS Users' Manual\fi). 
Below we highlight some key classes.





\subsubdivision{The OSInstance Class}\label{section:osinstanceclass}

The OSInstance\index{OSInstance@{\tt OSInstance}|(} class is the in-memory representation of an optimization instance and is a key
class for users of the OS project. This class has an API defined by a collection of {\tt get()} methods for
extracting various components (such as bounds and coefficients) from a problem instance, a collection of
{\tt set()} methods for modifying or generating an optimization instance, and a collection of {\tt calculate()}
methods for function, gradient, and Hessian evaluations.  See Section~\ref{section:osinstanceAPI}.
We now describe how to create an {\tt OSInstance} object and the close relationship between the OSiL\index{OSiL} schema
and the {\tt OSInstance} class.

\subsubdivision{Creating an {\tt OSInstance} Object}

The OSCommonInterfaces component contains an {\tt OSiLReader}  class for reading an instance in an OSiL string and
creating an in-memory {\tt OSInstance} object.  Assume that {\tt sOSiL} is a string that will hold the instance in OSiL format. Creating an {\tt OSInstance} object is illustrated in Figure~\ref{figure:creatingosinstanceobject}.

\begin{figure}[ht]
\centering
   \small {\obeyspaces\let =\
\fbox{\tt\begin{tabular}{@{}l@{}}
OSiLReader *osilreader = NULL;\\[\Sb]
OSInstance *osinstance = NULL;\\[\Sb]
osilreader = new OSiLReader();\\[\Sb]
osinstance = osilreader->readOSiL( sOSiL);\\[\Sb]
\end{tabular} }} \bigskip
\caption{Creating an {\tt OSInstance} Object} \label{figure:creatingosinstanceobject}
\end{figure}

\subsubdivision{Mapping Rules}\label{section:mappingrules}

The {\tt OSInstance} class has two members, {\tt instanceHeader} and {\tt instanceData}.  
These correspond to the XML elements {\tt <instanceHeader>} and {\tt <instanceData>}.
They are of type {\tt InstanceHeader} and {\tt InstanceData}, respectively, which in turn  
correspond to the OSiL schema's complexTypes {\tt InstanceHeader} and {\tt InstanceData}, and 
in themselves are C++ classes.

    Moving down one level, Figure~\ref{figure:instancedata} shows that the {\tt InstanceData} class has in turn 
the members {\tt variables}, {\tt objectives}, {\tt constraints}, {\tt linearConstraintCoefficients}, 
{\tt quadraticCoefficients}, and {\tt nonlinearExpressions}, corresponding to the respective elements 
in the OSiL file that have the same name. Each of these are instances of associated classes which correspond
to complexTypes in the OSiL schema.


\begin{figure}[ht]
\centering
   \small {\obeyspaces\let =\
\fbox{\tt\begin{tabular}{@{}l@{}}
class OSInstance\{\\[\Sb]
public:\\[\Sb]
     OSInstance(); \\[\Sb]
     InstanceHeader *instanceHeader;\\[\Sb]
     InstanceData *instanceData;    \\[\Sb]
\}; //class OSInstance\\[\Sb]
\end{tabular} }} \bigskip
\caption{The {\tt OSInstance} class} \label{figure:osinstance}
\end{figure}


\begin{figure}[ht]
\centering
   \small {\obeyspaces\let =\
\fbox{\tt\begin{tabular}{@{}l@{}}
class InstanceData\{\\[\Sb]
public:\\[\Sb]
     InstanceData();\\[\Sb]
     Variables *variables;\\[\Sb]
     Objectives *objectives;\\[\Sb]
     Constraints *constraints;\\[\Sb]
     LinearConstraintCoefficients *linearConstraintCoefficients;\\[\Sb]
     QuadraticCoefficients *quadraticCoefficients;\\[\Sb]
     NonlinearExpressions *nonlinearExpressions;\\[\Sb]
\}; // class InstanceData
\end{tabular} }} \bigskip
\caption{The {\tt InstanceData} class} \label{figure:instancedata}
\end{figure}


\begin{figure}[hb]
%\includegraphics[scale=0.8]{../../figures/paradigm1.eps}
\centering
   \scriptsize {\obeyspaces\let =\
\fbox{\tt\begin{tabular}{@{}l@{}}
\textsf{\textbf{Schema complexType  \hspace{3.64in}  In-memory class}}\\[\Sa]
<xs:complexType name="Variables">  <-------------------------------------------->  class Variables\{\\[\Sb]
                                                                                   public:\\[\Sb]
  <xs:sequence>                                                                      Variables();\\[\Sb]
    <xs:element name="var" type="Variable" maxOccurs="unbounded"/>  <----------->    Variable *var;\\[\Sb]
  </xs:sequence>                                                                   \\[\Sb]
  <xs:attribute name="numberOfVariables" type="xs:nonnegativeInteger"                 \\[\Sb]
                use="required"/>  <--------------------------------------------->    int numberOfVariables;\\[\Sb]
</xs:complexType>                                                                  \}; // class Variables\\[\Sb]
 \\[\Sb]
 \\[\Sb]
<xs:complexType name="Variable">  <--------------------------------------------->  class Variable\{\\[\Sb]
                                                                                   public:\\[\Sb]
                                                                                     Variable();\\[\Sb]
  <xs:attribute name="name" type="xs:string" use="optional"/>  <---------------->    string name;\\[\Sb]
  <xs:attribute name="type" use="optional" default="C">  <---------------------->    char type;\\[\Sb]
    <xs:simpleType>\\[\Sb]
      <xs:restriction base="xs:string">\\[\Sb]
        <xs:enumeration value="C"/>\\[\Sb]
        <xs:enumeration value="B"/>\\[\Sb]
        <xs:enumeration value="I"/>\\[\Sb]
        <xs:enumeration value="S"/>\\[\Sb]
        <xs:enumeration value="D"/>\\[\Sb]
        <xs:enumeration value="J"/>\\[\Sb]
      </xs:restriction>\\[\Sb]
    </xs:simpleType>\\[\Sb]
  </xs:attribute>\\[\Sb]
  <xs:attribute name="lb" type="xs:double" use="optional" default="0"/>  <------>    double lb;\\[\Sb]
  <xs:attribute name="ub" type="xs:double" use="optional" default="INF"/>  <---->    double ub;\\[\Sb]
</xs:complexType>                                                                  \}; // class Variable\\[\Sb]
 \\[\Sb]
 \\[\Sb]
\textsf{\textbf{OSiL elements          \hspace{1.83in}  In-memory objects}}\\[\Sa]
<variables numberOfVariables="2">                   OSInstance *osinstance;\\[\Sb]
   <var lb="0" name="x0" type="C"/>                 osinstance->instanceData->variables->numberOfVariables=2;\\[\Sb]
   <var lb="0" name="x1" type="C"/>                 osinstance->instanceData->variables->var=new Variable*[2];\\[\Sb]
</variables>                                        osinstance->instanceData->variables->var[0]->lb=0;\\[\Sb]
                                                    osinstance->instanceData->variables->var[0]->name="x0";\\[\Sb]
                                                    osinstance->instanceData->variables->var[0]->type={\rm 'C'};\\[\Sb]
                                                    osinstance->instanceData->variables->var[1]->lb=0;\\[\Sb]
                                                    osinstance->instanceData->variables->var[1]->name="x1";\\[\Sb]
                                                    osinstance->instanceData->variables->var[1]->type={\rm 'C'};
\end{tabular} }} \medskip\\[\Sb]
\caption{The {\tt <variables>} element as an {\tt OSInstance} object} \label{figure:osinstancevariables}
\end{figure}


Figure~\ref{figure:osinstancevariables} uses the {\tt Variables} class to provide a closer look 
at the correspondence between schema and class. On the right, the {\tt Variables} class contains 
the data member {\tt numberOfVariables} and a pointer to the object {\tt var} of class {\tt Variable}. 
The {\tt Variable} class has  data members {\tt lb} (double), {\tt ub} (double), {\tt name} (string), 
and {\tt type} (char). On the left the corresponding XML {\tt complexType}s are shown, with arrows indicating 
the correspondences. The following rules describe the mapping between the OSiL schema and the 
{\tt OSInstance} class. 
(In order to facilitate the mapping, we insist in the schema construction that every {\tt complexType} be named, 
even though this is not strictly necessary in XML.)
%
\begin{itemize}

\item  Each {\tt complexType} in an OSiL schema corresponds to a class in {\tt OSInstance}.
Thus the OSiL schema's {\tt complexType} {\tt Variables} corresponds to {\tt OSInstance}'s class {\tt Variables}.
Elements in an actual XML file then correspond to objects in {\tt OSInstance};
for example, the {\tt <variables>} element that is of type {\tt Variables} in an OSiL file
corresponds to a {\tt variables} object in {\tt OSInstance}. 
%(Since {\tt <variables>} is an element of the
%element {\tt <instanceData>} in OSiL, the {\tt variables} object is contained in class {\tt instanceData}.)

\item An attribute or element used in the definition of a {\tt complexType} is a member of the 
corresponding {\tt OSInstance} class, and the type of the attribute or element matches the type 
of the member.  In Figure~\ref{figure:osinstancevariables}, for example, {\tt lb} is an attribute 
of the OSiL {\tt complexType} named {\tt Variable}, and {\tt lb} is a member of the {\tt OSInstance} 
class {\tt Variable}; both have type {\tt double}.  Similarly, {\tt <var>} is an element in the definition 
of the OSiL {\tt complexType} named {\tt Variables}, and {\tt var} is a member of the {\tt OSInstance} 
class {\tt Variables}; the {\tt <var>} element has type {\tt Variable} and the {\tt var} member is a 
{\tt Variable} object.

\item A schema sequence corresponds to an array. For example, in Figure~\ref{figure:osinstancevariables} the complexType {\tt Variables} has a sequence of {\tt <var>} elements that are of type {\tt Variable}, and the corresponding {\tt Variables} class has a member that is an array of type {\tt Variable}.

\item
XML allows a wide range of data subtypes, which do not always have counterparts in the {\tt OSInstance} object.
For instance, the attribute {\tt type} in the {\tt <var>} element forms an enumeration, while the corresponding
member of the {\tt Variable} class is declared as {\tt char}. 

\item
XML allows default values for optional attributes; these default values can be set inside of the constructor of the corresponding data member. 
\end{itemize}
%
General nonlinear terms are stored in the data structure as {\tt OSExpressionTree} objects, which are the subject of the next section.

     The {\tt OSInstance} class has a collection of {\tt get()}, {\tt set()}, and {\tt calculate()} methods 
that act as an API for the optimization instance and are described in Section~\ref{section:osinstanceAPI}.




\subsubdivision{The OSExpressionTree OSnLNode Classes}\label{section:osexpressiontreeclass}

The {\tt OSExpressionTree}\index{OSExpressionTree@{\tt OSExpressionTree}} class provides the in-memory representation
of the nonlinear terms.  Our design goal is  to allow for efficient parsing of OSiL\index{OSiL} instances,
while providing an API that meets the needs of diverse solvers.  Conceptually, any nonlinear expression in the
objective or constraints is represented by a tree.  The expression tree for the nonlinear part of the
objective function~(\ref{eq:roobj}), for example, has the form illustrated in Figure~\ref{figure:expressiontree}.
The choice of a data structure to store such a tree --- along with the associated methods of an API --- is a key aspect
in the design of the {\tt OSInstance} class.

\begin{figure}[ht]
\centering
%\includegraphics[scale=0.38]{\figurepath/expressiontree.png}
\includegraphics[scale=0.38]{./figures/expressiontree.png}
\caption{Conceptual expression tree for the nonlinear part of the objective (\ref{eq:roobj}).}\label{figure:expressiontree}
\end{figure}


A base abstract class {\tt OSnLNode} is defined and  all of an OSiL file's
operator and operand elements used in defining a
nonlinear expression are extensions of the base element type {\tt OSnLNode}. There is an element type {\tt OSnLNodePlus}, 
for example, that extends {\tt OSnLNode}; then in an OSiL instance file, there are {\tt <plus>} elements that 
are of type {\tt OSnLNodePlus}.   Each {\tt OSExpressionTree} object contains a pointer to an {\tt OSnLNode} object 
that is the root of the corresponding expression tree.  To every element that extends the {\tt OSnLNode} type in an 
OSiL instance file, there corresponds a class that derives from the {\tt OSnLNode} class in an {\tt OSInstance} 
data structure.  Thus we can construct an expression tree of homogenous nodes, and methods that operate on the 
expression tree to calculate function values, derivatives, postfix notation, and the like do not require switches 
or complicated logic.


\begin{figure}[ht]
\centering
   \small {\obeyspaces\let =\
\fbox{\tt\begin{tabular}{@{}l@{}}
   double OSnLNodePlus::calculateFunction(double *x)\{\\[\Sb]
      m\_dFunctionValue = \\[\Sb]
         m\_mChildren[0]->calculateFunction(x) +\\[\Sb]
         m\_mChildren[1]->calculateFunction(x);\\[\Sb]
      return m\_dFunctionValue;\\[\Sb]
   \} //calculateFunction\\[\Sb]
\end{tabular} }} \medskip
\caption{The function calculation method for the {\tt plus} node class with polymorphism}
   \vspace{-8pt} \label{figure:calcfunction}
\end{figure}


The {\tt OSInstance} class has a variety of {\tt calculate()} methods, based on two pure virtual functions 
in the {\tt OSInstance} class.  The first of these, {\tt calculateFunction()}, takes an array of {\tt double} 
values corresponding to decision variables, and evaluates the expression tree for those values.  Every class
that extends {\tt OSnLNode} must implement this method.  As an example, the {\tt calculateFunction} method 
for the {\tt OSnLNodePlus} class is shown in Figure~\ref{figure:calcfunction}.  Because the OSiL instance 
file must be validated against its schema, and in the schema each {\tt <OSnLNodePlus>} element is specified 
to have exactly two child elements, this {\tt calculateFunction} method can assume that there are exactly 
two children of the node that it is operating on.  The use of polymorphism and recursion makes adding new 
operator elements easy; it is simply a matter of adding a new class and implementing the {\tt calculateFunction()} 
method for it.



Although in the OSnL schema, there are 200+ nonlinear operators, only the following {\tt  OSnLNode} classes are currently supported in our implementation.

\begin{itemize}
\item OSnLNodeVariable
\item OSnLNodeTimes
\item OSnLNodePlus
\item OSnLNodeSum
\item OSnLNodeMinus
\item OSnLNodeNegate
\item OSnLNodeDivide
\item OSnLNodePower
\item OSnLNodeProduct
\item OSnLNodeLn
\item OSnLNodeSqrt
\item OSnLNodeSquare
\item OSnLNodeSin
\item OSnLNodeCos
\item OSnLNodeExp
\item OSnLNodeIf
\item OSnLNodeAbs
\item OSnLNodeMax
\item OSnLNodeMin
\item OSnLNodeE
\item OSnLNodePI
\item OSnLNodeAllDiff
\end{itemize}
\index{OSInstance@{\tt OSInstance}|)}



\subsubdivision{The OSOption Class}\label{section:osoptionclass}

The {\tt OSOption}\index{OSOption@{\tt OSOption}|(} class is the in-memory representation of the options 
associated with a particular optimization task. It is another key
class for users of the OS project. This class has an API defined by a collection of {\tt get()} methods for
extracting various components (such as initial values for decision variables, solver options, job parameters, etc.), 
and a collection of {\tt set()} methods for modifying or generating an option instance. The relationship between
in-memory classes and objects on one hand and complexTypes and elements of the OSoL schema follow the same mapping rules
laid out in Section~\ref{section:mappingrules}.
\index{OSOption@{\tt OSOption}|)}

\subsubdivision{The OSResult Class}\label{section:osresultclass}

Similarly the {\tt OSResult}\index{OSResult@{\tt OSResult}|(} class is the in-memory representation of the 
results returned by the solver and other information associated with a particular optimization task. 
This class has an API defined by a collection of {\tt set()} methods that allow a solver to create a result instance
and a collection of {\tt get()} methods for
extracting various components (such as optimal values for decision variables, optimal objective function value, 
optimal dual variables, etc.). The relationship between
in-memory classes and objects on one hand and complexTypes and elements of the OSoL schema follow the same 
mapping rules laid out in Section~\ref{section:mappingrules}.
\index{OSResult@{\tt OSResult}|)}



\subdivision{OSModelInterfaces}\label{section:osmodelinterfaces}

This part of the OS library is designed to help integrate the OS standards with other standards and modeling systems.

\subsubdivision{Converting MPS Files}

The MPS standard\index{MPS format|(} is still a popular format for representing linear and integer programming problems.
In {\tt OSModelInterfaces,} there is a class {\tt OSmps2osil}\index{OSmps2osil@{\tt OSmps2osil}|(} that can be used
to convert files in MPS format into the OSiL\index{OSiL} standard. It is used as follows.

\begin{verbatim}
OSmps2osil *mps2osil = NULL;
DefaultSolver *solver  = NULL;
solver = new CoinSolver();
solver->sSolverName = "cbc";
mps2osil = new OSmps2osil(  mpsFileName);
mps2osil->createOSInstance() ;
solver->osinstance = mps2osil->osinstance;
solver->solve();
\end{verbatim}

The {\tt OSmps2osil} class constructor takes a string which should be the
file name of the instance in MPS format. The constructor then uses the
{\tt CoinUtils}\index{COIN-OR projects!CoinUtils@{\tt CoinUtils}} library to read and parse the MPS file.  The class method {\tt createOSInstance} then builds  an in-memory {\tt osinstance} object  that can be used by a solver.
\index{OSmps2osil@{\tt OSmps2osil}|)}\index{MPS format|)}

\subsubdivision{Converting AMPL nl Files}\label{section:nl2osil}

AMPL is a popular modeling language that saves  model instances in the AMPL nl format\index{AMPL nl format|(}.
The {\tt OSModelInterfaces} library provides a class, {\tt OSnl2osil}\index{OSnl2osil@{\tt OSnl2osil}},
for reading an nl file and creating a
corresponding in-memory  {\tt osinstance}\index{OSInstance@{\tt OSInstance}} object. It is used as follows.

\begin{verbatim}
OSnl2osil *nl2osil = NULL;
DefaultSolver *solver  = NULL;
solver = new LindoSolver();
nl2osil = new OSnl2osil( nlFileName);
nl2osil->createOSInstance() ;
solver->osinstance = nl2osil->osinstance;
solver->solve();
\end{verbatim}


The {\tt OSnl2osil} class works much like the {\tt OSmps2osil}\index{OSmps2osil@{\tt OSmps2osil}} class. The
{\tt OSnl2osil} class constructor takes a string which should be the file name of the instance in nl format. The constructor then uses the AMPL ASL library routines to read and parse the nl file. The class method {\tt createOSInstance} then builds  an in-memory {\tt osinstance} object  that can be used by a solver.

In Section~\ref{section:amplclient}  we describe the {\tt OSAmplClient}\index{OSAmplClient@{\tt OSAmplClient}}
executable that acts as a ``solver'' for AMPL. The {\tt OSAmplClient} uses the {\tt OSnl2osil} class to convert
the instance in nl format to OSiL\index{OSiL} format before calling a solver either locally or remotely.
\index{AMPL nl format|)}


\subdivision{OSParsers}\label{section:osparsers}

The OSParsers component of the OS library contains reentrant parsers that  read OSiL\index{OSiL|(},
OSoL\index{OSoL} and OSrL\index{OSrL} strings and build, respectively, in-memory 
OSInstance\index{OSInstance@{\tt OSInstance}}, OSOption\index{OSOption@{\tt OSOption}} and 
OSResult\index{OSResult@{\tt OSResult}}  objects.


The OSiL parser is invoked through an {\tt OSiLReader} object as illustrated below. Assume {\tt osil} is a string with the problem instance.
\begin{verbatim}
OSiLReader *osilreader = NULL;
OSInstance *osinstance = NULL;
osilreader = new OSiLReader();
osinstance = osilreader->readOSiL( osil);
\end{verbatim}
The {\tt  readOSiL} method  has a single argument which is a (pointer to a) string. 
The {\tt  readOSiL} method then calls an underlying method {\tt yygetOSInstance} that parses the OSiL string. 
The major components of the OSiL schema  recognized by the parser are
\begin{verbatim}
<instanceHeader>
<instanceData>
<variables>
<objectives>
<constraints>
<linearConstraintCoefficients>
<quadraticCoefficients>
<nonlinearExpressions>
\end{verbatim}
There are other components in the OSiL\index{OSiL|)} schema, but they are not yet implemented.
In most large-scale applications the {\tt <variables>,} {\tt <objectives>}, {\tt <constraints>}, and {\tt <linearConstraintCoefficients>}
will comprise the bulk of the instance memory.  Because of this, we have ``hard-coded'' the OSiL parser
to read these specific elements very efficiently.
The parsing of the {\tt <quadraticCoefficients>} and {\tt <nonlinearExpressions>} is done using code generated
by {\tt flex}\index{flex@{\tt flex}} and {\tt bison}\index{bison@{\tt bison}}. 
\ifdevelop
The file  
{\tt OSParseosil.l} is used by {\tt flex}\index{flex@{\tt flex}} to generate {\tt OSParseosil.cpp} and the file 
{\tt OSParseosil.y} is used by {\tt bison}\index{bison@{\tt bison}} to generate {\tt OSParseosil.tab.cpp}.
In {\tt OSParseosil.l} we use the {\tt reentrant} option and in {\tt OSParseosil.y} we use the
{\tt pure-parser} option to generate reentrant parsers. The {\tt OSParseosil.y} file  contains both our
``hard-coded'' parser and the grammar rules for the  {\tt <quadraticCoefficients>} and
{\tt <nonlinearExpressions>} sections.
We are currently using GNU {\tt bison} version 3.2 and {\tt flex} 2.5.33.

\fi
The typical OS user will have no need to edit either {\tt OSParseosil.l} or {\tt OSParseosil.y} 
and therefore will not have to worry about running either {\tt flex} or {\tt bison} to generate the parsers.
\ifdevelop 
The generated parser code from {\tt flex} and {\tt bison} is distributed with the project and works on all 
of the platforms listed in Table~\ref{table:testedplatforms}.  If the user does edit either {\tt OSParseosil.l} 
or {\tt OSParseosil.y} (or any of their constituent parts --- see comments in the opening sections of these files), then {\tt OSParseosil.cpp} and {\tt OSParseosil.tab.cpp} need to be regenerated with 
{\tt flex} and {\tt bison}. In order to make this work, the {\tt configure} step must be run with the
option {\tt --with-flex-bison} (see the notes on page~\pageref{itemize:unixBuildNotes}).
%
(This requires Unix or a unix-like environment (Cygwin, MinGW, MSYS, etc.) under Windows.)
The files OSParseosil.l.1 and OSParseosil.y.1 contain {\tt \#define} statements that allow debugging to be turned on. For even more detailed tracing of the {\tt bison} parser, the value of {\tt osildebug} can be set to a nonzero value in the calling program just before the call to the parser:
\begin{verbatim}
    extern int osildebug;
    osildebug = 1;
\end{verbatim}
{\bf Note that this code generates vast amounts of output.}
\medskip
\fi

The files {\tt OSParseosrl.l} and {\tt OSParseosrl.y} are used by {\tt flex} and {\tt bison} to  
generate the code {\tt OSParseosrl.cpp} and {\tt OSParseosrl.tab.cpp} for parsing strings in OSrL format. The comments made above about the OSiL parser apply to the OSrL parser. 
\ifdevelop 
The OSrL parser, like the OSiL parser, is invoked using an {\tt OSrL} reading object.
This is illustrated below ({\tt osrl} is a string in OSrL format).
\begin{verbatim}
OSrLReader *osrlreader = NULL;
osrlreader = new OSrLReader();
OSResult *osresult = NULL;
osresult = osrlreader->readOSrL( osrl);
\end{verbatim}

\fi
The OSoL parser follows the same layout and rules.
The files {\tt OSParseosol.l} and {\tt OSParseosol.y} are used by {\tt flex} and {\tt bison} to  generate the code 
{\tt OSParseosol.cpp} and {\tt OSParseosol.tab.cpp} for parsing strings in OSoL format. 
\ifdevelop
The OSoL parser
is invoked using an {\tt OSoL} reading object.
This is illustrated below ({\tt osol} is a string in OSoL format).
\begin{verbatim}
OSoLReader *osolreader = NULL;
osolreader = new OSoLReader();
OSOption *osoption = NULL;
osoption = osolreader->readOSoL( osol);
\end{verbatim}

There is also a lexer {\tt OSParseosss.l} for tokenizing the command line for the OSSolverService executable 
described in Section~\ref{section:ossolverservice}.
\fi

\ifbible
\subsubsection{Generic parser rules}\label{section:ParserRules}

In this section we describe some recommendations concerning elements, rules, names, 
content and structure of bison rule files for parsing documents used within the OS framework. 
The emphasis is on uniformity; computational efficiency is secondary. It is expected that the compiler will be able to optimize most if not all of the redundancies away.

\begin{enumerate}

\item    
For ease of development, trouble shooting and maintenance it is useful to have treatment of the 
different elements that is as uniform as possible. Computational efficiency is secondary; it is expected 
that the compiler will be able to deal with such issues. 

\item \label{enum:element}
	Every $<${\it element}$>$ is parsed using three production rules: {\it  elementStart\/}, {\it elementAttributes\/} 
and {\it elementContent\/}. 

\item
	{\it elementStart\/} matches the opening tag (``$<${\it element\/}''); its code section can be used 
to verify that the element was indeed expected at this spot, particularly in cases where this is hard 
to infer from the Schema. There are two instances when such checks need to be made:

\begin{enumerate}
\item	If the elements do not have to appear in any particular order, 
it is necessary to verify that there was no prior occurrence of this $<${\it element}$>$ within the scope 
of its parent.
\item If the element is contained in a repeat group, we must make sure that there are not 
more occurrences than specified in the {\it numberOf}$\ldots$ attribute of its parent.
\end{enumerate}
In addition the code can be used to initial the attribute list. The occurrence of attributes is tracked 
with indicators {\it xxxAttPresent}, which can be set to {\tt false} in this section. If an element 
has an optional {\it numberOf}$\ldots$ attribute, the variable holding the number of these items 
should be set to zero here to provide a default when the {\it numberOf}$\ldots$ attribute is missing.

\item	{\it elementAttributes\/} is included as a separate rule so that checks can be made after 
the entire list of attributes has been processed. It is necessary to check that all mandatory attributes 
have indeed been provided, and there may be other checks as needed. The production rule is

{\it elementAttributes: elementAttList}

where {\it elementAttList\/} is a standard list rule, which expands into

{\it elementAttList:} $\vert$ {\it elementAttList elementAtt\/}.

\item	{\it elementAtt\/} matches any of a list of attributes allowed under the current element, as in

{\it elementAtt: elementxxxAtt\/} $\vert$ {\it elementyyyAtt} $\vert$ {\it elementzzzAtt} $\ldots$

\item	Each {\it elementxxxAtt\/} is used to perform specific data checks, such as 
membership in an enumerative list, nonnegativity, etc., and to store the attribute value into 
the internal data structure. Moreover, attribute names, unlike element names, tend to be reused frequently. 
Thus {\it elementxxxAtt\/} may be a generic rule shared among many elements. 

\item	{\it elementxxxAtt\/} is also used to verify that the attribute has not been seen previously 
within the scope of the current element, to change its status from not present to present, 
and to assign the attribute value to a temporary variable.

\item	If an element allows only a single attribute, the above can be streamlined, 
the rule {\it elementxxxAtt\/} replacing the rule {\it elementAttributes\/}.

\item	If an element has no attributes, this rule is simply omitted.

\item	An element attribute may be used to record the number of child elements that are given in an array list. 
The parser records the number of child elements actually encountered and compares against the declared number. 
Any discrepancy is recorded. Such {\it numberOf}$\ldots$ attributes also allocate the storage space 
for the child elements and set the counter to~0. 

\item	{\it elementContent\/} can be empty or nonempty. This is normally expressed by the rule

{\it elementContent: elementEmpty} $\vert$ {\it elementBody}

In some rare cases modifications from this rule are needed in order to avoid reduce/reduce conflicts 
when an element has several optional children that must occur in a particular sequence, for instance

{\it variables: variableValues variableValuesString basisStatus otherVariableResultsArray}

where each of the child elements may be omitted --- or indeed all of them together.

\item	The code section in the {\it elementContent\/} rule can be used for consistency checks, 
storage of information into the data structure and, most importantly, to increment counters. 

\item	Empty element content is typically either ``$><$/{\it element\/}$>$'' or simply ``/$>$''. 
Code may be needed to detect empty element content and throw an appropriate error.

\item	{\it elementBody\/} expands into a variety of different patterns, as needed. There could be
\begin{enumerate}
\item	an array of $<${\it child\/}$>$ elements, which is distinguished from an element list 
by using the rule name {\it childArray\/}, which expands into 

{\it childArray:} $\vert$ {\it childArray childElement}
\item	several children in arbitrary order ({\it childList}) with a similar expansion
\item      a sequence of children, which are listed in order:

{\it childElement1  childElement2  childElement3 $\ldots$}

\item	other constructs as appropriate.
\end{enumerate}
\item	Each $<${\it childElement\/}$>$ would then be treated again as under point~\ref{enum:element}.

\end{enumerate}


\subsubsection{Parser development}\label{section:ParserDevelopment}

Since trunk revision 4818 (September 2014) the parser files have been maintained 
in several pieces to facilitate the development of shared data elements 
(mostly contained in the OSgL schema). 
The {\tt make} step assembles these pieces into the flex and bison files before 
calling the {\tt flex} \index{flex{\tt flex}} and {\tt bison}  \index{bison{\tt bison}} processors. 
To maintain the parsers it is therefore useful to adhere to a common six-step procedure:

\begin{enumerate}
\item Update the schema files.
\item Update the corresponding header files to reflect changes to the schemata.
\item Write or modify constructor and destructor methods for the changed C++ classes.
\item Update the parser files. (It is best to make sure the XML elements are recognized
        properly before adding any code.)
\item Implement {\tt get()} and {\tt set()} methods for the modified classes.
\item Add code to the parsers that properly populates the new classes.
\end{enumerate} 
         

\fi

\subdivision{OSSolverInterfaces}\label{section:ossolverinterfaces}


The {\tt OSSolverInterfaces} library is designed to facilitate linking the OS library with various solver APIs.
We first describe how to take a problem instance in OSiL\index{OSiL} format and connect to a solver 
that has a COIN-OR OSI interface. See the OSI project \url{www.projects.coin-or.org/Osi}.
We then describe hooking to the COIN-OR nonlinear code {\tt Ipopt}\index{COIN-OR projects!Ipopt@{\tt Ipopt}}. 
See \url{www.projects.coin-or.org/Ipopt}.
\ifknitro
Finally we describe hooking to two commercial solvers Knitro\index{Knitro} and LINDO\index{LINDO}.
\else
Finally we describe hooking to the commercial solver LINDO\index{LINDO}.
\fi
The OS library has been tested with the following solvers using the Osi Interface.

\begin{itemize}
\item Bonmin
\item Cbc
\item Clp
\item Couenne
\item Cplex
\item DyLP
\item Glpk
\item Ipopt
\item SYMPHONY
\item Vol
\end{itemize}

In the {\tt OSSolverInterfaces} library there is an abstract class
{\tt DefaultSolver} that has the following key members:

\begin{verbatim}
std::string osil;
std::string osol;
std::string osrl;
OSInstance *osinstance;
OSResult   *osresult;
OSOption   *osoption;
\end{verbatim}
and the pure virtual function
\begin{verbatim}
virtual void solve() = 0 ;
\end{verbatim}
In order to use a solver through the COIN-OR {\tt Osi}\index{COIN-OR projects, {\tt Osi}} 
interface it is
necessary to create an object in the {\tt CoinSolver} class which inherits
from the {\tt DefaultSolver} class and implements the appropriate
{\tt solve()} function.  We illustrate with the {\tt Clp} solver.

\begin{verbatim}
DefaultSolver *solver  = NULL;
solver = new CoinSolver();
solver->m_sSolverName = "clp";
\end{verbatim}

Assume that the data file containing the problem has been read into
the string {\tt osil} and the solver options are in the string {\tt osol}.
Then the {\tt Clp} solver is invoked as follows.

\begin{verbatim}
solver->osil = osil;
solver->osol = osol;
solver->solve();
\end{verbatim}

Finally, get the solution in {\tt OSrL} format as follows

\begin{verbatim}
cout << solver->osrl << endl;
\end{verbatim}

\ifknitro   %--------------------------------------------------------------------------
Even though LINDO and Knitro are commercial solvers and do not have a COIN-OR {\tt Osi} interface, these solvers are
used in exactly the same manner as a COIN-OR solver. For example, to invoke the LINDO solver we do the following.

\begin{verbatim}
solver = new LindoSolver();
\end{verbatim}

Similarly for Knitro and Ipopt. In the case of  Knitro, the {\tt KnitroSolver} class inherits from both
{\tt DefaultSolver} class and the Knitro {\tt NlpProblemDef} class. See {\tt\UrlKnitroMan} for more information 
on the Knitro solver C++ implementation and the {\tt NlpProblemDef} class.  Similarly, for Ipopt 
\else

Some commercial solvers, e.g., LINDO\index{LINDO|(}, do not have a COIN-OR {\tt Osi} interface, 
but it is possible to write wrappers so that they can be used in exactly the same manner 
as a COIN-OR solver. For example, to invoke the
LINDO solver we do the following.

\begin{verbatim}
solver = new LindoSolver();
\end{verbatim}
\index{LINDO|)}

A similar call is used for {\tt Ipopt}\index{COIN-OR projects!Ipopt@{\tt Ipopt}}. In this case, 
\fi         %--------------------------------------------------------------------------
the {\tt IpoptSolver} class inherits from both the {\tt DefaultSolver} class and the Ipopt {\tt TNLP} class. See 

\medskip
\noindent{\small\tt\UrlIpoptDoc}
\medskip

\noindent for more information on the Ipopt solver C++ implementation and the {\tt TNLP} class.


In the examples above,  the problem instance was assumed to be read from a file into the string {\tt osil}
and then into the class member {\tt solver->osil.} However, everything can be done entirely in memory.
For example, it is possible to use the {\tt OSInstance}\index{OSInstance@{\tt OSInstance}} class to create
an in-memory problem representation and give this representation directly to a solver class that inherits
from {\tt DefaultSolver}. The class member to use is {\tt osinstance.} This is illustrated in the example
given in Section~\ref{section:exampleOSInstanceGeneration}.


\subdivision{OSUtils}

The OSUtils component of the OS library contains utility codes. For example, the {\tt FileUtil} class contains
useful methods for reading files into {\tt string} or {\tt char*} and writing files from {\tt string} and {\tt char*}.
The {\tt OSDataStructures} class holds other classes for things such as sparse vectors, sparse Jacobians, and sparse
Hessians. The {\tt MathUtil} class contains a method for converting between sparse matrices in row and column major form.%
\index{OSLibrary@{\tt OSLibrary}|)}



\throwpage

\division{The  OSInstance API}\label{section:osinstanceAPI}

The OSInstance API can be used to:

\begin{itemize}

\item  get information about model parameters, or convert the {\tt OSExpressionTree} into a prefix or postfix
representation through a collection  of {\tt get()} methods,

\item modify, or even create an instance from scratch, using a number of {\tt set()} methods,

\item provide information to solvers that require function evaluations, Jacobian and Hessian sparsity patters,  
function gradient evaluations, and Hessian evaluations.

\end{itemize}



\subdivision{Get Methods}

The {\tt get()} methods are used by other classes to access data in an existing {\tt OSInstance} object or get 
an expression tree representation of an instance in postfix or prefix format.   Assume {\tt osinstance} is an 
object in the {\tt OSInstance} class created as illustrated in Figure~\ref{figure:creatingosinstanceobject}. 
Then, for example,
\begin{verbatim}
osinstance->getVariableNumber();
\end{verbatim}
will return an integer which is the number of variables in the problem,
\begin{verbatim}
osinstance->getVariableTypes();
\end{verbatim}
will return a {\tt char} pointer to the variable types ({\tt C} for continuous, {\tt B} for binary, 
and {\tt I} for general integer),
\begin{verbatim}
getVariableLowerBounds();
\end{verbatim}
will  return a {\tt double} pointer to the lower bound on each variable. There are similar {\tt get()} methods 
for the constraints. There are numerous {\tt get()} methods for the data in the {\tt <linearConstraintCoefficients>} 
 element, the {\tt <quadraticCoefficients>} element, and the {\tt <nonlinearExpressions>} element.

When an {\tt osinstance} object is created, it is stored as an expression tree in an {\tt OSExpressionTree} object. 
However, some solver APIs (e.g., LINDO) may take the data in a different format such as postfix and prefix. 
There are methods to return the data in either postfix or prefix format.

First define a {\tt vector} of pointers to {\tt OSnLNode} objects.
\begin{verbatim}
std::vector<OSnLNode*> postfixVec;
\end{verbatim}
then get the expression tree for the objective function (index = -1) as a postfix vector of nodes.
\begin{verbatim}
postfixVec = osinstance->getNonlinearExpressionTreeInPostfix( -1);
\end{verbatim}
If, for example, the {\tt osinstance} object was the in-memory representation of   the instance illustrated 
in  Section~\ref{section:rosenbrockXML} and Figure~\ref{figure:expressiontree} then the code
\begin{verbatim}
for (i = 0 ; i < n; i++){
    cout << postfixVec[i]->snodeName << endl;
}
\end{verbatim}
will produce
\begin{verbatim}
number
variable
minus
number
power
number
variable
variable
number
power
minus
number
power
times
plus
\end{verbatim}

This postfix traversal of the expression tree in Figure~\ref{figure:expressiontree} lists all the nodes
by recursively processing all subtrees, followed by the root node.
The method {\tt processNonlinearExpressions()} in the {\tt LindoSolver} class in the {\tt OSSolverInterfaces} 
library component illustrates the use of a postfix vector of {\tt OSnLNode} objects to build a Lindo model instance.


\subdivision{Set Methods}

The {\tt set()} methods can be used to build an in-memory {\tt OSInstance}
 object. A code example of how to do this is in Section~\ref{section:exampleOSInstanceGeneration}.

\subdivision{Calculate Methods}

The {\tt calculate()} methods are described in Section~\ref{section:ad}.


\subdivision{Modifying an   {\tt OSInstance} Object}\label{section:osinstanceMod}

The OSInstance API is designed to be used to either build an in-memory {\tt OSInstance} object 
or provide information about the in-memory object (e.g., the number of variables).   
This interface is not designed for problem modification.  We plan on later providing an {\tt OSModification} 
object for this task. However, by directly accessing an {\tt OSInstance} object it is possible 
to modify parameters in the following classes:

\begin{itemize}
\item {\tt Variables}

\item {\tt Objectives}

\item {\tt Constraints}

\item {\tt LinearConstraintCoefficients}
\end{itemize}

For example, to modify the first nonzero objective function coefficient of the first objective  function to 10.7 the user would write,

\begin{verbatim}
osinstance->instanceData->objectives->obj[0]->coef[0]->value = 10.7;
\end{verbatim}
If the user wanted to modify the actual number of nonzero coefficients as declared by 
\begin{verbatim}
osinstance->instanceData->objectives->obj[0]->numberOfObjCoef;
\end{verbatim}
then the only safe course of action would be to delete the current {\tt OSInstance} object 
and build a new one  with the modified coefficients. It is strongly recommend that no changes 
are made involving allocated memory -- i.e., any kind of {\tt numberOf***}.  
Modifying an objective function coefficient is illustrated in the OSModDemo example. 
See Section \ref{section:exampleOSModDemo}.

After modifying an {\tt OSInstance} object, it is necessary to set certain boolean variables 
to true in order for these changes to get reflected in the OS solver interfaces.

\begin{itemize}
\item {\tt Variables} -- if any changes are made to a parameter in this class set

\begin{verbatim}
osinstance->bVariablesModified = true;
\end{verbatim}

\item {\tt Objectives} -- if any changes are made to a parameter in this class set

\begin{verbatim}
osinstance->bObjectivesModified = true;
\end{verbatim}

\item {\tt Constraints} -- if any changes are made to a parameter in this class set

\begin{verbatim}
osinstance->bConstraintsModified = true;
\end{verbatim}

\item {\tt LinearConstraintCoefficients} -- if any changes are made to a parameter in this class set

\begin{verbatim}
osinstance->bAMatrixModified = true;
\end{verbatim}
\end{itemize}

At this point, if the user desires to modify an {\tt OSInstance} object that contains nonlinear terms, 
the only safe strategy is to delete the object and build a new object that contains the modifications. 



\subdivision{Printing a Model for Debugging}\label{section:printModel}

The OSiL representation for the test problem {\tt rosenbrockmod.osil} is given in 
Appendix~\ref{section:rosenbrockXML}.  Many users will not find the OSiL representation 
useful for model debugging purposes.  For users who wish to see a model in a standard infix 
representation we provide a method {\tt printModel()}.  Assume that we have an {\tt osinstance} 
object in the {\tt OSInstance} class that represents the model of interest.  The call
\begin{verbatim}
osinstance->printModel( -1)
\end{verbatim}
will result in printing the (first) objective function indexed by -1.  In order to print 
constraint~$k$ use
\begin{verbatim}
osinstance->printModel( k)
\end{verbatim}
In order to print the entire model use
\begin{verbatim}
osinstance->printModel( )
\end{verbatim}

 
Below we give the result of {\tt osintance->printModel( )} for the problem {\tt rosenbrockmod.osil}.
\begin{verbatim}
Objectives:
min 9*x_1 + (((1 - x_0) ^ 2) + (100*((x_1 - (x_0 ^ 2)) ^ 2)))

Constraints:
(((((10.5*x_0)*x_0) + ((11.7*x_1)*x_1)) + ((3*x_0)*x_1)) + x_0) <= 25  
10 <= ((ln( (x_0*x_1)) + (7.5*x_0)) + (5.25*x_1))

Variables:
x_0 Type = C  Lower Bound =  0  Upper Bound =  1.7976931348623157e308
x_1 Type = C  Lower Bound =  0  Upper Bound =  1.7976931348623157e308
\end{verbatim}
 


\throwpage

\division{The OS Algorithmic Differentiation Implementation}\label{section:ad}

The OS library provides a set of {\tt calculate} methods for calculating  function values, gradients, and Hessians.
The {\tt calculate} methods are part of the {\tt OSInstance} class and are designed to work with solver APIs.
For instance, {\tt Ipopt} requires derivatives but does not provide its own differentiation routines, 
expecting the user to make them available through callbacks.


\subdivision{Algorithmic Differentiation:  Brief Review}\label{section:adtheory}

First and second derivative calculations are made using 
{\it algorithmic differentiation}\index{Algorithmic differentiation|(}.
Here we provide a brief review of this topic.  An excellent reference on algorithmic differentiation
is Griewank\index{Griewank, A.@{\it Griewank, A.}}~\cite{griewank2000}.  The OS package uses the COIN-OR project 
CppAD\index{COIN-OR projects!CppAD@{\tt CppAD}|(} ({\tt\UrlCppad}), which  is also an excellent resource with extensive  
documentation and information about algorithmic differentiation.
See the documentation written by  Brad Bell\index{Bell, Bradley M.@{\it Bell, Bradley M.}}~\cite{bell2007}.    
The development here is from the CppAD documentation.  
Consider the function $f:X \rightarrow Y$ from $ \mathbb{R}^{n}$ to $ \mathbb{R}^{m}$.
(That is, $Y = f(X).$) Assume that $f$ is twice continuously differentiable, so that in particular the second order 
partials
\begin{eqnarray}
\DD{f_{k}}{x_{i}}{x_{j}}\ \ \  \mbox{and}\ \ \     \DD{f_{k}}{x_{j}}{x_{i}} \label{eq:mixedPartials}
\end{eqnarray}
exist and are equal to each other for all $k=1,\ldots,m$ and $i,j=1,\ldots,n$. The task is to compute the derivatives 
of~$f$.
 
First express the input vector as a function of~$t$ by
\begin{eqnarray}
X(t) = x^{(0)} +  x^{(1)} t +  x^{(2)} t^{2}
\end{eqnarray}
where $ x^{(0)},$ $x^{(1)},$ and $x^{(2)}$ are vectors in $ \mathbb{R}^{n}$  and $t$ is a scalar.  By judiciously choosing $x^{(0)}, x^{(1)},$ and $x^{(2)}$ we will be able to derive many different expressions of interest.  Note first that
\begin{eqnarray*}
X(0) &=& x^{(0)}, \\
X^{\prime}(0) &=& x^{(1)}, \\
X^{\prime \prime }(0) &=& 2 x^{(2)}.
\end{eqnarray*}
In general,  $x^{(k)}$ corresponds to the $k^{\rm th}$ order Taylor coefficient, i.e.,
\begin{eqnarray}
x^{(k)} = \frac{1}{k!}X^{(k)}(0), \quad k = 0, 1, 2.  \label{eq:xTaylorCoeff}
\end{eqnarray}
Then $Y(t) = f(X(t))$ is a function from $ \mathbb{R}^{1}$ to $ \mathbb{R}^{m}$ and is expressed in terms of its Taylor series expansion as
\begin{eqnarray}
Y(t)  = y^{(0)} +  y^{(1)} t +  y^{(2)} t^{2} + o(t^{3}),
\end{eqnarray}
where
\begin{eqnarray}
y^{(k)} = \frac{1}{k!} Y^{(k)}(0), \quad k = 0, 1, 2.  \label{eq:yTaylorCoeff}
\end{eqnarray}



The following are shown in Bell~\cite{bell2007}.
\begin{eqnarray}
y^{(0)} = f(x^{(0)}). \label{eq:forward0Result}
\end{eqnarray}
Let $e^{(i)}$ denote the $i^{\rm th}$ unit vector.  If $x^{(1)} = e^{(i)}$ then $y^{(1)}$ is equal to the
$i^{\rm th}$ column of the Jacobian matrix of $f(x)$ evaluated at $x^{(0)}.$ That is
\begin{eqnarray}
y^{(1)} = \D{f}{x_{i}}(x^{(0)}).  \label{eq:forward1Result}
\end{eqnarray}

In addition, if $x^{(1)} = e^{(i)}$ and $x^{(2)} = 0$ then for function $f_{k}(x),$ (the $k^{\rm th}$ 
component of~$f$)
\begin{eqnarray}
y^{(2)}_{k} =  \frac{1}{2} \DD{f_{k}(x^{(0)})}{x_{i}}{x_{i}}.  \label{eq:forward2Resulta}
\end{eqnarray}

In order to evaluate the mixed partial derivatives, one can instead set $x^{(1)} = e^{(i)} + e^{(j)}$ and $x^{(2)} = 0.$    This gives for function $f_{k}(x),$
\begin{eqnarray}
y^{(2)}_{k} =  \frac{1}{2} \left( \DD{f_{k}(x^{(0)})}{x_{i}}{x_{i}}  +   \DD{f_{k}(x^{(0)})}{x_{i}}{x_{j}} 
+  \DD{f_{k}(x^{(0)})}{x_{j}}{x_{i}} +  \DD{f_{k}(x^{(0)})}{x_{j}}{x_{j}}  \right), \label{eq:forward2Resultb}
\end{eqnarray}
or, expressed in terms of the mixed partials,
\begin{eqnarray}
  \DD{f_{k}(x^{(0)})}{x_{i}}{x_{j}}  = y_{k}^{(2)}  -  \frac{1}{2} \left( \DD{f_{k}(x^{(0)})}{x_{i}}{x_{i}}  
+  \DD{f_{k}(x^{(0)})}{x_{j}}{x_{j}}  \right). \label{eq:forward2Resultc}
\end{eqnarray}
\index{Algorithmic differentiation|)}\index{COIN-OR projects!CppAD@{\tt CppAD}|)}




\subdivision{Using OSInstance Methods: Low Level Calls}\label{section:lowlevelADcalls}

  The code snippets used in this section  are from the example code {\tt algorithmicDiffTest.cpp} in the
{\tt algorithmicDiffTest} folder in the {\tt examples} folder.  The  code is based on the following example.

\begin{alignat}{2}
& \mbox{Minimize} & \quad  x_{0}^{2} + 9x_{1} \label{eq:adobj}\\
& \mbox{s.t.} & 33 - 105 + 1.37 x_{1} + 2x_{3} + 5 x_{1} &\le 10  \label{eq:adeq0}\\
& & \ln(x_{0} x_{3}) + 7x_{2} &\ge 10 \label{eq:adeq1} \\
& & x_{0}, x_{1}, x_{2}, x_{3} &\ge 0 \label{eq:adeq2}
\end{alignat}

The OSiL representation of the instance  (\ref{eq:adobj})--(\ref{eq:adeq2}) is given in Appendix~\ref{section:adexample}.
This example is designed to illustrate several features of OSiL. Note that in constraint  (\ref{eq:adeq0}) the
constant~33 appears in the {\tt <con>} element corresponding to this constraint
and the constant~105 appears as a {\tt <number>} OSnL node in the {\tt <nonlinearExpressions>} section.
This distinction is important, as it will lead to different treatment by the code as documented below.
%There are no nonlinear terms in the instance that involve variable $x_{1}.$
Variables $x_{1}$ and $x_{2}$  do not appear in any nonlinear terms.
The terms $5x_{1}$ in  (\ref{eq:adeq0}) and $7 x_{2}$ in (\ref{eq:adeq1}) are expressed in the
{\tt <objectives>} and {\tt <linearConstraintCoefficients>} sections, respectively, and will again
receive special treatment by the code. However, the term $1.37x_1$ in (\ref{eq:adeq0}),
along with the term $2x_3$, is expressed in the {\tt <nonlinearExpressions>} section,
%Variables $x_{1}$ and $x_{2}$  do not appear in any nonlinear terms.
%However, in the OSInstance API, variable $x_{1}$  appears in the   {\tt <nonlinearExpressions>} section;
hence $x_1$  is treated as a nonlinear variable for purposes of algorithmic differentiation.
Variable $x_{2}$ never appears in the  {\tt <nonlinearExpressions>} section and is therefore treated as a linear variable and not used  in any algorithmic differentiation calculations.
Variables that do not appear in the {\tt <nonlinearExpressions>} are never part of the algorithmic differentiation calculations.

Ignoring the nonnegativity constraints, instance (\ref{eq:adobj})--(\ref{eq:adeq2})  defines a mapping  
from $ \mathbb{R}^{4}$ to~$\mathbb{R}^{3}$:




\begin{eqnarray}
    \left[
        \begin{array}{r}
            x_0^2+9x_1 \\
            33 - 105 + 1.37x_1 + 2x_3 + 5x_1 \\
            \ln (x_0x_3) + 7x_2
        \end{array}
    \right]
&=&
    \left[
        \begin{array}{r}
            9x_1 \\
            33 + 5x_1 \\
            7x_2
        \end{array}
    \right]
+
    \left[
        \begin{array}{r}
            x_0^2 \\
            - 105 + 1.37x_1 + 2x_3  \\
            \ln (x_0x_3)
        \end{array}
    \right]
  \nonumber  \\
  &=&  \left[
        \begin{array}{r}
            9x_1 \\
            33 + 5x_1 \\
            7x_2
        \end{array}
    \right]
+
    \left[
        \begin{array}{r}
            f_1(x) \\
            f_2(x) \\
            f_3(x)
        \end{array}  \label{eq:definef1}
    \right],
\end{eqnarray}

\begin{eqnarray}
\hbox{\rm where}\ f(x) :=
%
    \left[
        \begin{array}{r}
            f_1(x) \\
            f_2(x) \\
            f_3(x)
        \end{array}   \label{eq:definef}
    \right].
\end{eqnarray}


The OSiL representation for the instance  in  (\ref{eq:adobj})--(\ref{eq:adeq2})  is read into an in-memory
OSInstance object as follows (we assume that {\tt osil} is a {\tt string} containing the OSiL instance)
\begin{verbatim}
osilreader = new OSiLReader();
osinstance = osilreader->readOSiL( &osil);
\end{verbatim}
There is a method in the {\tt OSInstance} class, {\tt initForAlgDiff()} that is used to initialize the nonlinear data structures.  A call to this method
\begin{verbatim}
osinstance->initForAlgDiff( );
\end{verbatim}
will generate a map of the indices of the nonlinear variables. This is critical because the algorithmic 
differentiation only operates on variables that appear in the {\tt <nonlinearExpressions>} section.  
An example of this map follows.
\begin{verbatim}
std::map<int, int> varIndexMap;
std::map<int, int>::iterator posVarIndexMap;
varIndexMap = osinstance->getAllNonlinearVariablesIndexMap( );
for(posVarIndexMap = varIndexMap.begin(); posVarIndexMap
    != varIndexMap.end(); ++posVarIndexMap){
    std::cout <<  "Variable Index = "   << posVarIndexMap->first  << std::endl ;
}
\end{verbatim}
The variable indices listed are 0, 1, and~3. Variable~2 does not appear in the {\tt <nonlinearExpressions>} section and
is not included in {\tt varIndexMap}. That is, the function $f$ in~(\ref{eq:definef}) will be considered as a map from 
$\mathbb{R}^{3}$ to~$\mathbb{R}^{3}$.

Once the nonlinear structures are initialized it is possible to take derivatives using algorithmic differentiation.
Algorithmic differentiation is done using either a forward or reverse sweep through an expression tree (or operation
sequence) representation of~$f$.  The two key {\tt public} algorithmic differentiation  methods in the {\tt OSInstance}%
\index{OSInstance@{\tt OSInstance}} class are {\tt forwardAD} and {\tt reverseAD}.
These are actually  generic ``wrappers'' around the corresponding CppAD methods with the same signature.
This keeps the OS API  public methods independent of any underlying algorithmic differentiation package.

The {\tt forwardAD} signature is
\begin{verbatim}
std::vector<double> forwardAD(int k, std::vector<double> vdX);
\end{verbatim}
where {\tt k} is the highest order Taylor coefficient of $f$ to be returned,  $\tt vdX$ is a vector of doubles in 
$ \mathbb{R}^{n},$ and the function return is a vector of doubles in~$\mathbb{R}^{m}.$  Thus, {\tt k} corresponds 
to the $k$ in Equations  (\ref{eq:xTaylorCoeff}) and (\ref{eq:yTaylorCoeff}),  where {\tt vdX} corresponds to the $x^{(k)}$ in Equation (\ref{eq:xTaylorCoeff}), and the $y^{(k)}$ in Equation (\ref{eq:yTaylorCoeff}) is the vector in range space returned by the call to {\tt  forwardAD}.    For example, by  Equation (\ref{eq:forward0Result}) the following call will evaluate each component function defined in (\ref{eq:definef}) corresponding only to the nonlinear part of (\ref{eq:definef1}) -- the part denoted by $f(x)$.
\begin{verbatim}
funVals = osinstance->forwardAD(0, x0);
\end{verbatim}
Since there are three components in the vector defined by  (\ref{eq:definef}), the return value  {\tt funVals} will have three components. For an input vector,
\begin{verbatim}
x0[0] = 1; // the value for variable x0 in function f
x0[1] = 5; // the value for variable x1 in function f
x0[2] = 5; // the value for variable x3 in function f
\end{verbatim}
the values returned by {\tt osinstance->forwardAD(0, x0)}  are 1, -63.15, and 1.6094, respectively.
The Jacobian of the example in (\ref{eq:definef}) is

\begin{eqnarray}
J =
\left[
\begin{array}{rrrr}
2x_{0} &9.00&0.00&0.00   \\
0.00&6.37&0.00&2.00 \\
1/x_{0}&0.00&7.00&1/x_{3}
\end{array}
\right] \label{eq:jac}
\end{eqnarray}
and the Jacobian $J_f$ of the nonlinear part is
%
\begin{equation}
    J_f = \left[
        \begin{array}{ccc}
            2x_0 & 0.00 & 0.00 \\
            0.00  & 1.37 & 2.00 \\
            1/x_0 & 0.00 & 1/x_3
        \end{array}
    \right].  \label{eq:jac2}
\end{equation}
When $x_{0} = 1,$ $x_{1} = 5,$ $x_{2} = 10,$ and $x_{3} = 5$ the Jacobian $J_f$ is
\begin{eqnarray}
    J_f = \left[
        \begin{array}{ccc}
            2.00 & 0.00 & 0.00 \\
            0.00 & 1.37 & 2.00 \\
            1.00 & 0.00 & 0.20
        \end{array}
    \right]. \label{eq:jac3}
\end{eqnarray}
A forward sweep with $k = 1$ will calculate the Jacobian column-wise.  See~(\ref{eq:forward1Result}).  
The following code will return column 3 of the Jacobian (\ref{eq:jac3}) which corresponds to the nonlinear variable~$x_{3}$.
\begin{verbatim}
x1[0] = 0;
x1[1] = 0;
x1[2] = 1;
osinstance->forwardAD(1, x1);
\end{verbatim}

Now calculate second derivatives.  To illustrate we use the results in (\ref{eq:forward2Resulta})-(\ref{eq:forward2Resultc}) and calculate
\begin{eqnarray*}
\DD{f_{k}(x^{(0)})}{x_{0}}{x_{3}} \quad k = 1, 2, 3.
\end{eqnarray*}
Variables $x_{0}$ and $x_{3}$ are the first and third nonlinear variables so by  (\ref{eq:forward2Resultb}) the $x^{(1)}$ should be the sum of the $e^{(1)}$ and $e^{(3)}$ unit vectors and used in the  first-order forward sweep calculation.
\begin{verbatim}
x1[0] = 1;
x1[1] = 0;
x1[2] = 1;
osinstance->forwardAD(1, x1);
\end{verbatim}
Next set $x^{(2)} = 0$ and do a second-order forward sweep.
\begin{verbatim}
std::vector<double> x2( n);
x2[0] = 0;
x2[1] = 0;
x2[2] = 0;
osinstance->forwardAD(2, x2);
\end{verbatim}
This call returns the vector of  values
\begin{eqnarray*}
y_{1}^{(2)}  = 1, \quad y_{2}^{(2)}  = 0, \quad y_{3}^{(2)} = -0.52.
\end{eqnarray*}
By inspection of (\ref{eq:definef1}) (or by appropriate calls to {\tt osinstance->forwardAD} --- not shown here),
$$
\begin{array}{rclcrcl}
\displaystyle{\DD{f_{1}(x^{(0)})}{x_{0}}{x_{0}}} &=&  2, & \qquad & 
\displaystyle{\DD{f_{1}(x^{(0)})}{x_{3}}{x_{3}}} &=&  0, \\ [12pt]
\displaystyle{\DD{f_{2}(x^{(0)})}{x_{0}}{x_{0}}} &=&  0, & \qquad & 
\displaystyle{\DD{f_{2}(x^{(0)})}{x_{3}}{x_{3}}} &=&  0, \\ [12pt]
\displaystyle{\DD{f_{3}(x^{(0)})}{x_{0}}{x_{0}}} &=& -1, & \qquad & 
\displaystyle{\DD{f_{3}(x^{(0)})}{x_{3}}{x_{3}}} &=& -0.04.
\end{array}
$$
Then by (\ref{eq:forward2Resultc}),
\begin{eqnarray*}
\DD{f_{1}(x^{(0)})}{x_{0}}{x_{3}} &=&  y_{1}^{(2)}  -  \frac{1}{2} \left( \DD{f_{1}(x^{(0)})}{x_{0}}{x_{0}}  +  \DD{f_{k}(x^{(0)})}{x_{3}}{x_{3}}  \right) = 1   -    \frac{1}{2}(2 +  0) = 0, \\
\DD{f_{2}(x^{(0)})}{x_{0}}{x_{3}} &=&  y_{2}^{(2)}  -  \frac{1}{2} \left( \DD{f_{2}(x^{(0)})}{x_{0}}{x_{0}}  +  \DD{f_{k}(x^{(0)})}{x_{3}}{x_{3}}  \right) = 0   -    \frac{1}{2}(0 +  0) = 0, \\
\DD{f_{3}(x^{(0)})}{x_{0}}{x_{3}} &=&  y_{3}^{(2)}  -  \frac{1}{2} \left( \DD{f_{3}(x^{(0)})}{x_{0}}{x_{0}}  +  \DD{f_{k}(x^{(0)})}{x_{3}}{x_{3}}  \right) = -0.52 -  \frac{1}{2}(-1 - 0.04) = 0.
\end{eqnarray*}
Making all of the first and second derivative calculations using forward sweeps is most effective when the number of rows exceeds the number of variables.


The {\tt reverseAD} signature is
\begin{verbatim}
std::vector<double> reverseAD(int k, std::vector<double> vdlambda);
\end{verbatim}
where {\tt vdlambda} is a vector of Lagrange multipliers.  This method returns a vector in the range space. If a reverse sweep of order $k$ is called, a forward sweep of all orders  through  $k -1$ must have been made prior to the call.

\subsubdivision{First Derivative Reverse Sweep Calculations}

In order to calculate first derivatives execute the following sequence of calls.
\begin{verbatim}
x0[0] = 1;
x0[1] = 5;
x0[2] = 5;
std::vector<double> vlambda(3);
vlambda[0] = 0;
vlambda[1] = 0;
vlambda[2] = 1;
osinstance->forwardAD(0, x0);
osinstance->reverseAD(1, vlambda);
\end{verbatim}
Since {\tt vlambda} only includes
the third function $f_3$, this sequence of calls will produce the third row of the
Jacobian $J_f$, i.e.,
$$
\D{f_{3}(x^{(0)})}{x_{0}}  = 1,  \quad \D{f_{3}(x^{(0)})}{x_{1}}  = 0, \quad  \D{f_{3}(x^{(0)})}{x_{3}}  = 0.2.
$$

\subsubdivision{Second Derivative Reverse Sweep Calculations}

In order to calculate second derivatives using {\tt reverseAD} forward sweeps of order 0 and 1 must have been 
completed.  The call to {\tt reverseAD(2, vlambda)} will return a vector of dimension $2n$ where~$n$ is the 
number of variables.  If the zero-order forward sweep is {\tt forwardAD(0,x0)} and the first-order forward 
sweep is {\tt forwardAD(1, x1)} where {\tt x1} $= e^{(i)},$ then the return vector 
{\tt z = reverseAD(2,  vlambda)} is
\begin{eqnarray}
z[2j - 2]  = \D{L (x^{(0)}, \lambda^{(0)})}{x_{j}}, \quad j = 1, \ldots, n
\end{eqnarray}
\begin{eqnarray}
z[2j - 1]  = \DD{L(x^{(0)}, \lambda^{(0)})}{x_{i}}{x_{j}}, \quad j = 1, \ldots, n
\end{eqnarray}
where
\begin{eqnarray}
L (x, \lambda) = \sum_{k = 1}^{m} \lambda_{k} f_{k}(x).
\end{eqnarray}



For example, the  following calls will calculate the third row (column) of the Hessian of the Lagrangian.
\begin{verbatim}
x0[0] = 1;
x0[1] = 5;
x0[2] = 5;
osinstance->forwardAD(0, x0);
x1[0] = 0;
x1[1] = 0;
x1[2] = 1;
osinstance->forwardAD(1, x1);
vlambda[0] = 1;
vlambda[1] = 2;
vlambda[2] = 1;
osinstance->reverseAD(2, vlambda);
\end{verbatim}
This returns
\begin{eqnarray*}
\D{L (x^{(0)}, \lambda^{(0)})}{x_{0}} = 3, \quad  
\D{L (x^{(0)}, \lambda^{(0)})}{x_{1}} = 2.74, \quad  
\D{L (x^{(0)}, \lambda^{(0)})}{x_{3}} = 4.2,
\end{eqnarray*}
\begin{eqnarray*}
\DD{L(x^{(0)}, \lambda^{(0)})}{x_{3}}{x_{0}} =0, \quad  
\DD{L(x^{(0)}, \lambda^{(0)})}{x_{3}}{x_{0}} = 0, \quad   
\DD{L(x^{(0)}, \lambda^{(0)})}{x_{3}}{x_{3}} =  -.04.
\end{eqnarray*}
The reason why
$$
\D{L (x^{(0)}, \lambda^{(0)})}{x_{1}} = 2 \times 1.37 = 2.74
$$
and not
$$
\D{L (x^{(0)}, \lambda^{(0)})}{x_{1}} = 1 \times  9 + 2 \times 6.37 = 9 + 12.74 = 21.74
$$
is that the terms $9x_1$ in the objective and $5x_1$ in the first constraint
are captured in the linear section of the OSiL input and therefore do not appear as nonlinear terms
in {\tt  <nonlinearExpressions>}. As noted before, {\tt forwardAD} and {\tt reverseAD} only operate on variables and terms
in either the {\tt <quadraticCoefficients>} or {\tt <nonlinearExpressions>} sections.

\subdivision{Using OSInstance Methods: High Level Calls}

The methods {\tt forwardAD} and {\tt reverseAD} are low-level calls and are not designed to work directly with solver APIs. The {\tt OSInstance} API has other methods that most users will want to invoke when linking with solver APIs.  We describe these now.


\subsubdivision{Sparsity Methods}

Many solvers such as {\tt Ipopt}\index{COIN-OR projects!Ipopt@{\tt Ipopt}} (\url{projects.coin-or.org/Ipopt}) 
\ifknitro or Knitro\index{Knitro} (\url{www.ziena.com}) \fi
require the sparsity pattern of the Jacobian of the constraint matrix and the Hessian of the Lagrangian function.
Note well that the constraint matrix of the example in Section~\ref{section:lowlevelADcalls}
constitutes only the last two rows of (\ref{eq:definef}) but does include the linear terms.
The following code illustrates how to get the sparsity pattern of the constraint Jacobian matrix

\begin{verbatim}
SparseJacobianMatrix *sparseJac;
sparseJac = osinstance->getJacobianSparsityPattern();
for(idx = 0; idx < sparseJac->startSize; idx++){
    std::cout << "number constant terms in constraint "   <<  idx << " is "
    << *(sparseJac->conVals + idx)  << std::endl;
    for(k = *(sparseJac->starts + idx); k < *(sparseJac->starts + idx + 1); k++){
        std::cout << "row idx = " << idx <<  "
        col idx = "<< *(sparseJac->indexes + k) << std::endl;
    }
}
\end{verbatim}

For the example problem this will produce

\begin{verbatim}
JACOBIAN SPARSITY PATTERN
number constant terms in constraint 0 is 0
row idx = 0  col idx = 1
row idx = 0  col idx = 3
number constant terms in constraint 1 is 1
row idx = 1  col idx = 2
row idx = 1  col idx = 0
row idx = 1  col idx = 3
\end{verbatim}

The   constant term in constraint 1 corresponds to the linear term $7x_2$,
which is added after the algorithmic differentiation has taken place.
However, the linear  term $5x_1$ in constraint 0 does not
contribute a nonzero in the Jacobian, as it is combined with the
term $1.37x_1$ that is treated as a nonlinear term and
therefore accounted for explicitly.
The {\tt SparseJacobianMatrix} object has a data member {\tt starts}
which is the index of the start of each constraint row.
The {\tt int} data member {\tt indexes}  gives  the variable index
of every potentially nonzero derivative. There is also a {\tt double} data member
{\tt values} that gives the value of the partial derivative of the corresponding
index at each iteration. Finally, there is an {\tt int} data member
{\tt conVals} that is the number of constant terms in each gradient.
A constant term is a partial derivative that cannot change at an iteration.
A variable is considered to have a constant derivative
if it appears in the {\tt <linearConstraintCoefficients>} section
but not in the {\tt <nonlinearExpressions>}.  For a row indexed by {\tt idx}
the variable indices are in the  {\tt indexes} array between the elements
{\tt sparseJac->starts + idx} and {\tt sparseJac->starts + idx + 1}.
The first  {\tt sparseJac->conVals + idx} variables listed are indices
of  variables with constant derivatives. In this example, when {\tt idx} is 1,
there is one  variable with a constant derivative and it is variable $x_{2}$.
(Actually variable $x_{1}$ has a constant derivative but the code does not check
to see if variables that appear in the {\tt <nonlinearExpressions>} section
have constant derivative.) The  variables with constant derivatives
never appear in the AD evaluation.

The following code illustrates how to get the sparsity pattern of the Hessian of the Lagrangian.
\begin{verbatim}
SparseHessianMatrix *sparseHessian;
sparseHessian = osinstance->getLagrangianHessianSparsityPattern( );
for(idx = 0; idx < sparseHessian->hessDimension; idx++){
    std::cout <<  "Row Index = " << *(sparseHessian->hessRowIdx + idx) ;
    std::cout <<  "  Column Index = " << *(sparseHessian->hessColIdx + idx);
}
\end{verbatim}
The {\tt SparseHessianMatrix} class has the {\tt int} data members {\tt hessRowIdx} and {\tt hessColIdx} 
for indexing  potential nonzero elements in the Hessian matrix. The {\tt double} data member {\tt hessValues} 
holds the value of the respective second derivative at each iteration.   
The data member {\tt hessDimension} is the number of nonzero elements in the Hessian.


\subsubdivision{Function Evaluation Methods}

There are several overloaded methods for calculating objective and constraint values.  The method
\begin{verbatim}
double *calculateAllConstraintFunctionValues(double* x, bool new_x)
\end{verbatim}
will return a {\tt double} pointer to an array of constraint function values evaluated at {\tt x.}  
If the value of {\tt x} has not changed since the last function call, then {\tt new\_x} should be set 
to {\tt false} and the most recent function values are returned.  When using this method, with this signature,  
all function values are calculated in {\tt double} using an {\tt OSExpressionTree} object.

A second signature for the {\tt calculateAllConstraintFunctionValues} is
\begin{verbatim}
double *calculateAllConstraintFunctionValues(double* x, double *objLambda,
    double *conLambda, bool new_x, int highestOrder)
\end{verbatim}
In this  signature, {\tt x} is a pointer to the current primal values, {\tt objLambda} is a vector of dual multipliers, 
{\tt conLambda} is a vector of dual multipliers on the constraints,  {\tt new\_x} is true if any components of {\tt x} 
have changed since the last evaluation, and {\tt highestOrder} is the highest order of derivative to be calculated 
at this iteration. The following code snippet illustrates defining a set of variable values for the example we are 
using and then the function call.
\begin{verbatim}
double* x = new double[4]; //primal variables
double* z = new double[2]; //Lagrange multipliers on constraints
double* w = new double[1]; //Lagrange multiplier on objective
x[ 0] = 1;    // primal variable 0
x[ 1] = 5;    // primal variable 1
x[ 2] = 10;   // primal variable 2
x[ 3] = 5;    // primal variable 3
z[ 0] = 2;    // Lagrange multiplier on constraint 0
z[ 1] = 1;    // Lagrange multiplier on constraint 1
w[ 0] = 1;    // Lagrange multiplier on the objective function
calculateAllConstraintFunctionValues(x, w, z,  true, 0);
\end{verbatim}
When making all high level calls for function, gradient, and Hessian evaluations we pass all the 
primal variables in the {\tt x} argument, not just the nonlinear variables. Underneath the call, 
the nonlinear variables are identified and used in AD function calls.

The use of the parameters  {\tt new\_x} and {\tt highestOrder}  is important and requires further explanation.    
The parameter  {\tt highestOrder}  is an integer variable that will take on the value 0, 1, or 2 (actually 
higher values if we want third derivatives etc.).  The value of this variable is the highest order derivative 
that is required of the current iterate. For example, if  a callback requires a function evaluation and 
{\tt highestOrder = 0} then only the function is evaluated at the current iterate.  However,  
if {\tt highestOrder = 2} then the function call
\begin{verbatim}
calculateAllConstraintFunctionValues(x, w, z, true, 2)
\end{verbatim}
will trigger  first and second derivative evaluations in addition to the function evaluations.

In the {\tt OSInstance} class code,  every time a forward ({\tt forwardAD}) or reverse sweep ({\tt reverseAD}) 
is executed a private  member, {\tt m\_iHighestOrderEvaluated}  is  set to the order of the sweep. For example, 
{\tt forwardAD(1, x)} will result in {\tt  m\_iHighestOrderEvaluated = 1}.  Just knowing the value  of 
 {\tt new\_x} alone is not sufficient. It is also necessary  to know {\tt highestOrder} and compare it with 
{\tt m\_iHighestOrderEvaluated.}  For example, if  {\tt new\_x}  is  false,  but {\tt m\_iHighestOrderEvaluated = 0},  
and   the callback requires a Hessian calculation, then it is necessary to calculate the first and second derivatives 
at the current iterate.

There are {\it  exactly two} conditions that  require a new function or derivative evaluation.   
A new evaluation is required if and only if

\begin{enumerate}
\item{}   The value of {\tt new\_x} is  true

\begin{center}
 --OR--
\end{center}


\item{} For the callback function the value of the input parameter {\tt highestOrder} is strictly greater 
than the current value  of    {\tt m\_iHhighestOrderEvaluated.}
\end{enumerate}

For an efficient implementation of AD it is important to be able to get the Lagrange multipliers and highest order 
derivative that is required from inside {\it any} callback -- not just the Hessian evaluation callback. 
For example, in {\tt Ipopt,} if  {\tt eval\_g}  or {\tt eval\_f} are called, and  for the current iterate, 
{\tt eval\_jac} and {\tt eval\_hess} are also going to be called, then  a more efficient AD implementation 
is possible if the Lagrange multipliers are available for {\tt eval\_g} and {\tt eval\_f}.

Currently, whenever {\tt new\_x = true} in the underlying AD implementation we do not retape 
(record into the CppAD data structure)  the function. This is because we currently throw an exception 
if there are any logical operators involved in the AD calculations. This may change in a future implementation.


There are also similar methods for objective function evaluations.  The method
\begin{verbatim}
double calculateFunctionValue(int idx, double* x, bool new_x);
\end{verbatim}
 will return the value of any constraint or objective function indexed by {\tt idx}. 
This method works strictly with {\tt double} data using an {\tt OSExpressionTree} object.

There is also a public variable, {\tt bUseExpTreeForFunEval} that, if set to {\tt true}, will cause the method
\begin{verbatim}
calculateAllConstraintFunctionValues(x, objLambda,  conLambda, true, highestOrder)
\end{verbatim}
to also use the OS expression tree for function evaluations when {\tt highestOrder = 0} rather than use 
the operator overloading in the CppAD tape.

\subsubdivision{Gradient Evaluation Methods}

One {\tt OSInstance} method for gradient calculations is
\begin{verbatim}
SparseJacobianMatrix *calculateAllConstraintFunctionGradients(double* x, double *objLambda,
     double *conLambda, bool new_x, int highestOrder)
\end{verbatim}
If a call has been placed to {\tt calculateAllConstraintFunctionValues} with {\tt highestOrder = 0}, then the appropriate call to get gradient evaluations is
\begin{verbatim}
calculateAllConstraintFunctionGradients( x, NULL, NULL,  false, 1);
\end{verbatim}
Note that in this function call {\tt new\_x = false}. This prevents a call to {\tt forwardAD()} with order 0 to get the function values.


If, at the current iterate, the Hessian of the Lagrangian function is also desired then an appropriate call is
\begin{verbatim}
calculateAllConstraintFunctionGradients(x, objLambda, conLambda, false, 2);
\end{verbatim}
In this case, if there was a prior call
\begin{verbatim}
calculateAllConstraintFunctionValues(x, w, z,  true, 0);
\end{verbatim}
then only first and second derivatives are calculated, not function values.

When calculating the gradients, if the number of nonlinear variables exceeds or is equal  to the number of rows,  a {\tt forwardAD(0, x)} sweep is used to get the function values,  and   a {\tt reverseAD(1, $e^{k}$)}  sweep for each unit vector  $e^{k}$ in the row space  is used to get the vector of first order partials for each row in the constraint Jacobian.  If the number of nonlinear variables is less then the number of rows then a {\tt forwardAD(0, x)} sweep  is used to get the function values and a {\tt forwardAD(1,  $e^{i}$)}  sweep for each unit vector  $e^{i}$ in the column space is used to get the vector of first order partials for each column in the constraint Jacobian.

Two other gradient methods are
\begin{verbatim}
SparseVector *calculateConstraintFunctionGradient(double* x,
    double *objLambda, double *conLambda,  int idx, bool new_x, int highestOrder);
\end{verbatim}
and
\begin{verbatim}
SparseVector *calculateConstraintFunctionGradient(double* x, int idx,
    bool new_x );
\end{verbatim}

Similar methods are available for the objective function; however, the objective function gradient methods treat the gradient of each objective function as a dense vector.


\subsubdivision{Hessian Evaluation Methods}

There are two methods for Hessian calculations.  The first method has the signature
\begin{verbatim}
SparseHessianMatrix *calculateLagrangianHessian( double* x,
    double *objLambda, double *conLambda, bool new_x, int highestOrder);
\end{verbatim}
so if either function or first derivatives have been calculated an appropriate call is
\begin{verbatim}
calculateLagrangianHessian( x, w, z, false, 2);
\end{verbatim}
If the Hessian of a single row or objective function is desired the following method is available
\begin{verbatim}
SparseHessianMatrix *calculateHessian( double* x, int idx, bool new_x);
\end{verbatim}






% Part 3 for folks who want to build the code
\throwpage

\pagenumbering{gobble}

\part{Building OS from source}

\pagenumbering{bychapter}

% this is a chapter of the user's manual
% it does not compile by itself

\section{Downloading the OS Project}\label{section:downloadsource}


\subsection{Auxiliary Software for Working with the OS Project} \label{section:auxiliarydownloads}

Compiling and modifying the OS project source\index{OS project!source code} code can be a daunting task,
made somewhat easier by the inclusion of configure scripts\index{configure!scripts} and makefiles\index{makefile}
in the distribution of the source. However, additional software packages are
sometimes needed or convenient, especially on Windows.
We collect in this section a number of recommended packages that we ourselves use in the development
and maintenance of the code.

\subsubsection{Subversion (SVN)}\label{section:svn}

The Subversion\index{Subversion} version control package is used to obtain the C++ source code\index{OS project!source code}.
Users with Unix operating systems will most likely have a command line svn client.  If an svn client is not present,
see~{\tt\UrlSvn} to download an svn client.   For Windows users we recommend the
svn client  TortoiseSVN.  (See~{\tt\UrlTortoiseSvn}.) Upon installation the TortoiseSVN client is integrated within the
Windows  Explorer.

\subsubsection{wget}\label{section:wget}
Certain third-party software (see Section~\ref{section:otherthirdparty}) is available in source form but is not
contained in the OS project distribution. Scripts are included to download this code using the
{\tt wget}\index{wget@{\tt wget}} executable.


%The {\tt wget} executable is used by the scripts, {\tt get.ASL},
%{\tt get.Blas}, etc. in the corresponding third-party subdirectories
%and makes it easy to download the software.
A Windows version of {\tt wget} is available at

%\begin{verbatim}
%http://www.christopherlewis.com/WGet/WGetFiles.htm
%\end{verbatim}

\medskip
\noindent{\tt\UrlWgetBinary}
\medskip

%Windows users are advised to download only the binary found in
%
%\begin{verbatim}
%http://www.christopherlewis.com/WGet/wget-1.10.2b.zip
%\end{verbatim}
%or the beta version of the new release at
%\begin{verbatim}
%http://www.christopherlewis.com/WGet/wget-1.11-beta-1b.zip
%\end{verbatim}


There is no need to rebuild the code locally, which relies on several levels of other software.

\subsubsection{Windows development platform}\label{section:windowsdevelopment}
A development platform is essential for users on Windows. OS Project provides support for Microsoft Visual Studio
(see Section~\ref{section:msvs}) and several unix emulators, including Cygwin (Section~\ref{section:cygwin}),
MinGW (Section~\ref{section:mingw}) and MSYS (Section~\ref{section:msys}). Download instructions for all of these
packages are included in the sections indicated.

\subsubsection{C++ compiler}\label{section:cpp}
A C++ compiler\index{C++ compiler} is needed to compile the OS source\index{OS project!source code}. This should be present 
under all unix\index{Unix} installations. If no C++ compiler is available on the system, the free {\tt gcc}\index{gcc} 
compiler can be downloaded from {\tt\UrlGcc}.

Microsoft Visual Studio can be configured with the Microsoft {\tt cl}\index{cl compiler@{\tt cl} compiler} compiler, 
which also works under MSYS\index{MSYS}. MinGW\index{MinGW} and Cygwin\index{Cygwin} are normally configured 
with the Gnu compiler collection ({\tt gcc}), although they can also be used with the {\tt cl} compiler. 
However, extreme care is needed if the last option is followed. {\tt gcc} and {\tt cl} have very different 
header files, and it is important to set up the {\tt \$PATH}\index{PATH@{\tt \$PATH}} variable correctly 
in order not to confuse the header files. 
In our experience, best results are achieved with the minimal unix-like installation, MSYS, and the 
Microsoft {\tt cl} compiler.

\subsubsection{Fortran Compiler}\label{section:fortran}
The COIN-OR project Ipopt\index{COIN-OR projects!Ipopt@{\tt Ipopt}} (see Section~\ref{section:ipopt}) and several of 
the third-party software described in Section~\ref{section:otherthirdparty} include Fortran\index{Fortran} subroutines, 
which must be compiled with a Fortran compiler if the user wants to include these projects in the build. A free 
Fortran~95 compiler
can be downloaded from {\tt\UrlGgs}. For Fortran~77 code (which includes the Blas\index{Third-party software!Blas},
HSL\index{Third-party software!HSL} and Lapack\index{Third-party software!Lapack} projects --- but {\bf not} Mumps\index{Third-party software!Mumps}) 
it might be sufficient to download the {\tt f2c}\index{f2c@{\tt f2c}} translator
which turns Fortran~77 code into code that can subsequently be fed into a C compiler.
The {\tt f2c} translator and the {\tt f2c} runtime library can be downloaded from {\tt\UrlFToC}.
Further details are available in the file {\tt BuildTools/compile\_f2c/INSTALL}, which is part of the OS distribution.

\subsubsection{{\tt flex} and {\tt bison} }\label{section:flex}
Users who want to edit the source code in the parsers described in
Section~\ref{section:osparsers} will need the additional  tools
{\tt flex}\index{flex@{\tt flex}} and {\tt bison}\index{bison@{\tt bison}}.  These can be downloaded from

%\begin{verbatim}
%http://sourceforge.net/project/showfiles.php?group\_id=2435
%\end{verbatim}

\medskip
\noindent{\tt\UrlMsysAddIns}
\medskip

\noindent and are listed at the Web site as

\begin{verbatim}
bison-2.3-MSYS-1.0.11-1
flex-2.5.33-MSYS-1.0.11-1
regex-0.12-MSYS-1.0.11-1
\end{verbatim}
The last file contains an important DLL, msys-regex-0.dll, without which {\tt flex} will not start.

\subsubsection{doxygen }\label{section:doxygen}
Doxygen\index{Doxygen}  ({\tt\UrlDoxygen}) is a document production system that can be used to prepare documentation
for the OS project and related software. For details, see Section~\ref{section:documentation}.

 
\subsection{Obtaining OS Source Code Using Subversion (SVN)}\label{section:downloadwithsvn}

For the rest of this documentation, we assume that  the name of the root directory
of the OS project\index{OS project!root directory} distribution is {\tt COIN-OS}.  
The {\tt COIN-OS} directory structure is illustrated in Figure~\ref{figure:osprojectrootdir}.
OS source code is mainly contained inside of the OS subdirectory. Other first level subdirectories are mostly
external projects (COIN-OR or third-party) that the OS project depends on.


\begin{figure}
\centering
%\includegraphics[scale=0.7]{\figurepath/OSProjectRootDirectory.png}
\includegraphics[scale=0.7]{./figures/OSProjectRootDirectory.png}
\caption{The OS distribution root directory.}
\label{figure:osprojectrootdir}
\end{figure}

\medskip

For Users on a Unix system\index{Downloading!subversion!unix}\index{Unix} such as Linux, Solaris, Mac OS X, etc., 
the source code is obtained as follows. In a command window execute:

%\begin{verbatim}
%svn co https://projects.coin-or.org/svn/OS/releases/1.1.0 COIN-OS
%\end{verbatim}

\medskip
\noindent{\tt svn co \UrlOsRelease\ COIN-OS}
\medskip

It is possible that on some systems you may get a message such as:
\begin{verbatim}
Error validating server certificate for 'https://projects.coin-or.org:443':
 - The certificate is not issued by a trusted authority. Use the
   fingerprint to validate the certificate manually!
Certificate information:
 - Hostname: projects.coin-or.org
 - Valid: from Jun 10 22:51:18 2007 GMT until Jun 15 21:00:28 2009 GMT
 - Issuer: 07969287, http://certificates.godaddy.com/repository, GoDaddy.com, Inc.,
Scottsdale, Arizona, US
 - Fingerprint: f7:26:0f:bb:e1:94:a5:23:7f:5c:cb:c3:9a:c4:74:51:e5:c7:4d:29
(R)eject, accept (t)emporarily or accept (p)ermanently?
\end{verbatim}

If so, select {\tt (p)} and you should not get this message again.

\medskip

For more information on downloading the OS project or other COIN-OR projects using SVN\index{SVN} see

\nopagebreak\medskip\nopagebreak
\noindent{\tt\UrlCoinDownload}.
\medskip

\vskip 8pt

On Windows\index{Downloading!subversion!Windows}\index{Microsoft Windows} with TortoiseSVN\index{TortoiseSVN}, 
create a directory {\tt COIN-OS}
in the desired location and right-click on this directory. Select the menu item {\tt SVN Checkout ...}
and in the textbox ``{\tt URL of Repository}'' give the URL for the version of the OS project you wish
to check out, for instance, 

\medskip
\noindent{\tt\UrlOsStable}.
\medskip


Now build the project as described in  Section~\ref{section:build}.

\medskip

The Java\index{Java} source code for  setting up a solver service with Apache Tomcat\index{Apache Tomcat} is 
checked out as follows:
%\begin{verbatim}
%svn co https://projects.coin-or.org/svn/OS/branches/OSjava  OSJava
%\end{verbatim}

\medskip
\noindent{\tt svn co \UrlOsJava\ OSJava}
\medskip

For more detail on running a Tomcat solver service  see  Section~\ref{section:tomcat}.








\subsection{Obtaining the OS Source Code From a Tarball or Zip File}\label{section:getTarBalls}

The OS source code can also be obtained from either a  tarball\index{Downloading!tarball} or
zip\index{Downloading!zip file} file.  This may be preferred for users who are not managing other
COIN-OR projects and wish to only work with periodic release versions of the code.  In order to obtain the code
from a Tarball or Zip file do the following.

\vskip 8pt

\begin{enumerate}[{\bf Step 1:}]

\item{}
In a browser open the link {\tt\UrlOsTarball}.  Listed at this page are files in the format:

\begin{verbatim}
OS-release_number.tgz
OS-release_number.zip
\end{verbatim}

\vskip 8pt

\item{}
Click on either the {\tt tgz} or {\tt zip} file and download to the desired directory.

\vskip 8pt

\item{}
Unpack the files. For {\tt tgz} do the following at the command line:
\begin{verbatim}
gunzip OS-release_number.tgz
tar -xvf OS-release_number.tar
\end{verbatim}

Windows users should be  able to double-click on the file {\tt OS-release\_number.zip} and have the directory unpacked.

\vskip 8pt

\item{}
(optional) Move the folder {\tt OS-release\_number} to the desired location and rename it to {\tt COIN-OS}.
\end{enumerate}


Now build the project as described in  Section~\ref{section:build}.







\subsection{Obtaining source for the OS Project API} \label{section:oslite}
The OS project is very extensive and relies on many other COIN-OR projects\index{COIN-OR}.
This may not be desirable for modeling language and solver developers who just wish to use the OS API
in conjunction with their modeling language or solver.  Hence there is also an ``OS lite'' download
that consists of all the code for the OS API and for reading and writing instance and solution files.
%
%We refer to this version of the project as {\tt OSCommon}\index{OSCommon@{\tt OSCommon}} and the code
%for this project is synched with the corresponding stable and release versions of the code.
%For example, to get Stable 1.1 of {\tt OSCommon} use the svn command
%\begin{verbatim}
%svn co https://projects.coin-or.org/svn/OS/stable/OSCommon1.1  OSCommon
%\end{verbatim}
%
We refer to this version of the project as {\tt OSCommon}\index{OSCommon@{\tt OSCommon}}. 
To get the current version of {\tt OSCommon} use the svn\index{SVN} command

\medskip
\noindent{\tt svn co \UrlOsCommon\ OSCommon}



\throwpage

\division{Building and Testing the OS Project}\label{section:build}


Once the OS source code is obtained, the OS libraries, {\tt OSSolverService}\index{OSSolverService@{\tt OSSolverService}} 
executable, and test examples can be built.
We describe how to do this on Unix/Linux\index{Unix} systems (see Section~\ref{section:unixbuilds})
and on Windows\index{Microsoft Windows} (see Section~\ref{section:windowsinstall}).

\subdivision{Building the OS Project on Unix/Linux Systems}\label{section:unixbuilds}

In order to build the OS project on Unix/Linux systems do the following.

\vskip 8pt

\begin{enumerate}[{\bf Step 1:}]
\item{} Connect to the OS distribution root directory ({\tt COIN-OS} in Figure~\ref{figure:osprojectrootdir}).

\vskip 8pt



\item{} \label{itemize:unixbuilds} Run the configure script that will generate the makefiles.
If you are running on a machine with a Fortran\index{Fortran} 95 compiler present (e.g., {\tt gfortran}), and you have
previously downloaded the third-party software packages {\tt BLAS}\index{Third-party software!Blas} 
and {\tt Mumps}\index{Third-party software!Mumps}
(see Section~\ref{section:ipopt}), run the command

\begin{verbatim}
./configure
\end{verbatim}
\index{configure}

\noindent otherwise use

\begin{verbatim}
./configure  COIN_SKIP_PROJECTS="Ipopt Bonmin"
\end{verbatim}
\index{COIN_SKIP_PROJECTS@{\tt COIN\_SKIP\_PROJECTS}}
as COIN-OR's {\tt Ipopt}\index{COIN-OR projects!Ipopt@{\tt Ipopt}} and
{\tt Bonmin}\index{COIN-OR projects!Bonmin@{\tt Bonmin}}
projects currently use Fortran to compile some of its dependent libraries.

\vskip 8pt
\noindent {\bf Notes:}

\begin{itemize}
\item If {\tt gfortran} is not present and you  wish to build the nonlinear solver {\tt Ipopt} see the instructions 
in Section~\ref{section:ipopt}.

\item When using {\tt configure} you may wish to use the {\tt -C} option. This
instructs {\tt configure} to use a cache file, {\tt config.cache}\index{configure!cache file}, to speed up configuration
by remembering and reusing the results of tests already performed.

\item For more information and options on the {\tt ./configure} script see

\noindent{\footnotesize {\tt\UrlCoinConfig}.}

\item You  cannot apply  {\tt COIN\_SKIP\_PROJECTS}\index{COIN_SKIP_PROJECTS@{\tt COIN\_SKIP\_PROJECTS}} to 
{\tt Cbc}\index{COIN-OR projects!Cbc@{\tt Cbc}}, 
{\tt Clp}\index{COIN-OR projects!Clp@{\tt Clp}}, 
{\tt Cgl}\index{COIN-OR projects!Cgl@{\tt Cgl}}, 
{\tt CoinUtils}\index{COIN-OR projects!CoinUtils@{\tt CoinUtils}}, 
%{\tt CppAD}\index{COIN-OR projects!CppAD@{\tt CppAD}}, 
or {\tt Osi}\index{COIN-OR projects, {\tt Osi}}.
These projects must be present.
\end{itemize}



\item{}  Run the make files.

\index{make@{\tt make}}
\begin{verbatim}
make
\end{verbatim}

\item{} Run the {\tt unitTest}\index{unitTest@{\tt unitTest}}.

\index{make test@{\tt make test}}
\begin{verbatim}
make test
\end{verbatim}

\ifknitro
Depending upon which third-party software you have installed, the result of running the {\tt unitTest} should look
something like (we have included the third-party solvers LINDO\index{LINDO} and Knitro\index{Knitro} in the test 
results below; they are not part of the default build):


{\small
\begin{verbatim}
HERE ARE THE UNIT TEST RESULTS:

Solved problem avion2.osil with Ipopt
Solved problem HS071.osil with Ipopt
Solved problem rosenbrockmod.osil with Ipopt
Solved problem parincQuadratic.osil with Ipopt
Solved problem parincLinear.osil with Ipopt
Solved problem callBack.osil with Ipopt
Solved problem callBackRowMajor.osil with Ipopt
Solved problem parincLinear.osil with Clp
Solved problem p0033.osil with Cbc
Solved problem rosenbrockmod.osil with Knitro
Solved problem callBackTest.osil with Knitro
Solved problem parincQuadratic.osil with Knitro
Solved problem HS071_NLP.osil with Knitro
Solved problem p0033.osil with SYMPHONY
Solved problem parincLinear.osil with DyLP
Solved problem volumeTest.osil with Vol
Solved problem p0033.osil with GLPK
Solved problem lindoapiaddins.osil with Lindo
Solved problem rosenbrockmod.osil with Lindo
Solved problem parincQuadratic.osil with Lindo
Solved problem wayneQuadratic.osil with Lindo
Test the MPS -> OSiL converter on parinc.mps using Cbc
Test the AMPL nl -> OSiL converter on hs71.nl using LINDO
Test a problem written in b64 and then converted to OSInstance
Successful test of OSiL parser on problem parincLinear.osil
Successful test of OSrL parser on problem parincLinear.osrl
Successful test of prefix and postfix conversion routines on problem rosenbrockmod.osil
Successful test of all of the nonlinear operators on file testOperators.osil
Successful test of AD gradient and Hessian calculations on problem CppADTestLag.osil

All tests completed successfully
\end{verbatim}
}
\else
Depending upon which third-party software you have installed, the result of running the {\tt unitTest}\index{OS project!unit test}
should look
something like (we have included the third-party solver LINDO\index{LINDO} in the test results below; it is not
part of the default build):


{\small
\begin{verbatim}
HERE ARE THE UNIT TEST RESULTS:

Solved problem avion2.osil with Ipopt
Solved problem HS071.osil with Ipopt
Solved problem rosenbrockmod.osil with Ipopt
Solved problem parincQuadratic.osil with Ipopt
Solved problem parincLinear.osil with Ipopt
Solved problem callBack.osil with Ipopt
Solved problem callBackRowMajor.osil with Ipopt
Solved problem parincLinear.osil with Clp
Solved problem p0033.osil with Cbc
Solved problem p0033.osil with SYMPHONY
Solved problem parincLinear.osil with DyLP
Solved problem volumeTest.osil with Vol
Solved problem p0033.osil with GLPK
Solved problem lindoapiaddins.osil with Lindo
Solved problem rosenbrockmod.osil with Lindo
Solved problem parincQuadratic.osil with Lindo
Solved problem wayneQuadratic.osil with Lindo
Test the MPS -> OSiL converter on parinc.mps using Cbc
Test the AMPL nl -> OSiL converter on hs71.nl using LINDO
Test a problem written in b64 and then converted to OSInstance
Successful test of OSiL parser on problem parincLinear.osil
Successful test of OSrL parser on problem parincLinear.osrl
Successful test of prefix and postfix conversion routines on problem rosenbrockmod.osil
Successful test of all of the nonlinear operators on file testOperators.osil
Successful test of AD gradient and Hessian calculations on problem CppADTestLag.osil

All tests completed successfully
\end{verbatim}
}
\fi

If you do not see
\begin{verbatim}
All tests completed successfully
\end{verbatim}
then you have not passed the unitTest and hopefully some semi-intelligible error message was given.

\vskip 8pt

\item{}  Install the libraries and executables.

\index{make install@{\tt make install}}
\begin{verbatim}
make install
\end{verbatim}

This will install all of the libraries in the  {\tt lib} directory.  In particular, the main OS library
{\tt libOS}\index{LibOS@{\tt LibOS}}
along with the libraries of the other COIN-OR projects  that download with the OS project will get installed
in the {\tt lib} directory.  In addition the {\tt make install} command will install several executable programs 
in the {\tt bin} directory, depending on the parameters on the {\tt configure command}.  One of these binaries 
is {\tt OSSolverService}\index{OSSolverService@{\tt OSSolverService}} which is the main OS project executable.
This is described in Section~\ref{section:ossolverservice}. In addition 
{\tt Clp}\index{COIN-OR projects!Clp@{\tt Clp}},
{\tt Cbc}\index{COIN-OR projects!Cbc@{\tt Cbc}}, 
{\tt Ipopt}\index{COIN-OR projects!Ipopt@{\tt Ipopt}},
{\tt Bonmin}\index{COIN-OR projects!Bonmin@{\tt Bonmin}},
{\tt Couenne}\index{COIN-OR projects!Couenne@{\tt Couenne}}
and {\tt SYMPHONY}\index{COIN-OR projects!SYMPHONY@{\tt SYMPHONY}}
get installed  in the {\tt bin} directory.
Necessary header files are installed in the {\tt include} directory.   In this case, {\tt bin}, {\tt lib}
and {\tt include} are all subdirectories of where {\tt ./configure}\index{configure} is run.   
If the user wants these files
installed elsewhere, then {\tt configure} should specify the {\tt prefix} of these directories.  That is,


\begin{verbatim}
./configure  --prefix=prefixDirectory  COIN_SKIP_PROJECTS="Ipopt Bonmin"
\end{verbatim}
\index{COIN_SKIP_PROJECTS@{\tt COIN\_SKIP\_PROJECTS}}

For example, running

\begin{verbatim}
./configure  --prefix=/usr/local  COIN_SKIP_PROJECTS="Ipopt Bonmin"
\end{verbatim}

\noindent and then running {\tt make}\index{make@{\tt make}}
 and {\tt make install}\index{make install@{\tt make install}}
 will put the relevant files in

\begin{verbatim}
/usr/local/bin
/usr/local/include
/usr/local/lib
\end{verbatim}

\end{enumerate}

\vskip 8pt

{\bf Run an Example!}  If {\tt make test}\index{make test@{\tt make test}}
 works, proceed to Section~\ref{section:ossolverservice} 
to run the key executable, {\tt OSSolverService}\index{OSSolverService@{\tt OSSolverService}}.



\subsubdivision{Building the OS Project on Mac OS X}\label{section:unixmacbuilds}

When building OS on Mac OS X 10.5.x (Leopard)   it may be necessary  to add the following to the configure line


\begin{verbatim}
ADD_CXXFLAGS="-mmacosx-version-min=10.4" 
ADD_CFLAGS="-mmacosx-version-min=10.4" 
ADD_FFLAGS="-mmacosx-version-min=10.4"
LDFLAGS="-flat_namespace"
\end{verbatim}

Also, the Mac OS X operating system does not come configured with the gcc compiler. Users wanting to build the OS project on the Mac should do the following:

\begin{itemize}
\item Install the Xcode developer tools.  These are available on the install DVD that comes with the machine or at the Apple developer site. See

\url{http://developer.apple.com/technology/xcode.html}

\item Install a Fortran compiler.  We have had good luck with the GNU {\bf gfortran} compiler. Platform specific binaries for the various Mac platforms (Leopard and Tiger, Intel and Power PC) are obtained at

\url{http://hpc.sourceforge.net/}

We followed the instructions and installed the binary using the command


\begin{verbatim}
sudo tar -xvf gcc-bin.tar -C /
\end{verbatim}

\end{itemize}

We have also successfully used the fink project, see

\url{http://www.finkproject.org/}

to download and build gcc/g++/gfortran compilers from source code. 


\subdivision{Building the OS Project on Windows}\label{section:windowsinstall}

There are a number of options open to Windows users.   First, if you wish to work with source code\index{OS project!source code}
we recommend downloading  the svn client, TortoiseSVN\index{TortoiseSVN}.  (See Section~\ref{section:svn}.)  
With TortoiseSVN
in the Windows Explorer connect to the directory (e.g., COIN-OS) where you wish to put the OS code.
Right-click on the directory and select {\tt SVN Checkout}.   In the textbox, {\tt URL of Repository}
give the URL for the version of the OS project you wish to check out, e.g.,

\medskip
\noindent{\tt\UrlOsStable}.
\medskip

Also, if you plan to build any of the projects contained in {\tt ThirdParty}\index{Third-party software}
(e.g., ASL)\index{Third-party software!ASL} we recommend using {\tt wget}\index{wget@{\tt wget}}. 
(See Section~\ref{section:wget}.)


\subsubdivision{Microsoft Visual Studio (MSVS)} \label{section:msvs}

\index{Microsoft Visual Studio|(}%
Microsoft Visual Studio solution and project files are provided for users of Windows and the Microsoft Visual Studio IDE.
We currently support Versions 8 and~9. These versions are also sometimes referred to by their
(approximate) release dates, which is 2008 for Version~9 and 2005 for Version~8.   In addition there is
a free version of the Visual Studio IDE C++ compiler,  called Visual C++ Express Edition\index{C++ compiler}.

The following steps are necessary to build the OS project using the  Microsoft Visual Studio IDE.

\begin{enumerate}[Step 1.] \setcounter{enumi}{-1}
\item{} If the C++ compiler {\tt cl} is already
installed,  go to  to Step~\ref{enumerate:winbuild2}.

\item{} Download and install the Visual C++ Express Edition, which is available for free at Microsoft's web site.
Version~9 is at {\tt\UrlCl}.
This download contains the Microsoft {\tt cl} C++ compiler along with necessary libraries.

\item{} \label{enumerate:winbuild2} The part of the OS library responsible for communication with a remote server depends on some
underlying Windows socket header files and libraries. These files are part of the commercial for-pay version,
but are not included in the Visual C++ Express download. If you have the Express Edition, it is necessary
to also download and install the Windows Platform SDK\index{Windows Platform SDK}, which can be found at

\medskip
\noindent{\scriptsize\tt\UrlSdk}.
\medskip

\item{} In the COIN-OR/OS directory you will find the folder MSVisualStudio,
which contains root directories organized by the version of Visual Studio.
We currently provide solution files for Version~8 and Version~9.
Each contains the file {\tt OS.sln}\index{OS sln@{\tt OS.sln}} and project files
for building the unitTest\index{unitTest@{\tt unitTest}} ({\tt OSTest.vcproj}\index{OSTest.vcproj@{\tt OSTest.vcproj}}),
the OSSolverService ({\tt OSSolverService.vcproj}\index{OSSolverService.vcproj@{\tt OSSolverService.vcproj}}) and
the OS libraries
({\tt libOSCommon.vcproj}\index{libOSCommon.vcproj@{\tt libOSCommon.vcproj}} and
({\tt libOSSolvers.vcproj}\index{libOSSolvers.vcproj@{\tt libOSSolvers.vcproj}}).
The Microsoft Visual Studio files are automatically downloaded with an SVN\index{SVN} checkout.
They are also contained in the tarballs (see Section~\ref{section:getTarBalls}).

Open the solution file or the individual project files (for instance by double-clicking
on them in Windows Explorer)  and select Build from the menu bar.
%If you have ASL\index{Third-party software!ASL} (see Section~\ref{section:ASL}) downloaded,
%you can also build the {\tt OSAmplClient}\index{OSAmplClient@{\tt OSAmplClient}} (see Section~\ref{section:amplclient})
%by modifying the Configuration Manager\index{Microsoft Visual Studio!Configuration Manager} and selecting the
%two projects {\tt libOSnl2OSiL}\index{libOSnl2OSiL@{\tt libOSnl2OSiL}}
%and {\tt OSAmplClient}\index{OSAmplClient@{\tt OSAmplClient}},
%which by default are not included in the build.

\item{} Run the {\tt unitTest}\index{unitTest@{\tt unitTest}}. Connect to the directory {\tt COIN-OR/OS/test} and run 
either the release or debug version of the {\tt unitTest} executable.
\end{enumerate}

%The solution file for version~7 provides two configurations, {\tt Debug} and {\tt Release}.
%The former includes debug information, but both are configured without Ipopt
%(see Section~\ref{section:ipopt}) or any of the third-party software described in
%section~\ref{section:otherthirdparty}.
%The solution file {\tt OS.sln}\index{OS sln@{\tt OS.sln}} contains three configurations, 
%{\tt Debug}\index{Microsoft Visual Studio!{\tt Debug} configuration} and 
%{\tt Release}\index{Microsoft Visual Studio!{\tt Release} configuration}, 
%both of which are configured without {\tt Ipopt}, as well as 
%{\tt Release-Plus}\index{Microsoft Visual Studio!{\tt Release-Plus} configuration},
%which can be used to add {\tt Ipopt}\index{COIN-OR projects!Ipopt@{\tt Ipopt}}, 
%{\tt Bonmin}\index{COIN-OR projects!Bonmin@{\tt Bonmin}} and ASL\index{Third-party software!ASL}
%(see Section~\ref{section:ASL}). In order to build this configuration successfully,
%the user must first download and process additional third-party software\index{Third-party software} as explained in 
%sections \ref{section:ipopt-msvs} and~\ref{section:ASL}.

The solution file {\tt OS.sln}\index{OS sln@{\tt OS.sln}} contains two configurations, 
{\tt Debug}\index{Microsoft Visual Studio!{\tt Debug} configuration} and 
{\tt Release}\index{Microsoft Visual Studio!{\tt Release} configuration}, 
both of which are configured without {\tt Ipopt}.

\index{Microsoft Visual Studio|)}%


\subsubdivision{Visual Studio Examples Distribution}\label{section:vsexamples}

Many users will not be interested in actually building the OS project from source code.   At the link
{\tt\UrlOsWin} are  binaries for using the OS project.
There are also Visual Studio project files for building applications that use the precompiled OS libraries.
In particular, download and unpack the file

\begin{verbatim}
OS-version_number-VisualStudio.zip
\end{verbatim}
\index{file naming conventions}

This zip archive contains a  {\tt bin} directory that holds  the executable  {\tt OSSolverService.exe}.
The {\tt OSSolverService.exe} is configured to run, out-of-the-box,   the following solvers.

\begin{itemize}

\item Bonmin

\item Clp

\item Cbc

\item Couenne

\item DyLP

\item Ipopt

\item SYMPHONY

\item Vol

\end{itemize}
The libraries necessary to run these solvers are included in the download.  {\it No additional software is necessary
to solve models with these solvers!}   See Section~\ref{section:ossolverservice} for details on how to use the
{\tt OSSolverService.exe} executable for solving optimization problems.


The {\tt bin} directory also contains the {\tt OSAmplClient.exe} executable. If the user has a Windows version of AMPL,
then AMPL can be used to invoke all of the solvers mentioned above through the {\tt OSAmplClient}.  For details
see Section~\ref{section:amplclient}.



This zip archive also contains a  {\tt lib} directory that holds  libraries
for a number of COIN-OR projects, including OS. It is possible to build
customized optimization applications that link against these libraries.
We provide several examples that use various aspects of the OS project
in order to build customized applications. The Visual Studio example solution
file is named {\tt osExamples.sln} and is found in the folder
{\tt MSVisualStudioOSExamples}. The solution file {\tt osExamples.sln}
currently contains nine projects (examples). These are described in more
detail in Section~\ref{section:examples}.

\iffalse
\begin{itemize}

\item[]  {\bf addCuts --} this project illustrates the use of  the {\tt Cbc} and {\tt Cgl} projects.
A file ({\tt p0033.osil}) in OSiL format is used to create an OSInstance object. The linear programming relaxation
is solved. Then, Gomory, simple rounding, and knapsack cuts are added using {\tt Cgl}.  The model is then optimized
using {\tt Cbc}.



\item[]  {\bf algorithmicDiff --} this project illustrates the {\tt calculate()} method calls in the {\tt OSInstance} class.
These {\tt calculate()} calls are used to calculate function values, gradients, and Hessians. These methods make underlying
calls to the {\tt CppAD} project.


\item[]  {\bf instanceGenerator --}  this project shows  how to build an instance using the {\tt OSInstance} class.
A number of key nonlinear operators are illustrated.


\item[]  {\bf osRemoteTest --}  this project shows  how to call a remote solver using Web Services.
{\bf Windows usrs should note}
that this project links to {\tt wsock32.lib}, which is not part of the Visual Studio  Express Package.  It is necessary
to also download and install the Windows Platform SDK\index{Windows Platform SDK}, which can be found at

\medskip
\noindent{\scriptsize\tt\UrlSdk}. 
\medskip
\noindent See also Section~\ref{section:msvs}.

\item[] {\bf osModDemo --} this provides yet another illustration of how to build an optimization instance using the
{\tt OSInstance} class.  In addition, this project illustrates how to modify and in-memory instance.   Finally, this project  shows how to build solver objects and use the solver object to
optimize the problem. In this particular case, the {\tt Clp} solver is used.

\end{itemize}


In addition, in the zip archive there is a folder {\tt MSVisualStudioTemplate}. This project contains a simple
{\tt Hello World} demo in the code {\tt demoCode.cpp}. However, the
solution file configured to link with all
of the libraries in the {\tt lib} directory and pointing to all of the
header files in the {\tt include} directory.
The user can simply replace what is currently in {\tt demoCode.cpp} with his or her own code.
\fi




\subsubdivision{Cygwin}\label{section:cygwin}

{\tt Cygwin} provides a Unix emulation environment for Windows. It comes with numerous tools and libraries including the {\tt gcc} compilers. See {\tt www.cygwin.com}.   Cygwin can be used with the Gnu Compiler Collection ({\tt gcc}) or with the Microsoft {\tt cl} compiler.

\vskip 8pt

\index{Cygwin|(}{\bf Using Cygwin with {\tt gcc}:}  With Cygwin and the corresponding {\tt gcc} compiler the OS project
is built exactly as described in Section~\ref{section:unixbuilds}. If you previously downloaded Cygwin with
gnome make version 3.81-1,  you must obtain a fixed 3.81 version from {\tt\UrlCygwinMake}.
(See also
%the Cygwin mailing list postings \url{http://cygwin.com/ml/cygwin/2006-09/msg00315.html} and \url{http://cygwin.com/ml/cygwin/2006-09/msg00153.html}) and
the discussion at {\tt\UrlCoinCygwin}.)


\vskip 8pt

{\bf Using Cygwin with Microsoft {\tt cl}:}   Users who are extremely adventuresome and have an abundance  of free time on their hands may wish to use Cygwin with the Microsoft {\tt cl} compiler to build the OS project.   The following steps have led to a successful build.


\begin{enumerate}[Step 1:]
\item{}  Download {\tt Cygwin}  from {\tt\UrlCygwinSetup} and install.




\item{}  Download  Visual Studio Express C++ at  

{\tt\UrlCl}.


\item{}  The part of the OS library responsible for communication with a remote server depends on some
underlying Windows socket header files and libraries. Therefore it is necessary to also download and install
the Windows Platform SDK\index{Windows Platform SDK}. Download the necessary files at

{\scriptsize\tt\UrlSdk}

 and install.



\item{}  Set the Cygwin search path configuration. This is important.
This step is necessary to ensure that Cygwin   looks for compilers, linkers, etc in the correct order.  The right order of directories  is: MSVS command directories, Cygwin command directories, and finally Windows command directories.  This is illustrated below.

\begin{itemize}

 \item First, Cygwin should look in the Microsoft Visual Studio directories.
If a standard Visual Studio install is done, the following  should be part of the
Cygwin search path.

\begin{verbatim}
.
:/cygdrive/c/Program Files/Microsoft Visual Studio 8/Common7/IDE
:/cygdrive/c/Program Files/Microsoft Visual Studio 8/VC/bin
:/cygdrive/c/Program Files/Microsoft Visual Studio 8/Common7/Tools
:/cygdrive/c/Program Files/Microsoft Visual Studio 8/SDK/v2.0/Bin
:/cygdrive/c/Program Files/Microsoft Visual Studio 8/VC/vcpackages
:/cygdrive/c/WINDOWS/Microsoft.NET/Framework/v2.0.50727
\end{verbatim}

\item Second, Cygwin should next search its  command directories.  The following is typical of a standard install.

\begin{verbatim}
/bin:/usr/local/bin:/usr/bin:/bin:/usr/X11R6/bin
\end{verbatim}

\item Third, Cygwin should search the Windows specific command directories.  The following is typical.

{\scriptsize
\begin{verbatim}
:/cygdrive/c/WINDOWS/system32:/cygdrive/c/WINDOWS
:/cygdrive/c/WINDOWS/System32/Wbem:/cygdrive/c/Program Files/ATI Technologies/ATI Control Panel
:/cygdrive/c/Program Files/Common Files/Roxio Shared/DLLShared/
:/cygdrive/c/Program Files/QuickTime/QTSystem/:/cygdrive/c/Program Files/Microsoft SQL Server/90/Tools/bin/
:/cygdrive/c/Program Files/Microsoft Platform SDK for Windows Server 2003 R2/Bin/
:/cygdrive/c/Program Files/Microsoft Platform SDK for Windows Server 2003 R2/Bin/WinNT/
:/cygdrive/c/Program Files/SSH Communications Security/SSH Secure Shell
:/cygdrive/d/SSH
\end{verbatim}
}


\end{itemize}
Open the Cygwin shell and check the value of {\tt \$PATH}\index{PATH@{\tt \$PATH}}. If directories don't appear in an order described above,
then the {\tt \$PATH} value needs to be reset.

%\item{} This step is necessary only if you wish to build with the AMPL {\tt ASL} solver
%library\index{Third-party software!ASL}.
%Unfortunately, and we regret this, but at the time of this writing the working version of {\tt ASL} for cygwin/
%cl build is its trunk version. This means that it is necessary to download the trunk version separately
%and replace the release version we have distributed with the trunk version.  The URL for the trunk
%version is
%
%\begin{verbatim}
%co https://projects.coin-or.org/svn/BuildTools/ThirdParty/ASL/trunk  ASL
%\end{verbatim}





\item{} Build the OS project (or any COIN-OR project). If you wish to avoid the FORTRAN\index{Fortran} related issues you should
build without {\tt Ipopt}\index{COIN-OR projects!Ipopt@{\tt Ipopt}}, 
{\tt Bonmin}\index{COIN-OR projects!Bonmin@{\tt Bonmin}} and {\tt Couenne}\index{COIN-OR projects!Couenne@{\tt Couenne}}. 
Issue the following command in the project root.
\begin{verbatim}
./configure COIN_SKIP_PROJECTS="Ipopt Bonmin Couenne" --enable-doscompile=msvc
\end{verbatim}
\index{COIN_SKIP_PROJECTS@{\tt COIN\_SKIP\_PROJECTS}}

If you wish to build with {\tt Ipopt} or {\tt Bonmin} and {\tt Couenne}, which depend on it, 
then FORTRAN is required --- and Visual Studio does not ship with a FORTRAN compiler.
The following is a work-around. (See also Section~\ref{section:ipopt}.)

\begin{enumerate}[Step a.]

\item{}  Obtain one of the   Harwell Subroutine Library (HSL)\index{Third-party software!HSL} routines
{\tt ma27ad.f} or {\tt MA57ad.f}. See {\tt\UrlHsl}.  Put the Harwell code in the
directory {\tt ThirdParty/HSL}. (Note the case in the file names, which is relevant in a unix-like environment.)




\item{}  Follow the instructions for downloading and installing the {\tt f2c}\index{f2c@{\tt f2c}} compiler from Netlib.
The installation instructions for this are in the {\tt INSTALL} file in
\begin{verbatim}
BuildTools/compile_f2c
\end{verbatim}



\item{}  Run the configure script

\begin{verbatim}
 ./configure  --enable-doscompile=msvc
\end{verbatim}


\end{enumerate}


\end{enumerate}
\index{Cygwin|)}



\subsubdivision{MinGW} \label{section:mingw}


MinGW\index{MinGW} (Minimalist GNU for Windows) is a set of runtime headers to be used with the GNU {\tt gcc} compilers for Windows.
See \url{www.mingw.org}. As with Cygwin, the OS project is  built exactly as described in Section~\ref{section:unixbuilds}.

The MinGW installation includes the {\tt gcc} compiler, which can interact negatively with the Microsoft {\tt cl} compiler.
For that reason it is advisable to download the even smaller installation MSYS (see next section) if you intend to
build any software with the Microsoft Visual Studio suite.

\vskip 8pt

{\bf Warning:} A user of  MSYS  with MinGW gcc version 4.4.0   got an error about a
missing library  ``pthreadsGC2.dll'' when running the OS {\tt unitTest.}  This user installed {\tt pthreadsGC2.dll} from
\begin{center}
 \url{ftp://sources.redhat.com/pub/pthreads-win32/dll-latest/lib/pthreadGC2.dll}
\end{center}
and reported that the problem then went away.


\subsubdivision{MSYS} \label{section:msys}

\index{MSYS|(}%
MSYS (Minimal SYStem) provides an easy way to use the COIN-OS build system with compilers/linkers of your own choice,
such as the Microsoft command line C++ {\tt cl} compiler.  MSYS is intended as an alternative to the DOS command window.
It is an application that gives the user a Bourne shell that can run {\tt configure}  scripts and makefiles\index{makefile}.
No compilers come with MSYS.
In the Cygwin\index{Cygwin}, MinGW\index{MinGW}, and MSYS\index{MSYS} hierarchy, it is at the bottom of the food chain in terms of tools provided.
However, it is very easy to use and build the OS project with MSYS.    In this discussion we assume that the user
has downloaded the OS source code (most likely  with TortoiseSVN)\index{TortoiseSVN} 
and that the {\tt cl} compiler\index{cl compiler@{\tt cl} compiler} is present.
The project is built using the following steps.

\vskip 8pt

\noindent {\bf Note:}

\begin{itemize}

\item If you wish to use the third-party software with MSYS it is best to get {\tt wget}\index{wget@{\tt wget}}.
See Section~\ref{section:wget}.

 \item Do not put any imbedded blanks in the path to the OS project.
\end{itemize}



Execute the following steps to use the Microsoft C++ {\tt cl} compiler with MSYS.


\begin{enumerate}[Step 1.]

\item{} Download {\tt MSYS} at

{\noindent{\small\tt\UrlMingw}}

and install.  Double-clicking on the MSYS icon will open a Bourne shell window.

\item{}  Download  Visual Studio Express C++ at 

\noindent{\scriptsize\tt\UrlCl}

and install.

 \item{}  The part of the OS library responsible for communication with a remote server depends on some underlying
Windows socket header files and libraries. Therefore it is necessary to also download and install
the Windows Platform SDK\index{Windows Platform SDK}. Download the necessary files at

\noindent{\scriptsize\tt\UrlSdk}

 and install.

\item{}   Set the Visual Studio environment variables so that paths to the necessary libraries and header files  are recognized.  Assuming that a standard installation was done for the Visual Studio Express and the Windows Platform SDK set the variables as follows:

\begin{verbatim}
PATH=C:\Program Files\Microsoft Visual Studio 8\Common7\IDE;
C:\Program Files\Microsoft Visual Studio 8\VC\BIN;
C:\Program Files\Microsoft Visual Studio 8\Common7\Tools;
C:\Program Files\Microsoft Visual Studio 8\SDK\v2.0\bin;
C:\WINDOWS\Microsoft.NET\Framework\v2.0.50727;
C:\Program Files\Microsoft Visual Studio 8\VC\VCPackages


INCLUDE=C:\Program Files\Microsoft Visual Studio 8\VC\INCLUDE;
C:\Program Files\Microsoft Platform SDK for Windows Server 2003 R2\Include

LIB = C:\Program Files\Microsoft Visual Studio 8\VC\LIB;
C:\Program Files\Microsoft Visual Studio 8\SDK\v2.0\lib;
C:\Program Files\Microsoft Platform SDK for Windows Server 2003 R2\Lib
\end{verbatim}

The environment variables can be set using the {\tt System Properties} in the Windows {\tt Control Panel}.


\item{}  In the MSYS command window connect to the root of the OS project and run the {\tt configure} 
script  followed by {\tt make}\index{make@{\tt make}}  as described in Section~\ref{section:unixbuilds}.

\end{enumerate}



{\bf Run an Example!}  If {\tt make test}\index{make test@{\tt make test}}
 works, proceed to Section~\ref{section:ossolverservice} to run the key executable, {\tt OSSolverService}.


Microsoft Windows users who wish to obtain MSYS for building the OS project can download
the appropriate software at {\tt \UrlMsys}.
The user may find this Web site confusing.
It is only necessary to download what is referred to as the {\bf MSYS Base System}.
As of this writing the most recent version is MSYS-\MsysVer.
This file is listed as {\tt \MsysFile} and the  binary download is

\noindent{\footnotesize\tt\UrlMsysBinary}

This will provide the necessary Bourne shell for executing the configure scripts.
Users who want to edit the source code in the parsers described in
Section~\ref{section:osparsers} will need the additional  tools
{\bf flex}\index{flex@{\tt flex}} and {\bf bison}\index{bison@{\tt bison}} 
as described in Section~\ref{section:flex}\index{MSYS|)}.



\subdivision{VPATH Installations} \label{section:vpath}

\index{VPATH|(}%
It is possible to build the OS project in a directory that is different from
the directory where the source code is present. This is called a {\tt VPATH}
compilation.  A {\tt VPATH}  compilation  is very useful if you wish to
build several versions (e.g., debug and non-debug versions, or versions with
availability of various combinations of third-party software) of the OS
project from a single copy of the source code.

For  example, assume you wish to build a debug version\index{debug version, MSYS|(} of the OS project in
the directory {\tt vpath-debug} and that {\tt ../COIN-OS} is the path to the
root of the OS project distribution.  Create the {\tt vpath-debug} directory,
leaving it empty for the moment.
From the {\tt vpath-debug} directory,
run {\tt configure} as follows:

\begin{verbatim}
../COIN-OS/configure --enable-debug
\end{verbatim}
%
After you run {\tt configure}, the OS distribution directory structure (see Figure~\ref{figure:osprojectrootdir})
will be mirrored in the {\tt vpath-debug} directory, and all of the necessary
{\tt Makefile}s\index{makefile|(} will be copied there.  Next from the {\tt vpath-debug} directory execute

\index{make@{\tt make}}
\begin{verbatim}
make
\end{verbatim}
%
and all of  the libraries created will be in their respective directories
inside {\tt vpath-debug} and not {\tt ../COIN-OS}.\index{debug version, MSYS|)}

\vskip 8pt
\noindent {\bf Notes:} 
\index{configure|(}
\begin{enumerate}
\item{} If you have already run the {\tt configure} script
inside the {\tt ../COIN-OS} directory, you cannot do a {\tt VPATH} build
until you have run
%
\index{make distclean@{\tt make distclean}}
\begin{verbatim}
make distclean
\end{verbatim}
%
in the {\tt ../COIN-OS} directory.

\item{}Note also that {\tt configure} automatically detects the presence of third-party software and prepares
the configuration and make files\index{makefile|)} accordingly. Once you have downloaded, 
e.g., Blas\index{Third-party software!Blas}, you must specify
%
\begin{verbatim}
configure COIN_SKIP_PROJECTS="ThirdParty/Blas"
\end{verbatim}
\index{COIN_SKIP_PROJECTS@{\tt COIN\_SKIP\_PROJECTS}}
%
if you want to recreate the default configuration.%
\index{VPATH|)}

\item{}If you work with the trunk\index{OS project!trunk version} version of OS, it is possible that files are added to
and removed from the distribution due to development activities. These files are not recognized properly
by the system unless it is reconfigured by running
%
\index{make distclean@{\tt make distclean}}
\begin{verbatim}
make distclean
\end{verbatim}
followed by 
\begin{verbatim}
./configure
\end{verbatim}
in the {\tt VPATH} directory.

\item{}You can customize compiler flags, linker options, include directories, and many other options by setting
appropriate environment variables. For further information you may want to consult the built-in help function by specifying
\begin{verbatim}
./configure --help
\end{verbatim}
or the help file available at the homepage of the {\tt BuildTools} project ({\UrlBuildtools}).
\end{enumerate}
\index{configure|)}


\subdivision{COIN-OR Projects Requiring Fortran}\label{section:ipopt}

\index{COIN-OR projects!Ipopt@{\tt Ipopt}|(}%
\index{COIN-OR projects!Bonmin@{\tt Bonmin}|(}%
\index{COIN-OR projects!Couenne@{\tt Couenne}|(}%
Ipopt, Bonmin and Couenne are COIN-OR projects 
(\url{http://projects.coin-or.org/Ipopt}, \url{http://projects.coin-or.org/Bonmin}, \url{http://projects.coin-or.org/Couenne})
and are included in the download with the OS project.
However, unlike the other COIN-OR projects that download with OS, these projects require third-party software
that is based on FORTRAN\index{Fortran} and is {\it not} part of the default distribution. Care must therefore be taken if
you wish to build OS with the Ipopt, Bonmin or Couenne solver. It is further important to know that there is a 
dependency between these three projects. Ipopt is the only one using Fortran directly, but Bonmin relies on Ipopt
for its solver, and Couenne is similarly dependent on both Ipopt and Bonmin. Neither Bonmin nor Couenne can therefore 
be installed in isolation.

You can exclude all three of these projects from the OS build by adding the option

\begin{verbatim}
COIN_SKIP_PROJECTS="Ipopt Bonmin Couenne"
\end{verbatim}
to the {\tt configure} script.

\ifipopt
\subsubdivision{Building Ipopt, Bonmin and Couenne in Unix or a Unix-like environment} \label{section:ipopt-unix}
If you are working in Unix or one of the Unix-like environments described in
section~\ref{section:windowsinstall}, you can proceed as follows.
\else
If you do choose to build {\tt Ipopt}, {\tt Bonmin} and {\tt Couenne}, it is best to work in Unix or one of the 
Unix-like environments described in Section~\ref{section:windowsinstall} (we recommend MSYS)\index{MSYS}.
\fi
To get the necessary third-party software\index{Third-party software}, first
connect into the {\tt ThirdParty} directory. Then execute the following commands:

\begin{verbatim}
$ cd Blas
$ ./get.Blas
$ cd ../Lapack
$ ./get.Lapack
$ cd ../Mumps
$ ./get.Mumps
\end{verbatim}



What you do next depends upon whether or not a FORTRAN\index{Fortran} compiler is present, and if so, which version
of FORTRAN.  There are several options. See also

%\begin{verbatim}
%http://www.coin-or.org/Ipopt/documentation/node13.html
%\end{verbatim}

\medskip
\noindent{\tt\UrlIpoptDocxiii}


\begin{enumerate}[{Option} 1.]

\item{}   If you \ifipopt\else are building in a Unix-like environment and \fi have a Fortran 95 compiler that
recognizes embedded preprocessor statements (such as {\tt gfortran} --- see~{\tt\UrlGfortran}
or {\tt g95} --- see~{\tt\UrlGgs}), you can simply run the {\tt configure} script and the FORTRAN
compiler will be detected and the {\tt Ipopt}, {\tt Bonmin} and {\tt Couenne} projects will be built.

\item{}   If your Fortran 95 compiler cannot deal with the preprocessor statements embedded in the
Mumps\index{Third-party software!Mumps} code, it may be possible to run the Fortran code through a preprocessor such as {\tt cpp}.
In the worst case you may have to resort to manual edits before you can build Ipopt --- or see 
Option~\ref{enumerate:ipopt3}.

\item{} \label{enumerate:ipopt3}
If you have a FORTRAN 77 compiler, you can replace Mumps by one of the Harwell Subroutine Library (HSL)%
\index{Third-party software!HSL} routines {\tt ma27ad.f} or {\tt MA57ad.f}. 
(Unix is case-sensitive, so note the file names carefully.) See

{\tt\UrlHsl}.  

You must obtain the Harwell code and put it in the directory {\tt \../ThirdParty/HSL}.  
Now run the {\tt configure}\index{configure} script as described in Section~\ref{section:unixbuilds}.

Note that the Harwell Subroutine Library is not governed by the Eclipse Public License\index{Eclipse Public License (EPL)}. It is the user's responsibility
to ensure adherence to appropriate copyright and distribution agreements.

\item{} \label{enumerate:ipopt4}
If you do not have a FORTRAN compiler and do not wish to obtain one, you can use the {\tt f2c}\index{f2c@{\tt f2c}}
translator from Netlib to translate HSL to {\tt C}.  The installation instructions for {\tt f2c}
are in the {\tt INSTALL} file in
\begin{verbatim}
BuildTools/compile_f2c
\end{verbatim}

\end{enumerate}

\noindent Two important points:


\begin{itemize}
\item Option~\ref{enumerate:ipopt4} also requires that one of the Harwell Subroutine Library (HSL) routines
{\tt ma27ad.f} or {\tt MA57ad.f} be present in the HSL directory.

\item If you run {\tt configure}\index{configure} with the {\tt --enable-debug} option on Windows, then when building the {\tt vcf2c.lib}, use the command line

\begin{verbatim}
CFLAGS = -MTd -DUSE_CLOCK -DMSDOS -DNO_ONEXIT
\end{verbatim}

\end{itemize}
\index{COIN-OR projects!Ipopt@{\tt Ipopt}|)}\index{COIN-OR projects!Bonmin@{\tt Bonmin}|)}%
\index{COIN-OR projects!Couenne@{\tt Couenne}|)}

\vskip 8pt

\ifipopt
\subsubdivision{Ipopt and Microsoft Visual Studio} \label{section:ipopt-msvs}

We regret that at present we cannot distribute a solution file
that can detect and reliably process the necessary third-party software to
build Ipopt. Users who need Ipopt on a Windows system are advised to download
the binary build as documented in Section~\ref{section:obtainingbinaries}.


\iffalse %------------------------------------------------------------------------
Users of Microsoft Visual Studio without access to a unix-like environment (Cygwin, MinGW or MSYS)
will have to prepare the third-party code after downloading. Since some of this code is written in Fortran,
you also need to obtain the {\tt f2c}\index{f2c@{\tt f2c}|(} Fortran to C translator. The steps are as follows.

\begin{enumerate}


\item{} From netlib, download the file

%\begin{verbatim}
%   http://www.netlib.org/f2c/libf2c.zip
%\end{verbatim}

{\tt\ \ \ \UrlFToCZip}

   and extract it in

\begin{verbatim}
   Ipopt\MSVisualStudio\v8
\end{verbatim}


 which is a folder in the root directory (see Figure~\ref{figure:osprojectrootdir}). Make sure that the files
are extracted into the subfolder {\tt libf2c} directly, instead of the subfolder {\tt libf2c$\tt\backslash$libf2c}.
One file created in this process should be

\begin{verbatim}
   Ipopt\MSVisualStudio\v8\libf2c\makefile.vc
\end{verbatim}


\item{} Open a Command Window (DOS prompt) and go into the directory

\begin{verbatim}
   Ipopt\MSVisualStudio\v8\libf2c\
\end{verbatim}

   Here, type

\begin{verbatim}
   nmake -f makefile.vc all
\end{verbatim}

   (If you see a problem related to the file {\tt comptry.bat}, edit the
   file {\tt makefile.vc} and just delete the line containing the one occurrence of
   '{\tt comptry.bat}'.)

Another possible error is that the system cannot find the header file {\tt unistd.h}.
If this occurs, add

\begin{verbatim}
-DNO_ISATTY
\end{verbatim}

at the end of line~9 of {\tt makefile.vc}.

\item{} Download the executable {\tt f2c.exe} from {\tt\UrlFToCBin}
and put it somewhere into your path
   (e.g., {\tt C:$\backslash$Windows})

\item{} Download the source code for Blas\index{Third-party software!Blas} (from {\tt\UrlBlas}),
Lapack\index{Third-party software!Lapack} (from {\tt\UrlLapack}),
and HSL\index{Third-party software!HSL} (see previous section).
Install each download into the appropriate subdirectory in {\tt ThirdParty}.

\item{} \label{enumerate:ipopt-step5}
In a DOS window, go to the directory

\begin{verbatim}
   Ipopt\MSVisualStudio\v8\libCoinBlas
\end{verbatim}

   and run the batch file

\begin{verbatim}
   convert_blas.bat
\end{verbatim}

   This runs the {\tt f2c} translator and generates new C files.%
\index{f2c@{\tt f2c}|)}


\item{} Repeat step~\ref{enumerate:ipopt-step5} in the directories

\begin{verbatim}
   Ipopt\MSVisualStudio\v8\libCoinLapack

   Ipopt\MSVisualStudio\v8\libCoinHSL
\end{verbatim}

   using the {\tt convert\_*.bat} files you find there.

\item{}
   Download the ASL\index{Third-party software!ASL} code and follow the steps in Section~\ref{section:ASL}.

\item{}
Now you can open the solution file

\begin{verbatim}
   OS\MSVisualStudio\v8\OS.sln
\end{verbatim}

and select the configuration {\tt Release-Plus}\index{Microsoft Visual Studio!{\tt Release-Plus} configuration}.
Open the Configuration Manager\index{Microsoft Visual Studio!Configuration Manager} (in the Build menu)
and set all projects to ``Build''
(by clicking the check-box next to the project name).
Then select Build (or press F7).
This will build all the necessary libraries for the
{\tt OSSolverService}\index{OSSolverService@{\tt OSSolverService}} executable
with the {\tt Ipopt} solver. The solution files for the {\tt Bonmin}\index{COIN-OR projects!Bonmin@{\tt Bonmin}} 
and {\tt Couenne}\index{COIN-OR projects!Couenne@{\tt Couenne}} solvers will be
available in a future release.

A {\tt unitTest}\index{unitTest@{\tt unitTest}},
the {\tt OSAmplClient}\index{OSAmplClient@{\tt OSAmplClient}} (see Section~\ref{section:amplclient})
and all the utility programs in Sections \ref{section:fileupload}
and~\ref{section:examples} are included in the build, as well.
\end{enumerate}

\fi     %------------------------------------ end of \iffalse
\fi     % end of \ifipopt


\subdivision{Other Third-Party Software} \label{section:otherthirdparty}

\index{Third-party software|(}%
This section deals with other third-party software not available for download at \url{www.coin-or.org}.
The OS project distribution includes the COIN-OR projects  {\tt Bonmin}\index{COIN-OR projects!Bonmin@{\tt Bonmin}},
{\tt Cbc}\index{COIN-OR projects!Cbc@{\tt Cbc}}, {\tt Clp}\index{COIN-OR projects!Clp@{\tt Clp}}, {\tt Cgl}\index{COIN-OR projects!Cgl@{\tt Cgl}},
{\tt CoinUtils}\index{COIN-OR projects!CoinUtils@{\tt CoinUtils}}, 
{\tt Couenne}\index{COIN-OR projects!Couenne@{\tt Couenne}}, {\tt CppAD}\index{COIN-OR projects!CppAD@{\tt CppAD}},
{\tt DyLP}\index{COIN-OR projects!DyLP@{\tt DyLP}},   {\tt Ipopt}\index{COIN-OR projects!Ipopt@{\tt Ipopt}},
{\tt Osi}\index{COIN-OR projects!Osi@{\tt Osi}}, {\tt SYMPHONY}\index{COIN-OR projects!SYMPHONY@{\tt SYMPHONY}}, and {\tt Vol}\index{COIN-OR projects!Vol@{\tt Vol}}.
(For details on any of these projects see the COIN-OR web site at {\tt\UrlCoinProjects}.)

However, the project is also designed to work with  several other open source and commercial software projects.
These are not distributed under the Eclipse Public Library and hence they cannot be downloaded
automatically by the system. 

For the open-source packages {\tt ASL}, {\tt Blas}, {\tt Lapack} and {\tt Mumps} --- a minimal set needed to build the Ipopt solver --- there are {\tt get.xxxx} scripts in the 
%In the OS distribution directory structure (see Figure~\ref{figure:osprojectrootdir}), there is a 
{\tt ThirdParty} directory  (see Figure~\ref{figure:osprojectrootdir}), which does not contain anything other than {\tt get.xxxx} scripts and other utilities.
The source code for any of these packages must be downloaded separately using the {\tt get.xxxx} scripts,
as {\tt configure}\index{configure|(} will not build these projects without the source code being present. After the download,
{\tt configure} will recognize the presence of these files in specific locations within the {\tt ThirdParty} folder hierarchy and will configure the makefiles\index{makefile} accordingly.

If the user wants to exclude these projects from the build after they have been downloaded and detected,
a new {\tt configure} is required with instructions to skip them. For instance, if the user experiences problems
with the Fortran\index{Fortran} compiler and its interaction with the system, the following command can be used
to skip all projects that use Fortran code:

\begin{verbatim}
configure COIN_SKIP_PROJECTS="Ipopt Bonmin Couenne ThirdParty/Blas ThirdParty/Lapack \
ThirdParty/Mumps"
\end{verbatim}
\index{COIN_SKIP_PROJECTS@{\tt COIN\_SKIP\_PROJECTS}}

Also in this class are the packages {\tt Metis}, a utility package that can optionally be used by 
{\tt Ipopt} to speed up computations, and {\tt Glpk}, a linear programming solver (see below). 

The last class of  external packages are commercial and closed-source programs, which may require special licenses, purchase agreements, and other  considerations which are strictly between the user and the third-party supplier. No {\tt get.xxxx} scripts are possible for these packages, and consequently it is hard to completely automate the build process. Nonetheless, it is possible to include these solvers, and we give below some indications as to how this might be accomplished.

In the {\tt inc} subdirectory of the {\tt OS}  directory, there is a header file, {\tt config\_os.h} that defines
the values of a number of
\index{COIN_HAS_XXXXX@{\tt COIN\_HAS\_XXXXX}|(}
\begin{verbatim}
COIN_HAS_XXXXX
\end{verbatim}
variables.

Many of the other header files contain {\tt \#include} statements inside {\tt  \#ifdef}  statements. For example,
\begin{verbatim}
#ifdef COIN_HAS_LINDO
#include "LindoSolver.h"
#endif
#ifdef COIN_HAS_GLPK
#include <OsiGlpkSolverInterface.hpp>
#endif
\end{verbatim}

If the project is configured with the simple {\tt ./configure} command given in Step~\ref{itemize:unixbuilds}
on page~\pageref{itemize:unixbuilds} with no arguments, then in the {\tt config\_os.h} header file the variables
associated with the third-party software described in this subsection will be undefined. For example:
\begin{verbatim}
/* Define to 1 if the Cplex package is used */
/* #undef COIN_HAS_CPX */
\end{verbatim}
unlike the configured COIN-OR projects that appear as
\begin{verbatim}
/* Define to 1 if the Clp package is used */
#define COIN_HAS_CLP 1
\end{verbatim}
In the following subsections we  describe how to incorporate various  third-party packages into the OS project
and see to it that the
\begin{verbatim}
COIN_HAS_XXXXX
\end{verbatim}
variable is defined in  {\tt config\_os.h}.
\index{COIN_HAS_XXXXX@{\tt COIN\_HAS\_XXXXX}|)}

\medskip
Make sure to run {\tt configure} after you have downloaded the required
source code, in order to modify the makefiles\index{makefile} appropriately. It is {\bf important to note} that even though there are
multiple files named {\tt configure} in various subdirectories, you should only ever run the master configure in the
distribution root directory, possibly accessed from a {\tt VPATH}\index{VPATH} as in Section~\ref{section:vpath}.
It sets important global variables and will call all other necessary configure files in turn.\index{configure|)}
You may also wish to view

{\small
%\begin{verbatim}
%https://projects.coin-or.org/BuildTools/wiki/user-configure#CommandLineArgumentsforconfigure
%\end{verbatim}
\noindent{\tt\UrlCoinConfigure}
}

\noindent for more information on command line arguments that are illustrated in the subsections below.%
\index{Third-party software|)}


\subsubdivision{AMPL Solver Library (ASL)} \label{section:ASL}

\index{Third-party software!ASL|(}%
The OS library contains a class, {\tt OSnl2osil}\index{OSnl2osil@{\tt OSnl2osil}} (see Section~\ref{section:nl2osil}),
and the program {\tt OSAmplClient}\index{OSAmplClient@{\tt OSAmplClient}} (see Section~\ref{section:amplclient}) that
require the use of the AMPL Solver Library~(ASL). See {\tt\UrlAmpl}  and  {\tt\UrlAmplSandia}.
Users with a Unix\index{Unix} system should locate the {\tt ASL} folder that is part of the distribution.
The {\tt ASL} folder is in the {\tt ThirdParty} folder
which is in the distribution root folder. Locate and execute the {\tt get.ASL} script.  Do this prior to running
the {\tt configure} script\index{configure}. The {\tt configure} script will then build the correct ASL library.

Microsoft  Visual Studio\index{Microsoft Visual Studio} users should note that {\tt OSAmplClient} is distributed 
as part of the binary distribution. For reasons explained in Section~\ref{section:ipopt-msvs} it is currently
not possible to distribute a solution file to let users build their own executable.

\iffalse %-----------------------------------------------------------------------------
Microsoft  Visual Studio\index{Microsoft Visual Studio} users will have to build the ASL library separately and
then link it with the OS library in the OS project file.  The necessary source files are at

%\begin{verbatim}
%http://netlib.sandia.gov/cgi-bin/netlib/netlibfiles.tar?filename=netlib/ampl/solvers
%\end{verbatim}

\noindent{\tt\UrlAmplSolvers}

After unpacking the distribution you will have to create the file
{\tt ThirdParty/ASL/details.c} by hand,
as follows: Copy the file {\tt details.c0} to {\tt details.c} and replace the
line
\begin{verbatim}
char sysdetails_ASL[] = "System_details";
\end{verbatim}
by
\vskip 8pt
\noindent{\tt char sysdetails\_ASL[] = "MS VC++ }$n${\tt .0";}
\vskip 8pt
\noindent
where $n$ is the version number of the {\tt cl} compiler on your system (most
likely 7, 8 or~9).

To avoid linker errors\index{linker errors} in MSVS, you may have to edit the file {\tt fpinitmt.c}.
Specifically, if you see the error ``multiply defined object \_matherr'', you must
hide the definition of {\tt \_matherr} in {\tt fpinitmt.c} and comment out lines 212--225
which read
\begin{verbatim}
 matherr_rettype
matherr( struct _exception *e )
{
	switch(e->type) {
	  case _DOMAIN:
	  case _SING:
		errno = set_errno(EDOM);
		break;
	  case _TLOSS:
	  case _OVERFLOW:
		errno = set_errno(ERANGE);
	  }
	return 0;
	}
\end{verbatim}

Then you must build the source code with the utility {\tt nmake}
which should be part of the Visual Studio distribution. (This can be done in a Command Window.)
The appropriate command is
\begin{verbatim}
nmake -f makefile.vc
\end{verbatim}
This produces the library file {\tt amplsolv.lib}, which is placed in the subfolder
{\tt ThirdParty$\tt\backslash$ASL$\tt\backslash$solvers}.

        
\ifipopt
Before you can use the {\tt Release-Plus}\index{Microsoft Visual Studio!{\tt Release-Plus} configuration} 
configuration in our solution file {\tt OS.sln}\index{OS sln@{\tt OS.sln}},
you must also prepare the source for the solver {\tt Ipopt}\index{COIN-OR projects!Ipopt@{\tt Ipopt}}
(see Section~\ref{section:ipopt-msvs}). If you want to add other third-party software or include debug information,
you may have to modify (or copy) this configuration and tailor it to your needs.
\else
Now you are ready to use MSVS. Both the {\tt Debug}\index{Microsoft Visual Studio!{\tt Debug} configuration} and 
{\tt Release}\index{Microsoft Visual Studio!{\tt Release} configuration} configurations contain two projects, 
{\tt libOSnl2OSiL} and {\tt OSAmplClient}, which use the ASL library and are normally deactivated. 
Activate these projects in the Configuration Manager\index{Microsoft Visual Studio!Configuration Manager} 
(available from the Build menu), then select Build.
%If you want to add other third-party software or include debug information, you may have to modify
%(or copy) this configuration and tailor it to your needs.
\fi
\fi
\index{Third-party software!ASL|)}

\subsubdivision{GLPK}

\index{Third-party software!GLPK|(}%
{\tt GLPK} is an open-source linear and integer-programming solver from the GNU organization. See {\tt\UrlGlpk}. 
GLPK is distributed under the GNU General Public Licence (GPL)\index{GNU General Public Licence (GPL)}, which is 
incompatible with the Eclipse Public License (EPL)\index{Eclipse Public License (EPL)} that governs OS. 
For that reason we are unable to distribute OS binaries linked to the GLPK solver.  
Users interested in GLPK must build OS from source and link to the GLPK libraries.

In order to use GLPK with OS in a unix environment, connect to {\tt
ThirdParty/Glpk} and execute {\tt get.Glpk}. Once the source code has been downloaded, run {\tt configure}, 
followed by a {\tt make}, as explained in Section~\ref{section:unixbuilds} or 
Section~\ref{section:vpath}.

Users on MSVS\index{Microsoft Visual Studio} can download the source by
anonymous {\tt ftp} from
\begin{verbatim}
ftp://ftp.gnu.org/gnu/glpk/glpk-version_number.tar.gz
\end{verbatim}

At the time of this writing, the most up-to-date version is \GlpkVer, which can be found at
%\begin{verbatim}
%ftp://ftp.gnu.org/gnu/glpk/glpk-4.30.tar.gz
%\end{verbatim}
\noindent{\tt\UrlGlpkDownload}
\index{Third-party software!GLPK|)}


\subsubdivision{SoPlex}

\index{Third-party software!SoPlex|(}%
{\tt SoPlex} was developed at the
Konrad-Zuse-Zentrum f\"ur Informationstechnik Berlin
and is available in source code. 
SoPlex is free for academic research and can be licensed for commercial use. 

Because of the licensing arrangement we are not able to provide {\tt get.soplex} scripts. Rather, each prospective user will have to download their own code from 

\noindent{\tt\UrlSoPlexDownload}.

There is no specific location where the source should be installed, but when the {\tt configure} script is run it is important to indicate where the required header and library files are found, for instance: 

\begin{verbatim}
configure --with-soplex-lib="-L$(HOME)/Soplex/soplex-1.7.1/lib -lsoplex"      \
--with-cplex-incdir="$(HOME)/Soplex/soplex-1.7.1/src"
\end{verbatim}
\index{Third-party software!SoPlex|)}%

\subsubdivision{Cplex} \label{section:Cplex}

\index{cplex@{\tt cplex}|(}%
Cplex is a linear, integer, and quadratic solver. See {\tt\UrlCplex}.
Cplex does not provide source code and you can only download the platform dependent binaries.
After installing the binaries and include files in an appropriate directory, run {\tt configure} to point to the
include and library directory. An example is given below:

\begin{verbatim}
configure --with-cplex-lib="-L$(HOME)/Cplex/cplex/lib -lcplex -lilocplex -lm -lpthread"   \
--with-cplex-incdir="$(HOME)/Cplex/cplex/include"
\end{verbatim}

You may also need the following environment variables (if they are not already set). The following  values were used in a working implementation.
\begin{verbatim}
LD_LIBRARY_PATH=$(LD_LIBRARY_PATH):$(HOME)/Cplex/cplex/lib
ILOG_LICENSE_FILE="$(HOME)/Cplex/cplex/access.ilm
%PATH=***:/usr/local/ilog/cplex81/bin/i86_linux2_glibc2.3_gcc3.2:***
%CLASSPATH=:/usr/local/ilog/cplex81/bin/i86_linux2_glibc2.3_gcc3.2:
\end{verbatim}
\index{cplex@{\tt cplex}|)}

\subsubdivision{Gurobi}

\index{Gurobi|(}%
Like {\tt cplex}, Gurobi is a commercial code that solves linear, integer and quadratic programs.
See {\tt\UrlGurobi}. Gurobi needs to be downloaded and installed similarly to {\tt cplex}, using directives on the {\tt configure} command as to where the library and header files are to be found, for example

\begin{verbatim}
configure --with-gurobi-lib="-L$(HOME)/gurobi/lib -lgurobi55 -lpthread -lm"    \
--with-gurobi-incdir="$(HOME)/gurobi/include"
\end{verbatim}

In addition it is necessary to set the environment variables {\tt LD\_LIBRARY\_PATH} and {\tt GRB\_LICENSE\_FILE} to point to the location of the libraries and the license file, e.g.,

\begin{verbatim}
LD_LIBRARY_PATH=${LD_LIBRARY_PATH}:$(HOME)/Cplex/cplex/lib
GRB_LICENSE_FILE='$(HOME)/gurobi/gurobi.lic'
\end{verbatim}

\noindent{\bf Remark.} If both {\tt cplex} (see Section \ref{section:Cplex}) and Gurobi are to be 
linked to OS, file locations for both programs must be added to the {\tt configure} command, as follows:

\begin{verbatim}
configure --with-cplex-lib="-L$(HOME)/Cplex/cplex/lib -lcplex -lilocplex -lm -lpthread""  \
--with-gurobi-lib="-L$(HOME)/gurobi/lib -lgurobi55 -lpthread -lm"    \
--with-cplex-incdir= $(HOME)/Cplex/cplex/include  --with-gurobi-incdir="$(HOME)/gurobi/include" 
\end{verbatim}
  
\index{Gurobi|)}%

\subsubdivision{Mosek}% and Xpress}

\index{Mosek|(}%
%\index{Xpress|(}%

In the same way another commercial LP and IP solver, Mosek (see {\tt\UrlMosek})
% and Xpress (see {\small\tt\UrlXpress})
, can be hooked to OS. The relevant information to add to the {\tt configure} command is

\begin{verbatim}
--with-mosek-lib="-L$(HOME)/$(MOSEK)/bin -lmosek64 -liomp5 -lpthread -lm"    \
--with-mosek-incdir="$(HOME)/$(MOSEK)/include"                          \
\end{verbatim}
%--with-xpress-lib="-L$(HOME)/$(XPRESS)/lib -lgurobi55 -lpthread -lm"    \
%--with-xpress-incdir="$(HOME)/$(XPRESS)/include"                          

with appropriate information for the environment variable {\tt MOSEK} %and {\tt XPRESS} 
to point to the correct file locations for the Mosek %and Xpress 
header files and libraries.
\index{Mosek|)}%
%\index{Xpress|)}%


\ifknitro
\subsubdivision{Knitro}

\index{Knitro|(}%
Knitro is a nonlinear solver. See {\tt\UrlKnitro}.  Ziena does not provide source code for Knitro.  You must download platform dependent binaries.   In order to use Knitro with the OS project, perform the following steps.

\begin{enumerate}[Step 1:]

\item{}  Download {\tt knitro} to the desired directory.

\item{}  Copy the file {\tt nlpProblemDef.h} from the {\tt examples/C++} directory to the {\tt include} directory.

\item{}  Edit the file {\tt nlpProblemDef.h} and delete the following lines:

\begin{verbatim}
NlpProblemDef::~NlpProblemDef (void)
{
    //---- DO NOTHING.
    return;
}
\end{verbatim}

\item{} Run {\tt configure} with appropriate values for  {\tt --with-knitro-lib} and {\tt --with-knitro-incdir}.
For example:

\begin{verbatim}
configure --with-knitro-lib="-L/home/kmartin/files/code/knitro/linux/lib -lknitro "
--with-knitro-incdir=/home/kmartin/files/code/knitro/linux/include
\end{verbatim}

\end{enumerate}
\index{Knitro|)}
\fi

\subsubdivision{LINDO}

\index{LINDO|(}%
LINDO is a commercial linear, integer, and nonlinear solver. See \url{http://www.lindo.com}.
LINDO does not provide source code and you can only download the platform dependent binaries.
After installing the binaries and include files in an appropriate directory, run {\tt configure} to point to the
include and library directory. An example is given below:

\begin{verbatim}
configure --with-lindo-incdir=/home/kmartin/files/code/lindo/linux/include
--with-lindo-lib="-L/home/kmartin/files/code/lindo/linux/lib -llindo -lmosek"
\end{verbatim}
\index{LINDO|)}

\subsubdivision{MATLAB}

\index{MATLAB|(}%
MATLAB is a commercial programing environment especially suited for the development and testing of 
computationally intensive tasks. (See \url{http://www.mathworks.com/products/matlab}.)
Install MATLAB on the client machine and follow the instruction in Section~\ref{section:usingmatlab}.%
\index{MATLAB|)}

\subsubdivision{Library Paths}

After running {\tt configure} as described above,  on Unix systems, it will be necessary to set the
environment variables {\tt LD\_LIBRARY\_PATH} or {\tt DYLD\_LIBRARY\_PATH} (on Mac OS X) to point to the
location of the installed third-party libraries in the case that the libraries are dynamic and not static libraries.


\subdivision{Bug Reporting}

Bug reporting\index{Bug reporting} is done through the project Trac\index{Trac system} page. This is at
%\begin{verbatim}
%http://projects.coin-or.org/OS
%\end{verbatim}

\medskip
\noindent{\tt\UrlTrac}
\medskip

To report a bug, you must be a registered user.  For  instructions on  how to register, go to
%\begin{verbatim}
%http://www.coin-or.org/usingTrac.html
%\end{verbatim}

\medskip
\noindent{\tt\UrlUsingTrac}
\medskip

After registering, log in and then file a trouble ticket by going to
%\begin{verbatim}
%http://projects.coin-or.org/OS/newticket
%\end{verbatim}

\medskip
\noindent{\tt\UrlNewticket}
\medskip


\subdivision{Documentation}\label{section:documentation}

\index{Doxygen|(}%
If you have Doxygen  (\url{http://www.doxygen.org}) available (the executable {\tt doxygen} should be in the {\tt path} command) 
then executing
\begin{verbatim}
make doxydoc
\end{verbatim}
in the project root directory will result in the Doxygen documentation being generated and stored in the {\tt doxydoc} 
folder in the project root.

In order to view the documentation, open a browser and open the file
\begin{verbatim}
projectroot/doxydoc/html/index.html
\end{verbatim}

By default, running Doxygen will generate documentation for only the  OS project.  Documentation will not be generated 
for the other COIN-OR projects in the project root. In the {\tt doxydoc}  folder is a configuration file 
{\tt doxygen.conf}.  This configuration file contains the {\tt EXCLUDE} parameter

\begin{verbatim}
EXCLUDE =  Bonmin \
   Cbc\
   Cgl \
   Clp \
   CoinUtils \
   Couenne \
   cppad \
   SYMPHONY \
   Vol \
   DyLP \
   ThirdParty \
   Osi \
   include
\end{verbatim}

This file can be edited, and any project for which documentation is desired, can be deleted from the {\tt EXCLUDE} list.%
\index{Doxygen|)}





\subdivision{Platforms}

The build process described in Section~\ref{section:unixbuilds} has been tested on Linux\index{Linux}\index{Unix},
Mac OS X\index{Mac OS X}, and on Windows using  MinGW/MSYS\index{MinGW}\index{MSYS} and Cygwin\index{Cygwin}.
The  {\tt gcc}/{\tt g++} and Microsoft {\tt cl} compiler have been tested.
A number of solvers have also been tested with the OS library. For a list of tested solvers and platforms see
Table~\ref{table:testedplatforms}.  More detail on the platforms listed in Table~\ref{table:testedplatforms}
is given in Table~\ref{table:platformdescription}.  For a list of other  platforms testing the OS project see 

\medskip
\noindent{\tt\UrlNightlyBuild}.
\medskip

\begin{table}
\caption{Tested Platforms for Solvers}
\centering
\label{table:testedplatforms}
\vskip 8pt
 \begin{tabular}{l|c|c|c|c|c|c|}
 &Mac&Linux&Cyg-gcc&Msys-cl&MinGW-gcc&MSVS \\ \hline
Bonmin       &x&x&x&x&x&x \\ \hline
Cbc          &x&x&x&x&x&x \\ \hline
Cgl          &x&x&x&x&x&x \\ \hline
Clp          &x&x&x&x&x&x \\ \hline
Couenne      &x&x& &x&x&  \\ \hline
Cplex        & &x& & & &  \\ \hline
DyLP         &x&x&x&x&x&x \\ \hline
Glpk         &x&x&x&x&x&  \\ \hline
Ipopt        &x&x&x&x&x&x \\ \hline
\ifknitro
Knitro       &x&x& & & &  \\ \hline
\fi
Lindo        &x&x& &x& &x \\ \hline
MATLAB       &x& & & & &  \\ \hline
OSAmplClient &x&x& &x& &x \\ \hline
SYMPHONY     &x&x&x&x&x&x \\ \hline
Vol          &x&x&x&x&x&x \\ \hline
\end{tabular}
\end{table}


 \begin{table}
\caption{Platform Description}
\centering
\label{table:platformdescription}
\vskip 8pt
 \begin{tabular}{l|c|c|c|}
 & {\bf Operating System} & {\bf Compiler} & {\bf  Hardware} \\ \hline
 Mac &Mac OS X 10.4.9&gcc 4.0.1&Power PC \\   \hline
  Mac &Mac OS X 10.4.10&gcc 4.0.1&Intel \\   \hline
 Linux &Ubuntu  7.10 &gcc 4.1.2& Dell Intel 32 bit chip\\ \hline
 Cyg-gcc &Windows 2003 Server&gcc 4.2.2& Dell Intel 32 bit chip \\ \hline
 Msys-cl &Windows XP&cl 14.00 &Dell Intel 32 bit chip \\ \hline
 MinGW-gcc &Windows XP&gcc 3.4.2&Dell Intel 32 bit chip \\ \hline
 MSVS &Windows XP&Visual Studio 8 and 9 &Dell Intel 32 bit chip \\ \hline
\end{tabular}
\end{table}


\throwpage


\section{The OS Project Components}\label{section:projectcomponents}

The directories in the  project root  are outlined in Figure~\ref{figure:osprojectrootdir}.

If you download the OS package, you get these additional COIN-OR projects. The links to the project home pages are provided below and give more information on these projects.
\begin{itemize}
\item {\tt Bonmin}\index{COIN-OR projects!Bonmin@{\tt Bonmin}} - {\tt\UrlBonmin}
\item {\tt BuildTools}\index{COIN-OR projects!BuildTools@{\tt BuildTools}} - {\tt\UrlBuildtools}
\item {\tt Cbc}\index{COIN-OR projects!Cbc@{\tt Cbc}} - {\tt\UrlCbc}
\item {\tt Cgl}\index{COIN-OR projects!Cgl@{\tt Cgl}} - {\tt\UrlCgl}
\item {\tt Clp}\index{COIN-OR projects!Clp@{\tt Clp}} - {\tt\UrlClp}
\item {\tt CoinUtils}\index{COIN-OR projects!CoinUtils@{\tt CoinUtils}} - {\tt\UrlCoinUtils}
\item {\tt Couenne}\index{COIN-OR projects!Couenne@{\tt Couenne}} - {\tt\UrlCouenne}
\item {\tt CppAD}\index{COIN-OR projects!CppAD@{\tt CppAD}} - {\tt\UrlCppad}
\item {\tt DyLP}\index{COIN-OR projects!DyLP@{\tt DyLP}}  - {\tt\UrlDylp}
\item {\tt Ipopt}\index{COIN-OR projects!Ipopt@{\tt Ipopt}} - {\tt\UrlIpopt}
\item {\tt Osi}\index{COIN-OR projects!Osi@{\tt Osi}} - {\tt\UrlOsi}
\item {\tt SYMPHONY}\index{COIN-OR projects!SYMPHONY@{\tt SYMPHONY}}   - {\tt\UrlSymphony}
\item {\tt Vol}\index{COIN-OR projects!Vol@{\tt Vol}} - {\tt\UrlVol}
\end{itemize}

The following directories are also in the project root.
\begin{itemize}
\item {\tt bin} - after executing {\tt make install}\index{make install@{\tt make install}}
the bin directory will contain {\tt OSSolverService}\index{OSSolverService@{\tt OSSolverService}}, {\tt clp}, {\tt cbc}  
and {\tt symphony}.

\item {\tt Data} - this directory contains numerous test problems that are used by the unit tests of
the COIN-OR projects just mentioned.

\item {\tt doxydoc} - is a folder for documentation.

\item {\tt include} - is a directory for header files. If the user wishes to write code to link against any of the libraries in the {\tt lib} directory, it may be necessary to include these header files.

\item {\tt lib} - is a directory of libraries. After running {\tt make install} the OS library along with all other COIN-OR libraries are installed in {\tt lib}.

\item {\tt ThirdParty} - is a  directory for third-party software. For example, if AMPL\index{AMPL} related software
such as {\tt OSAmplClient}\index{OSAmplClient@{\tt OSAmplClient}} is used, then certain AMPL libraries need to be present.
This should go into the {\tt ASL} directory in {\tt ThirdParty.}
\end{itemize}


The directories in the OS directory are outlined in Figure~\ref{figure:osdirectory}.  The OS directories include the following:


\begin{figure}
\centering
%\includegraphics[scale=0.8]{\figurepath/OSDirectory.png}
\includegraphics[scale=0.8]{./figures/OSDirectory.png}
\caption{The OS directory.}
\label{figure:osdirectory}
\end{figure}



\begin{itemize}


\item {\tt applications} - is a directory that holds  the 
{\tt OSAmplClient}\index{OSAmplClient@{\tt OSAmplClient}}
and {\tt OSFileUpload}  applications in subdirectories called, respectively, {\tt amplClient} and {\tt fileUpload}.
See Section \ref{section:amplclient} and~\ref{section:fileupload}.

\item {\tt data} - is a directory that holds test problems. These test problems are used by the
{\tt unitTest}\index{unitTest@{\tt unitTest}} of the OS Project. Many of these files are also used to illustrate
how the {\tt OSSolverService}\index{OSSolverService@{\tt OSSolverService}} works.
See Section~\ref{section:ossolverservice}.

\item {\tt doc} - is the directory with documentation, including this {\it OS User's Manual}.

\item {\tt examples} - is a directory with code examples that illustrate various aspects of the OS project.
These are described in Section~\ref{section:examples}.

\item {\tt inc} - is the directory with the config\_os.h file which has information about which projects
are included in the distribution.

\item {\tt m4} - is a directory that  contains macro scripts written in the m4 language for auto configuration.

\item {\tt MSVisualStudio} - is a directory that  contains folders for the solution files for the
Microsoft Visual Studio\index{Microsoft Visual Studio} IDE.  The subdirectories are organized by the version
of Visual Studio. We currently provide solution files for versions 8 and~9.

\item {\tt schemas} - is the directory that contains the W3C XSD (see \url{www.w3.org}) schemas that are
behind the OS standards. These are described in more detail in Section~\ref{section:schemadescriptions}.

\item {\tt src} - is the directory with all of the source code for the OS Library and for the executable
{\tt OSSolverService}. The OS Library components are described in Section~\ref{section:oslibrary}.

\item {\tt stylesheets} - this directory contains the XSLT stylesheet that is used to transform the solution
instance in OSrL format into HTML so that it can be displayed in a browser.

\item {\tt test} - this directory contains the {\tt unitTest}.


\item  {\tt wsdl} - is a directory of WSDL (Web Services Discovery Language) files. These are used to specify
the inputs and outputs for the methods and other invocation details provided by a Web service. The most relevant
file for the current version of the OS project is {\tt OShL.wsdl}.
This describes the set of inputs and outputs for the methods implemented in the {\tt OSSolverService}.
See Section~\ref{section:ossolverservice}.

\end{itemize}



\throwpage

\division{Modifying the project}\label{section:ModifyingProject} 

There are several caveats about modifications to the code. This section collects together the different situations that may arise and how to deal with them properly.

\begin{enumerate}

\item Simple edits to code or data.

Commit such changes only after the project builds correctly and passes the complete unit test 
({\tt ./unitTest nb}) on both unix and Windows. It is a good idea to do an update just prior to a commit. This keeps conflicts local in case there have been changes by other folks and allows consolidation before a disaster occurs. 

\item Adding source files or data.

In addition to running the unit test it is necessary to update the configure and make files. 
If source files are involved, this requires changing the {\tt Makefile.am} file in the subdirectory 
containing the code. (Make sure to list header files in two separate places.) Data files must be listed in 
{\tt configure.ac} in order to be copied correctly when executing a vpath build. 


{\bf Note:} After any changes to the file structure it is necessary to create new configure and make files by running {\tt BuildTools}:
\begin{verbatim}
BuildTools/run_autotools OS
\end{verbatim}

Make sure you have the correct version of {\tt BuildTools} installed. (You may have to download proper versions from the internet.)
See {\tt https://projects.coin-or.org/BuildTools/wiki/pm-main} and  {\tt https://projects.coin-or.org/BuildTools/wiki/pm-get-autotools}.

Modify the {\tt configure.ac} file in unix only, because Windows uses different line endings that the 
{\tt BuildTools} software cannot handle properly. 


\medskip

After you have run {\tt BuildTools} you should do a completely new {\tt configure} and {\tt make}, followed by a unit test. Commit only after the unit test succeeds. (Don't forget to commit both changed code/data files and the modified config and make files.)

\item Making changes to externals.

To add, delete or modify an external project, edit the {\tt Dependencies} file in the top COIN directory.
After that it is necessary to set the {\tt svn:externals} property, as follows:

\begin{verbatim}
svn propset svn:externals -F Dependencies .
svn propget svn:externals .
\end{verbatim}

Do an {\tt svn update} to verify that the externals have been set properly.
This is followed by a  {\tt configure} and {\tt make}, and then a unit test. 

\item Changing the documentation.

This is easier, but make sure that pdflatex works and do not forget to commit both the source and pdf file. 

\end{enumerate}

\subdivision{Third-party contributions}\label{section:significantChanges}

Simple bug fixes are not an issue, but when a third party contributes significant portions of code, the issue of copyright arises and must be dealt with. The following are FSF guidelines taken from 
{\tt http://www.gnu.org/prep/maintain/html\_node/Legally-Significant.html}.

If a person contributes more than around 15 lines of code and/or text that is legally significant for copyright purposes, we need copyright papers for that contribution, as described above.
 
A change of just a few lines (less than 15 or so) is not legally significant for copyright. A regular series of repeated changes, such as renaming a symbol, is not legally significant even if the symbol has to be renamed in many places. Keep in mind, however, that a series of minor changes by the same person can add up to a significant contribution. What counts is the total contribution of the person; it is irrelevant which parts of it were contributed when.
 
Copyright does not cover ideas. If someone contributes ideas but no text, these ideas may be morally significant as contributions, and worth giving credit for, but they are not significant for copyright purposes. Likewise, bug reports do not count for copyright purposes.
 
When giving credit to people whose contributions are not legally significant for copyright purposes, be careful to make that fact clear. The credit should clearly say they did not contribute significant code or text.
 
When people’s contributions are not legally significant because they did not write code, do this by stating clearly what their contribution was. For instance, you could write this:
 
\begin{verbatim}
/*
 * Ideas by:
 *   Richard Mlynarik <mly@adoc.xerox.com> (1997)
 *   Masatake Yamato <masata-y@is.aist-nara.ac.jp> (1999)
 */
\end{verbatim}

{\tt Ideas by:} makes it clear that Mlynarik and Yamato here contributed only ideas, not code. 
Without the {\tt Ideas by:} note, several years from now we would find it hard to be sure whether they had contributed code, and we might have to track them down and ask them.
 
When you record a small patch in a change log file, first search for previous changes by the same person, and see if per past contributions, plus the new one, add up to something legally significant. If so, you should get copyright papers for all per changes before you install the new change.
 
If that is not so, you can install the small patch. Write `(tiny change)' after the patch author's name, like this:
 
\begin{verbatim}
2002-11-04  Robert Fenk  <Robert.Fenk@gmx.de>  (tiny change)
\end{verbatim}

\medskip

Some further ideas as well as a copy of the Contributor's Statement of Respect for Ownership (CSRO)
can be found at %{\tt http://www.coin-or.org/contributions.html#contributions}.
\begin{verbatim}
http://www.coin-or.org/contributions.html#contributions
\end{verbatim}

\throwpage



\section{Setting up a Solver Service with Apache Tomcat}\label{section:tomcat}

\index{Apache Tomcat|(}
This section explains how to download and use the Java implementation of the
remote solver service described in Section~\ref{section:servicemethods}.
The server side of the Java distribution is based on the Tomcat~6.0
implementation (we have tested releases tomcat-6.0.26 or tomcat-6.0.33).   In order to build an OS Solver Service, the user should do an
svn checkout:

\begin{verbatim}
svn co https://projects.coin-or.org/svn/OS/branches/OSjava OSjava
\end{verbatim}

The {\tt OSjava} folder contains the file {\tt INSTALL.txt}. Please follow the
instructions in {\tt  INSTALL.txt} under the heading:
\begin{verbatim}
== Install An OS Web Server==
\end{verbatim}

Installing the OS Web Server based on the instructions in {\tt INSTALL.txt}
assumes that the user has installed:

\begin{itemize}
  \item Eclipse IDE.  See  \url{http://www.eclipse.org/downloads/}.  The instructions in
  INSTALL.txt were tested using {\tt Eclipse Galileo}.  
  
  \item An {\tt OSSolverService} that is compatible  with the server platform.  
  The {\tt OSSolverService} executable for several different platforms is
  available at \url{http://www.coin-or.org/download/binary/OS/OSSolverService/}. 
  The user can also build the executable as described in this Manual.  See
  Section \ref{section:build}.
  
  \item Tomcat 6.0. See \url{http://tomcat.apache.org/}.
  
  \item We assume the Java virtual machine 1.6, but this procedure has also been tested, and does work with JVM 1.5.
\end{itemize}

After the final installation is complete on the server we recommend testing  by
doing someting like the following. On a client machine, create  the file {\tt
testremote.config} with the following lines of text
\begin{verbatim}
serviceLocation http://***.***.***.***:8080/OSServer/services/OSSolverService
osil /parincLinear.osil
\end{verbatim}
where {\tt ***.***.***.***} is the IP address of the Tomcat server machine. Then, assuming the files
{\tt testremote.config} and {\tt parincLinear.osil} are in the same directory on the client machine as the
{\tt OSSolverService} execute:
\begin{verbatim}
./OSSolverService config testremote.config
\end{verbatim}
You should get back an OSrL result printed to the screen.




\vskip 8pt
In the following discussion, we assume that the root folder for Tomcat running
on the server is named {\tt tomcat}. \vskip 8pt

If you already have a Tomcat 6.0 server with Axis installed, and have created an
{\tt OSServer.war} file based on {\tt INSTALL.txt}, do the following:
\begin{enumerate}
\item{} copy the file {\tt OSServer.war} into the Tomcat {\tt tomcat/webapps}
directory.

\item{}  Stop and start Tomcat.
\end{enumerate}

In the directory,
\begin{verbatim}
tomcat/webapps/OSServer/WEB-INF/code/OSConfig
\end{verbatim}
there is a configuration file {\tt OSParameter.xml} that can be modified to fit individual user needs. 
You can configure such parameters as service name, service URL/URI. 
Refer to the xml file for more detail. Descriptions for all the parameters are within the file itself.

\vskip 8pt

Below is a summary of the common and important directories
 and files you may want to know.

\begin{itemize}
\item
\begin{verbatim}
tomcat/webapps/OSServer/
\end{verbatim}
contains the OS Solver Service Web application. All directories and files
outside of this folder are Tomcat server related.
\item
\begin{verbatim}
tomcat/webapps/OSServer/WEB-INF
\end{verbatim}
contains private and important configuration, library, class and executable files to run the Optimization Service.

\item
\begin{verbatim}
tomcat/webapps/OSServer/WEB-INF/code/OSConfig
\end{verbatim}
contains configuration files for Optimization Services, such as the {\tt OSParameter.xml} file.
\item
\begin{verbatim}
tomcat/webapps/OSServer/WEB-INF/code/temp
\end{verbatim}
contains temporarily saved files such as submitted OSiL/OSoL input files, and OSrL output files. This folder can get bigger as the service starts to run more jobs. For maintenance purpose, you may want to keep an eye on it.
\item
\begin{verbatim}
tomcat/webapps/OSServer/WEB-INF/code/log
\end{verbatim}
contains log files from the running services in the current Web application.
\item
\begin{verbatim}
tomcat/webapps/OSServer/WEB-INF/classes
\end{verbatim}
contains solver binaries that actually carry out the optimization process.
\item
\begin{verbatim}
tomcat/webapps/OSServer/WEB-INF/code/backup
\end{verbatim}
contains backup files from some of the above directories. This folder can get bigger as the service starts to run more jobs.
\item
\begin{verbatim}
tomcat/webapps/OSServer/WEB-INF/classes
\end{verbatim}
contains class files to run the Optimization Services.
\item
\begin{verbatim}
tomcat/webapps/OSServer/WEB-INF/lib
\end{verbatim}
contains library files needed by the Optimization Services.
\item
\begin{verbatim}
tomcat/conf
\end{verbatim}
contains configuration files for the Tomcat server, such as http server port.
\item
\begin{verbatim}
tomcat/bin
\end{verbatim}
contains executables and scripts to start and shut down the Tomcat server.
\end{itemize}
\index{Apache Tomcat|)}

Before trying to call the OSSolverService, make sure you have the following libraries installed in
webapps/OSServer/WEB-INF/lib.  Not having all of these libraries is one of the most common errors. 

\begin{verbatim}
OSCommon.jar
OSThirdParty.jar
axis.jar
commons-codec-1.5.jar
commons-discovery-0.4.jar
commons-email-1.2.jar
commons-fileupload-1.2.2.jar
commons-logging-1.1.1.jar
fastutil-5.1.5.jar
jaxrpc.jar
log4j-1.2.16.jar
saxon9-xpath.jar
saxon9.jar
wsdl4j-1.5.1.jar
xercesImpl.jar
xml-apis.jar
\end{verbatim}
Also,  if you are running the 1.5 JVM instead of 1.6 you need
the {\tt saaj.jar}. 




\throwpage

\division{File Upload:  Using a File Upload Package}\label{section:fileupload}

\index{OSFileUpload@{\tt OSFileUpload}|(}
When the {\tt OSAgent}\index{OSAgent@{\tt OSAgent}}  class methods {\tt solve}\index{solve@{\tt solve}} and
{\tt send}\index{send@{\tt send}} are used, the problem instance in OSiL\index{OSiL} format is packaged into
a SOAP\index{SOAP protocol} envelope and communication with the server is done using Web Services (for example Tomcat
Axis)\index{Apache Tomcat}. However, packing an XML file into a SOAP envelope may add considerably to the
size of the file (e.g., each {\tt $<$} is replaced with {\tt \&lt;}  and each {\tt $>$} is replaced with {\tt \&gt;}).
Also, communicating with a Web Services servlet can further slow down the communication process.
This could be a problem for large instances. An alternative approach is to use the {\tt OSFileUpload}
executable on the client end and the Java servlet {\tt OSFileUpload} on the server end.  The {\tt OSFileUpload}
client executable is contained in the {\tt fileUpload}  directory inside the {\tt applications} directory.

This servlet is based upon the Apache Commons FileUpload. See 

\medskip
\noindent{\tt\UrlApacheFileupload}
\medskip

The {\tt OSFileUpload} Java class, {\tt OSFileUpload.class} is in the directory
\begin{verbatim}
webapps\os\WEB-INF\classes\org\optimizationservices\oscommon\util
\end{verbatim}
relative to the Web server root.  The source code {\tt OSFileUpload.class} is in the directory
\begin{verbatim}
COIN-OS/OS/applications/fileUpload
\end{verbatim}

Before you can use {\tt OSFileUpload}, you must give a valid URL for the location of the server.
This information must be provided in line 82 of the source code {\tt OSFileUpload.cpp} before 
issuing the {\tt make}\index{make@{\tt make}} command (in a unix environment) or the build (under MS VisualStudio)\index{Microsoft Visual Studio}.

The {\tt OSFileUpload} client executable (see {\tt OS/applications/fileUpload}) takes one argument on the command line,
which is the location of the file on the local directory to upload to the server. For example,
\begin{verbatim}
OSFileUpload ../../data/osilFiles/parincQuadratic.osil
\end{verbatim}
The {\tt OSFileUpload} executable first creates an {\tt OSAgent} object.
\begin{verbatim}
OSSolverAgent* osagent = NULL;
osagent = new OSSolverAgent("http://kipp.chicagobooth.edu/fileupload/servlet/OSFileUpload");
\end{verbatim}
The {\tt OSAgent}  has a method {\tt OSFileUpload} with the signature
\begin{verbatim}
std::string OSFileUpload(std::string osilFileName, std::string osil);
\end{verbatim}
where {\tt osilFileName} is  the name of the OSiL problem instance to be written on the server and {\tt osil}
is the string with the actual instance. Then
\begin{verbatim}
osagent->OSFileUpload(osilFileName, osil);
\end{verbatim}
will place a call to the server, upload the problem instance in the {\tt osil} string, and cause the server
to write on its hard drive a file named {\tt osilFileName}. In our implementation, the uploaded file
({\tt parincQuadratic.osil}) is saved to the {\tt/home/kmartin/temp/parincQuadratic.osil} on the server hard drive.
This location is used in the {\tt osol} file as shown below.

Once the file is on the server, invoke the local {\tt OSSolverService} by
\begin{verbatim}
./OSSolverService config ../data/configFiles/testremote.config
\end{verbatim}
where the {\tt config} file is as follows. Notice there is no {\tt osil}  option as the OSiL file has already
been uploaded and its instance location (``local'' to the server) is specified in the {\tt osol} file.
\begin{verbatim}
osol ../data/osolFiles/remoteSolve2.osol
serviceLocation http://74.94.100.129:8080/OSServer/services/OSSolverService
serviceMethod solve
\end{verbatim}
and the {\tt osol} file is
\begin{verbatim}
<osol xmlns="os.optimizationservices.org"
      xmlns:xsi="http://www.w3.org/2001/XMLSchema-instance"
      xsi:schemaLocation="os.optimizationservices.org
      http://www.optimizationservices.org/schemas/2.0/OSiL.xsd">
    <general>
         <instanceLocation locationType="local">
             /home/kmartin/temp/parincQuadratic.osil
         </instanceLocation>
        <solverToInvoke>ipopt</solverToInvoke>      
    </general>
</osol>
\end{verbatim}

\iffalse   %this needs more work...
As an alternative to using the command line executable {\tt OSFileUpload}, there is also an html form
{\tt fileupload.html} that can be used to upload files. For example, the URL
%\begin{verbatim}
%http://gsbkip.chicagogsb.edu/os/fileupload.html
%\end{verbatim}

\medskip
\noindent{\tt\UrlKippFileupload}
\medskip

\noindent will bring up the necessary form that allows the user to browse a directory and select the file to upload.
This URL is based on the assumption that the {\tt OSJava} classes were deployed as described in
Section~\ref{section:tomcat}. The file {\tt fileupload.html} is in the directory {\tt WebApps/os}.
In our html form implementation, after you upload the OSiL\index{OSiL} file, it shows you the path of the
uploaded file that is saved on the server, so that you can put it in the corresponding {\tt osol} file.
\fi
\index{OSFileUpload@{\tt OSFileUpload}|)}


%\iffalse %------------------------------------------------------



% Part 4: end items: future plans, bibliography, etc.
\throwpage

\division{OS Release Procedure}\label{section:ReleaseProcedure} 

\subdivision{Preparing the release}\label{section:ReleasePrep}

There are four scripts in the {\tt BuildTools} directory that help with the release, namely

\begin{verbatim}
prepare_new_stable
commit_new_stable
prepare_new_release
commit_new_release
\end{verbatim}

\medskip

The first of these, {\tt prepare\_new\_stable}, prepares a new stable version. The simplest form of invoking it is

\begin{verbatim}
($BUILDTOOLSDIR)/prepare_new_stable <loc>
\end{verbatim}

where {\tt loc} is the location that forms the basis of the new stable version, e.g., {\tt OS/trunk}.
{\tt prepare\_new\_stable} uses svn to check out the code, but it locates the latest stable versions of all the dependencies instead of the trunk versions.

Some command line options (from comments in the script):

\begin{description}
  \item[\tt -p]			Suppress checkout (useful for testing)
  \item[\tt -m] 		Bump the major version number.
\end{description}

This script will do the following:

\begin{itemize}
  \item Set the new stable version number as the next minor version number in
    the current major version number. Use the -m flag to bump the major
    version number.

  \item Convert externals from trunk to the top stable branch. Externals which
    are currently stable or release are left untouched. Use -t to suppress
    the change from trunk to stable. Set\_externals is then invoked to set
    release externals where available.

  \item Check out externals. The BuildTools version used by externals (if any)
    is checked, and the script issues a warning if it doesn't match the
    version used by the source URL.

 \item Run the scripts to download any ThirdParty code.

  \item Run {\tt run\_autotools} to rebuild configure and make files.

  \item Run configure, make, and make test

  \item Tweak the externals to upgrade trunk  dependencies to stable.
\end{itemize}

If there is any error during these tasks the script will stop and you should
examine the output.
   
\medskip

The script {\tt commit\_new\_stable} commits the new stable revision. It should be run only after 
{\tt prepare\_new\_stable} has been processed correctly.  

The simplest form of invoking it is

\begin{verbatim}
($BUILDTOOLSDIR)/commit_new_stable <loc>
\end{verbatim}

\noindent
where {\tt loc} is the location that forms the basis of the new stable version, e.g., {\tt OS/trunk}. In this form, it only performs a ``dry run'', that is, it only prints the commands that are to be executed, without actually doing anything. Once satisfied, use the command line parameter -c to do the commit.

{\tt prepare\_new\_release} and {\tt commit\_new\_release} are the corresponding scripts to prepare and commit releases (which are based on stable versions).

\subsubdivision{Version numbers}{\label{section:VersionNumbering}

The version number is defined in the file {\tt configure.ac}. There are actually two versions of this file,
in the root directory as well as the top OS directory. Each of them contains a call to {\tt AC\_INIT}, 
in which the second parameter is the version number. Change this number appropriately as follows:
For trunk versions, use {\bf trunk}, so that {\tt OSgetVersionNumber()} can print the correct svn revision number in the header of the output; for stable and release versions use the proper number ({\tt n.n} for stable versions, {\tt n.n.n} for releases).

\subdivision{Release testing}\label{section:ReleaseTesting}

\begin{itemize}

\item[1.] Run the {\bf nightlyBuild.py} script.

\item[2.] Test the examples.  They are in {\bf OS/examples}.  Do a {\bf make install} before running these.

\begin{itemize}

\item[a.]  Connect to the {\bf algorithmicDiff} folder, build and run {\bf OSAlgorithmicDiffTest.cpp}.  This takes no arguments.  This will test a bunch of the AD routines.



\item[b.]  Connect to the {\bf instanceGenerator} folder, build and run {\bf OSInstanceGenerator.cpp}.  This takes no arguments.

\item[c.]  Connect to the {\bf osTestCode} folder, build and run {\bf OSTestCode.cpp}.  This takes a single argument which is the location of any OSiL file.


\end{itemize}

\item[3.] Test the applications.  They are in {\bf OS/applications}.

\begin{itemize}
 
 
 \item[a.]  Test {\bf OSAmplClient}.  This is not a stand-alone application and is designed to be called from  {\bf ampl}.   Probably the easiest way to test this is to test the {\bf OSAmplClient}  that gets installed in the {\bf bin} directory as a result of {\bf make install}. To make life easy, temporarily copy your {\bf ampl} executable into this {\bf bin} directory.   Also copy the test problem {\bf hs71.nl}  from {\bf OS/data/amplFiles/} into the {\bf bin} directory.   Do five tests. Three local and two remote.
 
 
 \vskip 10pt
 
 
 {\bf Test 1:}  Inside  {\bf ampl} execute the following
 


\begin{verbatim}
model hs71.mod;
option solver OSAmplClient;
option OSAmplClient_options "solver xyz";
solve;
\end{verbatim}

The result should be an error saying:
\begin{verbatim}
<message>a supported solver has not been selected</message>
\end{verbatim}
 
 \vskip 10pt
 
 {\bf Test 2:}  Inside  {\bf ampl} execute the following
 


\begin{verbatim}
model hs71.mod;
option solver OSAmplClient;
option OSAmplClient_options "solver ipopt";
solve;
display x1;
\end{verbatim}

The result of {\bf display x3} should be 3.82115. 


 \vskip 10pt
 
 {\bf Test 3:}  Inside  {\bf ampl} execute the following
 


\begin{verbatim}
model hs71.mod;
option solver OSAmplClient;
option OSAmplClient_options "solver cbc";
solve;
\end{verbatim}

You should get an error message saying:
\begin{verbatim}
<message>Cbc cannot do nonlinear or quadratic</message>
\end{verbatim}


\vskip 10pt

{\small
{\bf Test 4:}  Inside  {\bf ampl} execute the following
\begin{verbatim}
model hs71.mod;
option solver OSAmplClient;
option OSAmplClient_options "solver ipopt";
option ipopt_options "service http://74.94.100.129:808/OSServer/Services/OSSolverService";
solve;
display x1;
\end{verbatim}
}%end small

The result of {\bf display x3} should be 3.82115. 


\vskip 10pt

{\small
{\bf Test 5:}  Inside  {\bf ampl} execute the following
\begin{verbatim}
model hs71.mod;
option solver OSAmplClient;
option OSAmplClient_options "solver clp";
option clp_options "service  http://74.94.100.129:808/OSServer/Services/OSSolverService";
solve;
display x3;
\end{verbatim}
}%end small

You should get an error message saying"
\begin{verbatim}
<message>Clp cannot do nonlinear or quadratic or integer</message>
\end{verbatim}



There is command script, {\bf testAmpl.run} in the directory {\bf OS/data/amplFiles} that contains the commands for all of these test. Simply start {\bf ampl} and execute
\begin{verbatim}
include testAmpl.run;
\end{verbatim}



\item[b.] Test the {\bf OSFileUpload} application.    Edit  {\bf OSFileUpload.cpp}. First comment out line 79 and then modify line 
\begin{verbatim}
osagent = new OSSolverAgent("http://******/os/servlet/OSFileUpload");
\end{verbatim}
to
{\small
\begin{verbatim}
osagent = new OSSolverAgent("http://gsbkip.chicagogsb.edu/os/servlet/OSFileUpload");
\end{verbatim}
}
Rebuild and run. This application takes one command line argument which is the file to be uploaded. 

\end{itemize}

\item[4.] Test the {\bf OSSolverService}.

\begin{itemize}
\item[a.] Test running a local solver.  (These examples assume that the {\bf OS/data} directory is one level above the directory  in which {\bf OSSolverService} is running. Test  for OSiL, mps, and nl files.

\begin{verbatim}
OSSolverService -config ../data/configFiles/testLocal.config
OSSolverService -config ../data/configFiles/testLocalMPS.config
OSSolverService -config ../data/configFiles/testLocalNL.config
\end{verbatim}


You should get the OSrL for the simple test problem.   In all of these look for {\tt <obj idx="-1">-7667.94</obj>} in the MPS test and {\tt <obj idx="-1">-7667.94</obj>} in the other two.


\item[b.]  Test the service methods on the remote server.  


\noindent{\bf Step 1:} Test remote {\bf solve()} method for OSiL, mps, and nl files.



\begin{verbatim}
OSSolverService -config ../data/configFiles/testRemote.config
OSSolverService -config ../data/configFiles/testRemoteMPS.config
OSSolverService -config ../data/configFiles/testRemoteNL.config
\end{verbatim}


You should get the OSrL for the simple test problem in each case.  In all of these look for {\tt <obj idx="-1">-7667.94</obj>}.

\vskip 10pt


\noindent{\bf Step 2:} Test remote {\bf getJobID()} method.


\begin{verbatim}
OSSolverService  -config ../data/configFiles/testRemotegetJobID.config
\end{verbatim}



You will get a long jobID.


\vskip 10pt


\noindent{\bf Step 3:} Test remote {\bf send()} method. Use the {\bf send()} method with the jobID just  generated.  To do this open the file
\begin{verbatim}
/data/osolFiles/sendWithJobID.osol
\end{verbatim}
and replace the existing jobID with the one just generated.  Then run
\begin{verbatim}
OSSolverService  -config ../data/configFiles/testRemoteSend.config
\end{verbatim}
The result should be ``send is true.''

\vskip 10pt

\noindent{\bf Step 4:} Test remote {\bf knock()} method.  See if the  job is complete.


\begin{verbatim}
OSSolverService  -config ../data/configFiles/testRemoteKnock.config
\end{verbatim}

You do not need to put in jobID information. The knock will get the status of all jobs. However, if you want just the status of the job you submitted put your jobID in the {\tt knock.osol} file. 

\vskip 10pt

\noindent{\bf Step 5:} Test remote {\bf retrieve()} method. Get the result.

\begin{verbatim}
OSSolverService  -config ../data/configFiles/testRemoteRetrieve.config
\end{verbatim}
Before executing this command make sure to put your jobID into the file {\bf retrieve.osol }.  Also, either delete the {\tt -browser} option or put in the path to your browser.
The result of the optimization will be put into a file called {\bf test.osrl} that will be in the directory in which you are running the {\bf OSSolverService.} 

\end{itemize}



{\bf IMPORTANT:}  Please do NOT commit the changes to these config files. 


\item[5.] Test {\bf OSCommon}. Build the OSCommon library.  Build the {\bf OSCommon} library.  Do a {\bf make install}.   Then connect to {\bf apiExamples} directory, build and run the {\bf apiExample.}
\end{itemize}


\throwpage


\section{Wish List for Next Release}\label{section:wishlist} 

% This is a list of future work. As a stand-alone file it does not process in TeX or LaTeX,
% but it can be \input into the osUsersManual, and it is small enough so that it can be
% mainained by a simple text editor.
%
% This version dated 5 February 2010
%
%%%%%%%%%%%%%%%%%%%%%%%%%%%%%%%%%%%%%%%%%%%%%%%%%%%%%%%%%%%%%%%%%%%%%%%%%%%%%%%%%%%%%%%%%%%%

\begin{itemize}
\item Implement a Gurobi solver interface.

\item Implement the {\tt mult} and {\tt incr} features in OSInstance/OSiL parsers.

\item Implement the SOS feature in OSiL

\item Implement a switch so that the solver output is put into OSrL output. This output should go into a {\tt solutionResult} element in {\tt otherSolutionResults.}

\item Put the GAMS OSiL read and OSrL write into the OSModelingInterfaces

\item Implement the Bcp solver

\item Implement OS as part of CoinUtils. (That is, break out some of the basic routines.)

\item Write a document on how to hook your solver to OS

\item Add a module to FlopC++ that writes OSiL

\item Add a module to one or both of the Python modeling language (Pymo and/or Pulp) that writes OSiL

\item Add an OS option to the OSSolverInterfaces that allows the user to get the log file of the solver. The user would have to use the specific solver option to set the level of log output.

\item Investigate the Amazon cloud computing

\item Installer for Windows

\item Treat \url{https://projects.coin-or.org/OS/ticket/14} 

\item Figure out how to put the version number on the executables

\item Add code so that we can take a LINGO postfix problem instance and generate an OSExpression tree

\item Right now the OSSolverService command line parser requires / for path -- allow {$\backslash$} for Windows users

\item Prepare constraint programming document/report

\item Prepare a paper on OSOption and OSResult.

\item Write documentation on the new Java example 

\item Build and document Java distribution for users who want to build OSiL from Java and 
call OSSolverService from Java. 

\item Warmstart for LP

\item Paper on SOS

\item Vet/finalize SDPA and verify SDPA2OSiL

\item Implement parser for matrices and cones

\item Proof of Concept: Hook to a solver (CSDP?)

\item Paper on matrix programming

\item Re-check scenario formulation

\item Implement parser for scenarios

\item Implement solver (DE/decomposition)

\item Put in proper error checking for problems that have zero variables

\item Put in a detailed example of how to build a problem instance using the OSInstance API

\item Run Artistic Style on the code so Gus and Kipp are consistent

\item Solution files for matrix programming (Imre)

\item Complex numbers in OSiL (Imre)

\item Figure out why Che-lin's problem dies when finding a sparse Hessian

\item GAMS list -- SOS can be used with Cbc and Bonmin, semicontinuous+semiinteger
variables with Cbc, user defined functions with Ipopt and Bonmin,
parameters for Cbc, and - most important - you can redirect the output.

\item Ticket 14

\item get a log file from Cbc and put OSrL

\end{itemize}

%\begin{itemize}
%\item Implement a Gurobi solver interface.
%
%\item Implement the {\tt mult} and {\tt inc} features in OSiL.
%
%\item Implement the SOS feature in OSiL
%
%\item Implement a switch so that the solver output is put into OSrL output. This output should go into a {\tt solutionResult} element in {\tt otherSolutionResults.}
%
%\item Put the GAMS OSiL read and OSrL write into the OSModelingInterfaces
%
%\item Implement the Bcp solver
%
%\item Implement OS as part of CoinUtils. (That is, break out some of the basic routines.)
%
%\item Write a document on how to hook your solver to OS
%
%\item Add a module to FlopC++ that writes OSiL
%
%\item Add an OS option to the OSSolverInterfaces that allows the user to get the log file of the solver. The user would have to use the specific solver option to set the level of log output.
%
%\item Investigate the Amazon cloud computing
%
%\item Installer for Windows
%
%\item Documentation on the new Java example 
%
%\item  Building some sort of Java distribution for users who want to build OSiL from Java and call OSSolverService from Java. The demand for this may be nontrivial.  And then document this.
%
%\item  Treat \url{https://projects.coin-or.org/OS/ticket/14} 
%
%\item Figure out how to put the version number on the executables
%

%\item GAMS list -- SOS can be used with Cbc and Bonmin, semicontinuous+semiinteger
%variables with Cbc, user defined functions with Ipopt and Bonmin,
%parameters for Cbc, and - most important - you can redirect the output  ;-) .
%\end{itemize}

\begin{comment}
I like the message handling of Ipopt, where you can register several
journals that can handle output for different categories with differing
print levels. For the GAMSlinks we then have a GamsJournal that is used
to direct the output into the GAMS output channels.

Cbc and Osi-interfaced solvers use CoinMessageHandler. I do not really
like them because the message handler sometimes insert spaces or
newlines which make them very inflexible about what kind of output you
can print, but we have a GamsMessageHandler that does the job. Not all
Osi links support this yet. For Clp, Cbc, DyLP it should work. For the
trunk versions of Cpx, Msk, Xpr it should also work. Symphony I don't
remember.

For Bonmin and Couenne one has to pass in both an Ipopt::Journal and a
CoinMessageHandler.
\end{comment}

%\fi %--------------------------------------------------
 


\throwpage

\input{chapters/appendix.tex}


\addcontentsline{toc}{section}{Bibliography}
% \addcontentsline{toc}{section}{Bibliography}% alternative for article class
\bibliography{osUsersManual}
%\bibliographystyle{amsplain}
%\bibliography{kippbib}
 
\printindex
  
\throwpage

\input{chapters/Kipps_collected_wisdom.tex}

\throwpage

\input{chapters/buildSystemInfo.tex}
\end{document}
