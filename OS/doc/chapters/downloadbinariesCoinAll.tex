\section{Downloading the CoinAll Binaries}\label{section:obtainingbinaries}

The CoinAll project is actually a meta-project consisting of most of the COIN-OR solvers and supporting utility projects.  We describe how to download this project. 

%Below we describe different methods for obtaining the binaries and C++ source code.
Most users will only be interested in obtaining the binaries, which we describe  next.
%in Section~\ref{section:obtainingbinaries}. The remaining sections of this chapter deal with obtaining 
%the source code for the project, which will be of interest mostly to developers.
It is also possible to obtain the source code for the projects, which will be of interest mostly to developers. 
\ifdevelop
Details can be found in  Section~\ref{section:downloadsource}.
\else
If binaries are not provided for a particular operating system, it may be possible to build them from the source.
For details it is best to start reading the wiki page for the individual project or projects of interest.
\fi


%If the user does not wish to compile source code, the OS library, OSSolverService executable
%and Tomcat server software configuration are available in binary format for some operating systems.     
The repository of the binaries is at {\tt\UrlCoinAllDownload}\index{Downloading!binaries}.
%
\ifdevelop
 Unlike the source code described in Section~\ref{section:downloadwithsvn}, the binary files 
are not subject to version control and can be downloaded using an ordinary browser. 
%If binaries are not provided for a particular operating system,
%it may be possible to build them from the source code. Since the source is under version control, 
%this requires svn. (See Sections \ref{section:svn}, \ref{section:downloadwithsvn} and~\ref{section:build}.)
\fi

The binary distribution for the CoinAll library and executables follows the following naming convention:


\begin{verbatim}
CoinAll-version_number-platform-compiler-build_options.tgz (zip)
\end{verbatim}
For example, CoinAll  Release 1.6.0 compiled with the Intel 11.1 compiler on a 64 bit Windows system is:
\begin{verbatim}
CoinAll-1.6.0-win64-intel11.1.zip
\end{verbatim}



For more detail on the naming convention and examples see:

\medskip
\noindent{\tt\UrlCoinNames}
\medskip

After unpacking the {\tt tgz} or {\tt zip} archives, the following folders are available.
\begin{itemize}

\item[] {\bf bin --} this directory contains all the executables.

\item[] {\bf examples --} this directory contains several examples that illustrate working with 
the libraries. Some data files for working with the examples are also included.

\item[]  {\bf include --} the header files that are necessary in order to link against the various libraries.

\item[] {\bf lib --} the libraries that are necessary for creating applications that use the  libraries.


\item[] {\bf  share --} license and author information for all the projects used by the CoinAll project as well as a number of further data files of linear and integer programming problems.

\end{itemize}




